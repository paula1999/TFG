\chapter*{Summary}
%\addcontentsline{toc}{chapter}{Summary} 

\begin{otherlanguage}{english}

The main goal of this project is to present, study and implement a cryptosystem resistant to attacks by quantum computers, the McEliece cryptosystem.

\spacedlowsmallcaps{Chapter 1}

In this chapter we will present the previous mathematical concepts necessary to understand the subsequent development of this work. We will need to introduce concepts such as finite fields, since they are the algebraic structure on which we will work. We will also deal with the complexity of algorithms and problems, which will be useful to know if we can obtain a solution in polynomial time. Eventually, we will introduce the basic concepts of population-based metaheuristics to further develop genetic algorithms, which will be useful in attempting to break the security of McEliece's cryptosystem.

\spacedlowsmallcaps{Chapter 2}

In the second chapter we will study linear codes, defined as a subspace of a finite field. We will introduce two concepts that will be of great importance throughout the development of this work: the generating matrix and the parity matrix. Each of these matrices presents a unique code. We will also define some significant measures along with their properties, such as the Hamming distance or the Hamming weight. We will deal with the complexity of the problem of finding a word with minimum weight and related issues. We will finish by presenting the Brouwer-Zimmermann algorithm for calculating the distance of a code.

\spacedlowsmallcaps{Chapter 3}

In this chapter we will develop and implement a new class of linear error correcting codes, the Goppa codes. We will study the parity matrix associated with these codes and we will obtain from it the generating matrix, which will be useful for encoding the messages. We will also analyze the corrective capacity, which will depend on the degree of the polynomial on which the code is defined. In addition, we will develop the encoding and decoding process, including examples of the implementation carried out. For the decoding we will use an efficient algorithm, the Sugiyama algorithm, which is based on the Euclid algorithm and will allow us to recover the messages.

\newpage

\spacedlowsmallcaps{Chapter 4}

The fourth chapter contains the study of the goal of this project, the McEliece cryptosystem. Previously, we will introduce the concept and objectives of cryptography, as well as two main types of cryptosystems. Next, we will present the possibility of the existence of quantum computers and the threat that they present to the security of some existing cryptosystems, proposing some alternatives that are capable of resisting their attacks, such as the McEliece cryptosystem. This system is constructed from the Goppa codes studied in the previous chapter. We will study the generation of its keys (public and private), as well as the encryption and decryption processes. We will also analyze a variant, the Niederreiter system, which is equivalent in security. We will present examples of the McEliece system using the implementation carried out to show its operation.

\spacedlowsmallcaps{Chapter 5}

In this last chapter, we will develop a couple of attacks using genetic algorithms to break the security of McEliece's cryptosystem. These attacks are based on calculating the distance of a code from an echelon matrix of a matrix that generates the equivalent code, which will be obtained by swapping the code columns. We will study and implement adaptations of the GGA and CHC algorithms. Finally, we will analyze the results obtained after applying these algorithms to the McEliece system.

To carry out this project, the following has been implemented in SageMath.

\begin{itemize}
    \item A class to define the Goppa codes, another class for your encoder, and another class for your decoder using Sugiyama's algorithm. The documentation of this class can be found in the annex \ref{annex:sage-Goppa}, along with the helpful functions used.
    \item A class to define the McEliece cryptosystem, whose documentation can be found in the annex \ref{annex:sage-McEliece}, along with the helpful functions used.
    \item One function for the GGA genetic algorithm and another function for the CHC genetic algorithm. The documentation of these functions, along with other helpful functions used, can be found in the annex \ref{annex:sage-geneticos}.
\end{itemize}

\small{\spacedallcaps{Keywords:} coding theory, \; Goppa, \; criptography, \; post-quantum criptography, \; McEliece, \; Niederreiter, \; genetic algorithms, \; SageMath}

\end{otherlanguage}