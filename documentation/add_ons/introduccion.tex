\chapter*{Introducción y objetivos}
\addcontentsline{toc}{chapter}{Introducción y objetivos}

%TODO: poner aquí introducción y objetivos

\textcolor{red}{Añadir introducción y objetivos}


\begin{comment}

La introducción deberá:
- Contextualizar el trabajo explicando antecedentes importantes para el desarrollo realizado y efectuando, en su caso, un estudio de los progresos recientes.
- Describir el problema abordado, de forma que el lector tenga desde este momento una idea clara de la cuestión a resolver o del producto  a desarrollar y una  visión general de la solución alcanzada.
- Exponer con claridad las técnicas y áreas matemáticas, así como los conceptos y herramientas de la ingeniería informática que se han empleado.
- Sintetizar el contenido de la memoria.
- Citar las principales fuentes consultadas.

Objetivos del trabajo:
- En  este apartado deberán aparecer con claridad los objetivos inicialmente previstos en la propuesta de TFG y los finalmente alcanzados con indicación de  dificultades, cambios y mejoras respecto a la propuesta inicial. Si procede, es conveniente apuntar de manera precisalas interdependencias entre los distintos objetivos y conectarlos con los diferentes apartadosde la memoria. 
Se pueden destacar aquí los aspectos formativos previos más utilizados.
\end{comment}


\begin{comment}
    Comentario del profe de la primera línea:
    Esto es verdad. Pero en algunas ocasiones se nombre a Hamming como el "padre" de la teoría de códigos y a Shannon como el "padre" de la teoría de la información (algo más general). Esto es así porque en el artículo de Shannon se pone como ejemplo cómo corrige un código de Hamming, y Shannon mismo dice que es una construcción de Richard Hamming (creo que se publica después, en 1950). También Golay tiene una nota sobre el tema en 1949. Busca el artículo de Hamming y verás algo más parecido a lo que tratamos aquí. Además, si te lees bien los artículos de Shannon y Hamming te será fácil escribir una introducción para este capítulo, o para la memoria.

    Yo:
    
    El inicio de la teoría de códigos surgió a partir de la publicación de Claude Shannon sobre ``Una teoría matemática sobre la comunicación"\ en 1948 \cite{Shannon_1948}. En este artículo, Shannon explica que es posible transmitir mensajes fiables en un canal de comunicación que puede corromper la información enviada a través de él siempre y cuando no se supere la capacidad de dicho canal.
    
    Con la teoría de códigos, podemos codificar datos antes de transmitirlos de tal forma que los datos alterados puedan ser decodificados al grado de precisión especificado. Así, el principal problema es determinar el mensaje que fue enviado a partir del recibido. El Teorema de Shannon nos garantiza que el mensaje recibido coincidirá con el que fue enviado un cierto porcentaje de las veces. Esto hace que el objetivo de la teoría de códigos sea crear códigos que cumplan las condiciones de este teorema.
\end{comment}