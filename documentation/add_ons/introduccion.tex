\chapter*{Introducción y objetivos}
\addcontentsline{toc}{chapter}{Introducción y objetivos}

%TODO: poner aquí introducción y objetivos

\textcolor{red}{Añadir introducción y objetivos}


\begin{comment}

La introducción deberá:
- Contextualizar el trabajo explicando antecedentes importantes para el desarrollo realizado y efectuando, en su caso, un estudio de los progresos recientes.
- Describir el problema abordado, de forma que el lector tenga desde este momento una idea clara de la cuestión a resolver o del producto  a desarrollar y una  visión general de la solución alcanzada.
- Exponer con claridad las técnicas y áreas matemáticas, así como los conceptos y herramientas de la ingeniería informática que se han empleado.
- Sintetizar el contenido de la memoria.
- Citar las principales fuentes consultadas.

Objetivos del trabajo:
- En  este apartado deberán aparecer con claridad los objetivos inicialmente previstos en la propuesta de TFG y los finalmente alcanzados con indicación de  dificultades, cambios y mejoras respecto a la propuesta inicial. Si procede, es conveniente apuntar de manera precisalas interdependencias entre los distintos objetivos y conectarlos con los diferentes apartadosde la memoria. 
Se pueden destacar aquí los aspectos formativos previos más utilizados.
\end{comment}

