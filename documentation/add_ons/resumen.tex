\chapter*{Resumen}
%\addcontentsline{toc}{chapter}{Resumen} 

La posible existencia de ordenadores cuánticos amenaza la seguridad de los sistemas criptográficos más conocidos y usados en la actualidad, como RSA. En este trabajo se propone, se estudia y se implementa un criptosistema resistente a estos ataques, el criptosistema de McEliece, junto a una variante, el criptosistema de Niederreiter. Ambos sistemas se construyen a partir de los códigos de Goppa, una clase de códigos de corrección de errores lineales que también desarrollaremos e implementaremos. Estos códigos permiten codificar mensajes y decodificarlos mediante un algoritmo eficiente, el algoritmo de Sugiyama. Previamente, introduciremos y presentaremos los conceptos necesarios para poder abordar el objetivo de este trabajo. Por último, se proponen ataques usando algoritmos genéticos para vulnerar la seguridad del criptosistema de McEliece.

\small{\spacedallcaps{Palabras clave:} teoría de códigos, \; Goppa, \; criptografía, \; criptografía post-cuántica, \; McEliece, \; Niederreiter, \; algoritmos genéticos, \; SageMath}

\newpage