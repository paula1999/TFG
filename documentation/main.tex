% Plantilla para un Trabajo Fin de Grado de la Universidad de Granada,
% adaptada para el Doble Grado en Ingeniería Informática y Matemáticas.
%
%  Autor: Mario Román.
%  Licencia: GNU GPLv2.
%
% Esta plantilla es una adaptación al castellano de la plantilla
% classicthesis de André Miede, que puede obtenerse en:
%  https://ctan.org/tex-archive/macros/latex/contrib/classicthesis?lang=en
% La plantilla original se licencia en GNU GPLv2.
%
% Esta plantilla usa símbolos de la Universidad de Granada sujetos a la normativa
% de identidad visual corporativa, que puede encontrarse en:
% http://secretariageneral.ugr.es/pages/ivc/normativa
%
% La compilación se realiza con las siguientes instrucciones:
%   pdflatex --shell-escape main.tex
%   bibtex main
%   pdflatex --shell-escape main.tex
%   pdflatex --shell-escape main.tex

% Opciones del tipo de documento
\documentclass[oneside,openright,titlepage,numbers=noenddot,openany,headinclude,footinclude=true,
cleardoublepage=empty,abstractoff,BCOR=5mm,paper=a4,fontsize=12pt,main=spanish]{scrreprt}

% Paquetes de latex que se cargan al inicio. Cubren la entrada de
% texto, gráficos, código fuente y símbolofs.
\usepackage[utf8]{inputenc}
\usepackage[T1]{fontenc}
\usepackage{fixltx2e}
\usepackage{graphicx} % Inclusión de imágenes.
\usepackage{grffile}  % Distintos formatos para imágenes.
\usepackage{longtable} % Tablas multipágina.
\usepackage{wrapfig} % Coloca texto alrededor de una figura.
\usepackage{rotating}
\usepackage[normalem]{ulem}
\usepackage{amsmath}
\usepackage{dsfont}
\usepackage{textcomp}
\usepackage{amssymb}
\usepackage{capt-of}
\usepackage[colorlinks=true]{hyperref}
\usepackage{tikz} % Diagramas conmutativos.
\usepackage{minted} % Código fuente.
\usepackage[numbers]{natbib}
\usepackage{booktabs} % Para tablas de figuras (ejemplo en salesman)
\usepackage{listings} % Código
\usepackage{comment}

\lstset{
	basicstyle=\ttfamily,%
	breaklines=true,%
	captionpos=b,                    % sets the caption-position to bottom
  tabsize=2,	                   % sets default tabsize to 2 spaces
  frame=none,
  numbers=left,
  xleftmargin=18pt,
  stepnumber=1,
  aboveskip=12pt,
  showstringspaces=false,
  keywordstyle=\bfseries,
  commentstyle=\itshape,
  numberstyle=\scriptsize\bfseries,
  %morekeywords={sage},
}

% Operadores

\DeclareMathOperator{\gr}{gr}
\DeclareMathOperator{\mcd}{mcd}

\lstset{ frame=Ltb,
framerule=0pt,
aboveskip=0.5cm,
framextopmargin=3pt,
framexbottommargin=3pt,
framexleftmargin=0.4cm,
framesep=0pt,
rulesep=.4pt,
backgroundcolor=\color{white},
rulesepcolor=\color{black},
%
stringstyle=\ttfamily,
showstringspaces = false,
basicstyle=\small\ttfamily,
commentstyle=\color{gray},
keywordstyle=\bfseries,
%
numbers=left,
numbersep=15pt,
numberstyle=\tiny,
numberfirstline = false,
breaklines=true,
}



\lstdefinestyle{Python}
{language=Python,
}


% Plantilla classicthesis
\usepackage[beramono,eulerchapternumbers,linedheaders,parts,a5paper,dottedtoc,
manychapters,pdfspacing]{classicthesis}

% Geometría y espaciado de párrafos.
\setcounter{secnumdepth}{0}
\usepackage{enumitem}
\setitemize{noitemsep,topsep=0pt,parsep=0pt,partopsep=0pt}
\setlist[enumerate]{topsep=0pt,itemsep=-1ex,partopsep=1ex,parsep=1ex}
\usepackage[top=1in, bottom=1.5in, left=1in, right=1in]{geometry}
\setlength\itemsep{0em}
\setlength{\parindent}{0pt}
\usepackage{parskip}
\usepackage{setspace}

% Profundidad de la tabla de contenidos.
\setcounter{secnumdepth}{3}

% Usa el paquete minted para mostrar trozos de código.
% Pueden seleccionarse el lenguaje apropiado y el estilo del código.
\usepackage{minted}
\usemintedstyle{colorful}
\setminted{fontsize=\small}
\setminted[haskell]{linenos=false,fontsize=\small}
\renewcommand{\theFancyVerbLine}{\sffamily\textcolor[rgb]{0.5,0.5,1.0}{\oldstylenums{\arabic{FancyVerbLine}}}}

% Path para las imágenes
\graphicspath{{figures/}}

% Archivos de configuración.
%------------------------
% Bibliotecas para matemáticas de latex
%------------------------
\usepackage{amsthm}
\usepackage{amsmath}
\usepackage{tikz}
\usepackage{tikz-cd}
\usetikzlibrary{shapes,fit}
\usepackage{bussproofs}
\EnableBpAbbreviations{}
\usepackage{mathtools}
\usepackage{scalerel}
\usepackage{stmaryrd}


%------------------------
% Estilos para los teoremas
%------------------------
\theoremstyle{plain}
\newtheorem{theorem}{Teorema}
\newtheorem{proposition}{Proposición}
\newtheorem{lemma}{Lema}
\newtheorem{corollary}{Corolario}

\theoremstyle{definition}
\newtheorem{definition}{Definición}

% Change the proof style so it's in English and add \qed at the end.
\renewenvironment{proof}{{\bfseries Demostración.}}{\qed}

\theoremstyle{remark}
\newtheorem{remark}{Observación}
\newtheorem{exampleth}{Ejemplo}

% New style for postulates so they are tabulated
\makeatletter
\newtheoremstyle{indented}
	{3pt}% space before
	{3pt}% space after
	{\addtolength{\@totalleftmargin}{3.5em}
		\addtolength{\linewidth}{-3.5em}
		\parshape 1 3.5em \linewidth}% body font
	{}% indent
	{\bfseries}% header font
	{.}% punctuation
	{.5em}% after theorem header
	{}% header specification (empty for default)
\makeatother

% Apply the new style
\theoremstyle{indented}
\newtheorem{postulate}{Postulate}
\newtheorem*{postulate 3'}{Postulate 3'}
\newtheorem*{postulate 2'}{Projective Measurement}

%------------------------
% Macros
% ------------------------

\newcommand*{\B}{\mathbb{B}}
\newcommand*{\C}{\mathds{C}}
\newcommand*{\R}{\mathbb{R}}
\newcommand*{\ra}{\rangle}
\newcommand*{\la}{\langle}

% Para poner sonrisa sobre puntos suspensivos. Uso: \overplace{n}{\dotsc}
\newcommand{\overplace}[2]{%
	\overset{\substack{#1\\\smile}}{#2}%
}

% Para añadir un salto de línea al nuevo párrafo. Use: \newparagraph
\newcommand{\newparagraph}[1]{\paragraph{#1}\mbox{}\\}  % En macros.tex se almacenan las opciones y comandos para escribir matemáticas.
\input{imports/classicthesis-config} % En classicthesis-config.tex se almacenan las opciones propias de la plantilla.

% Color institucional UGR
% \definecolor{ugrColor}{HTML}{ed1c3e} % Versión clara.
\definecolor{ugrColor}{HTML}{c6474b}  % Usado en el título.
\definecolor{ugrColor2}{HTML}{c6474b} % Usado en las secciones.

% Datos de portada
\usepackage{titling} % Facilita los datos de la portada
\author{Paula Villanueva Núñez} 
\title{Criptoanálisis del Criptosistema \\ de McEliece clásico \\ mediante algoritmos genéticos}

% Portada
\usepackage{datetime}
\renewcommand\maketitle{
  \begin{titlepage}
    \begin{addmargin}[-2.5cm]{-3cm}
      \begin{center}
        \normalsize  
        \hfill
        \vfill
        \vspace{1.8cm}

        \spacedallcaps{TRABAJO FIN DE GRADO} \\ \medskip 
        \small{\spacedallcaps{Doble grado en Ingeniería Informática y Matemáticas}} \\  \bigskip\bigskip
        \vspace{1.5cm}

        \begingroup
        \LARGE{\color{ugrColor}\spacedallcaps{\textbf{\thetitle}}} \\ \bigskip
        \endgroup

        \vfill
        \vspace{1.5cm}

        \textbf{Autor}\\
        \spacedlowsmallcaps{\theauthor}

        \textbf{Director}\\
        \spacedlowsmallcaps{Gabriel Navarro Garulo} \\  \bigskip

        \hfill
        \vspace{1.7cm}

        \includegraphics[scale=0.18]{figures/fciencias_logo.png} 
        \quad
        \includegraphics{figures/etsiit_logo.png}


        \spacedlowsmallcaps{Facultad de Ciencias} \\
        \spacedlowsmallcaps{E.T.S. de Ingenierías Informática y de Telecomunicación} \\ \medskip
        \vspace{1.0cm}
        %\textit{Curso académico 2021-2022}
        \textit{Granada, a \today}

        \vfill                      

      \end{center}  
    \end{addmargin}       
  \end{titlepage}}
\usepackage{wallpaper}
\usepackage[main=spanish]{babel}


\usepackage{slantsc}

% Algoritmos

\usepackage[linesnumbered, onelanguage, ruled]{algorithm2e}
\SetAlCapFnt{\sffamily}
\SetAlCapNameFnt{\sffamily}
\newcommand\mycommentfont[1]{\sffamily\textcolor{darkgray}{#1}}
\SetCommentSty{mycommentfont}
%\SetAlFnt{\sffamily}
%\SetKwSty{sffamily}
\SetAlCapSkip{.5em}
\SetAlgoCaptionLayout{raggedright}
\SetAlgorithmName{Algoritmo}{Algoritmo}{Lista de algoritmos}
%\SetAlgoSkip{medskip}
\SetAlgoInsideSkip{medskip}

% Entorno algoritmo con líneas pero caption debajo
\makeatletter
\newenvironment{Ualgorithm}[1][htpb]{\def\@algocf@post@ruled{\color{gray}\hrule  height\algoheightrule\kern6pt\relax}%
\def\@algocf@capt@ruled{under}%
\setlength\algotitleheightrule{0pt}%
\SetAlgoCaptionLayout{raggedright}%
\begin{algorithm}[#1]}
{\end{algorithm}}
\makeatother

% Comandos
\theoremstyle{plain}
\newtheorem{problemth}{Problema}

\begin{document}

\ThisULCornerWallPaper{1}{figures/ugrA4.pdf}
\maketitle

% TODO: dedicatoria
\newpage

\thispagestyle{empty}
\phantomsection

\hfill
\vfill

\begin{flushright}
	\itshape
	% TODO: poner aquí la dedicatoria
\end{flushright}

\vfill

\endinput
\newpage

\onehalfspacing

\chapter*{Resumen}
\addcontentsline{toc}{chapter}{Resumen} 


%TODO

\small{\spacedallcaps{Palabras clave:}}

\newpage
\null
\thispagestyle{empty}

\chapter*{Summary}
\addcontentsline{toc}{chapter}{Summary} 

%TODO:
Añadir resumen en inglés

\small{\spacedallcaps{Keywords:}}

%\newpage
%\null
%\thispagestyle{empty}

% Indice
\tableofcontents

% TODO: Introducción
\chapter*{Introducción y objetivos}
\addcontentsline{toc}{chapter}{Introducción}

% TODO: poner aquí introducción y objetivos
Añadir introducción


\newpage
\null
\thispagestyle{empty}

% 0. Preliminares. 
% Esto es voluntario. 
% Consiste en poner lo que ya sabías, porque lo has dado en el grado, y que te haga falta 
% (por ejemplo, algebra lineal, cuerpos finitos, operaciones con polinomios, algoritmos genéticos, etc).

% TODO

\chapter{Preliminares}

En este capítulo se desarrollarán las herramientas necesarias para poder afrontar el criptosistema de McEliece que precisa este trabajo. Se abordarán conceptos relacionados con el álgebra lineal, anillos, cuerpos finitos, polinomios, algoritmos genéticos, etc.

\section{Anillos}

En esta sección introduciremos el concepto de anillo para poder definir el concepto de cuerpo, así como las principales propiedades de esta estructura algebraica.

\begin{definition}
    Un \emph{anillo} $(A, +, \cdot )$ es un conjunto $A$ junto con dos operaciones binarias $A \times A \rightarrow A$ denotadas por la suma (denotada por $+$) y producto (denotado por $\cdot$) que verifican los siguientes axiomas:

    \begin{itemize}
        \item Propiedad asociativa de la suma: 
        $$  a + (b + c) = (a + b) + c \qquad \forall a,b,c \in A$$
        \item Existencia del elemento neutro para la suma:
        $$ 0 + a = a = a + 0 \qquad \forall a \in A$$
        \item Existencia del elemento inverso para la suma:
        $$ \forall a \in A \; \; \exists -a \in A \quad a + (-a) = 0 = (-a) + a $$
        \item Propiedad conmutativa de la suma:
        $$ a + b = b + a \qquad \forall a,b \in A $$
        \item Propiedad asociativa del producto:
        $$ a \cdot (b \cdot c) = (a \cdot b) \cdot c \qquad \forall a,b,c \in A $$
        \item Propiedad distributiva del producto:
        $$ a \cdot (b + c) = a \cdot b + a \cdot c, \quad (b + c) \cdot a = b \cdot a + c \cdot a \qquad \forall a,b,c \in A $$
        
    \end{itemize}

    Un anillo de llama \emph{conmutativo o abeliano} si se verifica la propiedad conmutativa del producto 
    $$ ab = ba \qquad \forall a,b \in A $$
\end{definition}

Es fácil comprobar que en todo anillo se verifica $0 \cdot x = x \cdot 0 = 0$ (obsérvese que $0 \cdot = (0 + 0) \cdot x = 0 \cdot x + 0 \cdot x$ y simplifíquese por el simétrico de $0 \cdot x$).

Además del anillo conmutativo, existen otros casos particulares de anillos. Diremos que un anillo es \emph{unitario} si es un anillo cuyo producto tiene elemento neutro, esto es, $\exists 1 \in A \; : \; x \cdot 1 = 1 \cdot x = x$, para todo $x \in A$.

Diremos que un elemento $a$ del anillo $A$ es \emph{invertible} si existe un elemento $a'$ en el anillo $A$ tal que $a \cdot a' = a' \cdot a = 1$. Este elemento $a'$ es único, lo llamaremos \emph{elemento inverso} y lo denotaremos por $a^{-1}$.

Cuando se da la igualdad $1 = 0$, diremos que el anillo es \emph{trivial} y tendrá un solo elemento.

\begin{definition}
    Sea $A$ un anillo, $1 \in A$ el elemento neutro del producto y $n \geq 1$ un número natural, definimos la cardinalidad de $A$ como:

    \[
        Car(A) = \left\{ \begin{array}{lc}
        0 &   \textit{si } n \cdot 1 \neq 0 \textit{ para cualquier } n \geq 1 \\
        \\ n & \textit{si n es el menor número natural no nulo para el que } n \cdot 1 = 0
        \end{array}
        \right.
    \]
\end{definition}

%TODO
% - Añadir homomorfismo de anillos? (apuntes de álgebra I)
% - Añadir subestructuras de anillos e ideales? (apuntes de álgebra I)
% - Añadir propiedades de los dominios de integridad? (apuntes de álgebra I)

\section{Cuerpos finitos}

Para presentar el concepto de cuerpo finito, necesitaremos definir previamente el concepto de cuerpo junto con algunas de sus propiedades más relevantes.

\begin{definition}
    Un \emph{cuerpo} $(K, +, \cdot)$ es un anillo conmutativo no trivial en el que todo elemento no nulo tiene un inverso multiplicativo. Se dice que un cuerpo es \emph{finito} si tiene un número finito de elementos.
\end{definition}

Sea $(K, +, \cdot)$ un cuerpo y $E \subset K$, diremos que $E$ es un \emph{subcuerpo} de $K$ o $K$ es una \emph{extensión} de $E$ si se cumple que $(E, +, \cdot)$ es un cuerpo cuando las operaciones $+$ y $\cdot$ se restringen a $E$.

Diremos que la \emph{característica} de un cuerpo es el número de elementos que tiene.

Todos los cuerpos finitos tienen un número de elementos $q = p^n$, para algún número primo $p$ y algún entero positivo $n$. Denotaremos por $\mathbb{F}_q$ a los cuerpos finitos con característica $q$, aunque otra común notación es $GF(q)$.

Observemos que si $p$ un número primo y $q$ es un número entero tal que $q = p^n$, entonces $\mathbb{F}_q$ es un espacio vectorial sobre $\mathbb{F}_p$ de dimensión $n$. Además, hay $q$ vectores en el espacio vectorial de dimensión $n$ sobre $\mathbb{F}_p$.

Notemos que todos los cuerpos finitos de orden $q$ son isomorfos, aunque cada cuerpo puede tener diferentes representaciones.

\begin{proposition}
    Sea $\mathbb{F}_q$ un cuerpo finito con $q = p^n$ elementos, entonces 

    $$p \cdot \alpha = 0, \qquad \forall \alpha \in \mathbb{F}_q.$$
\end{proposition}

\begin{proposition}
    Sea $\mathbb{F}_q$ un cuerpo finito con característica $p$, se cumple que

    $$( \alpha + \beta )^p = \alpha^p + \beta^p, \qquad \forall \alpha, \beta \in \mathbb{F}_q.$$
\end{proposition}

% TODO: homomorfismo de cuerpos


\subsection{Anillo de polinomios sobre cuerpos finitos}

En esta sección vamos a introducir el concepto de polinomio junto con sus operaciones.

\begin{definition}
    Sea $A$ un anillo conmutativo. El \emph{conjunto de polinomios} en la variable $x$ con coeficientes en $A$ está compuesto por el siguiente conjunto
    $$\{ a_n x^n  + a_{n-1} x^{n-1} + \cdots + a_1 x + a_0 \; : \; a_0, ..., a_n \in A \}.$$
    Este conjunto se representa por $A[X]$.
\end{definition}

En el conjunto de polinomios definimos una suma y un producto. 

Sean $f = a_n x^n + a_{n-1} x^{n-1} + \cdots + a_1 x + a_0$ y $g = b_m x^m + b_{m-1} x^{m-1} + \cdots + b_1 x + b_0$ dos polinomios. Supongamos que $m \leq n$, tomando $b_i = 0$ para todo $n \geq i > m$, definimos las operaciones de suma y producto de polinomios:

$$f + g = (a_n + b_n)x^n + (a_{n-1} + b_{n-1})x^{n-1} + \cdots + (a_1 + b_1)x + (a_0 + b_0).$$
$$f \cdot g = a_n b_m x^{n+m} + (a_n b_{m+1} + a_{n-1} b_m) x^{n+m-1} + \cdots + (a_1 b_0 + a_0 b_1)x + a_0 b_0.$$

De esta forma, diremos que el conjunto $A[X]$ con las operaciones anteriores es un \emph{anillo de polinomios en X con coeficientes en A}.

\begin{definition}
    Para un polinomio $f = a_n + a_{n-1} x^{n-1} + \cdots + a_1 x + a_0 \neq 0$ el mayor índice $n$ tal que $a_n \neq 0$ se llama \emph{grado de f} y se representa por $\gr(f)$. Si $f = 0$ definimos $\gr(f) = - \infty$.

    Llamaremos \emph{término (de grado i)} a cada uno de los sumandos $a_i X^i$ del polinomio $f$. El \emph{término líder} es el término no nulo de mayor grado. El coeficiente $a_n \neq 0$ del término líder se llama \emph{coeficiente líder} y el término de grado cero $a_0$ se llama \emph{término constante}. Si el coeficiente líder es $1$, diremos que el polinomio es \emph{mónico}.
\end{definition}

A continuación tenemos algunas propiedades de los polinomios.

\begin{proposition}
    Sea $A$ un anillo conmutativo y sean $f,g \in A[X]$ dos polinomios, tenemos que 
    $$\gr(f + g) \leq \max{(\gr(f), \; \gr(g))},$$
    $$\gr(f \cdot g) \leq \gr(f) + \gr(g)$$
    Si $\gr(f) \neq \gr(g)$, se verifica 
    $$\gr(f + g) = \max{(\gr(f), \; \gr(g))}$$
\end{proposition}

Podemos trasladar estos resultados al caso de los cuerpos finitos. Sea $\mathbb{F}_q$ un cuerpo finito, un polinomio $f(x)$ estará definido sobre dicho cuerpo si es de la forma $f(x) = \sum_{i = 0}^{n} a_i \cdot x^i$, donde $a_i \in \mathbb{F}_q$ para todo $i = 0, ..., n$. Análogamente, diremos que $f(x) \in \mathbb{F}_q$.

Sean $f(x)$ y $g(x)$ polinomios en $\mathbb{F}_q[x]$, diremos que $f(x)$ divide a $g(x)$ si existe un polinomio $h(x) \in \mathbb{F}_q[x]$ tal que $g(x) = f(x) h(x)$ y lo denotaremos por $f(x) \vert g(x)$. El polinomio $f(x)$ se llama \emph{divisor} o \emph{factor} de $g(x)$.

El \emph{mayor común divisor} de $f(x)$ y $g(x)$ es el polinomio mónico en $\mathbb{F}_q[x]$ con mayor grado que divide a $f(x)$ y a $g(x)$. Este polinomio es único y se denota por $\mcd(f(x), \; g(x))$. Diremos que los polinomios $f(x)$ y $g(x)$ son \emph{primos relativos} si $\mcd(f(x), \; g(x)) = 1$.

El siguiente resultado es de gran utilidad, pues sirve para calcular los divisores de un polinomio e incluso para calcular el máximo común divisor. Nos dará las bases para definir posteriormente el Algoritmo de Euclides.

\begin{theorem}
    \label{th:div_alg}
    Sean $f(x)$ y $g(x)$ polinomios en $\mathbb{F}_q[x]$ con $g(x)$ no nulo.
    \begin{itemize}
        \item Existen dos polinomios únicos $c(x), r(x) \in \mathbb{F}_q[x]$ tales que
        $$f(x) = g(x) c(x) + r(x), \qquad \textit{donde } \gr(r(x)) < \gr(g(x)) \textit{ o } r(x) = 0.$$
        \item Si $f(x) = g(x) c(x) + r(x)$, entonces $\mcd(f(x), \; g(x)) = \mcd(g(x), \; r(x))$.
    \end{itemize}
\end{theorem}

Los polinomios $c(x)$ y $r(x)$ se llaman \emph{cociente} y \emph{resto}, respectivamente.

Usando este resultado de forma recursiva, obtendremos el máximo común divisor de los polinomios $f(x)$ y $g(x)$. Este procedimiento se conoce como \emph{Algoritmo de Euclides}. El siguiente resultado describe este algoritmo.

\begin{theorem}[Algoritmo de Euclides]
    Sean $f(x)$ y $g(x)$ polinomios definidos en $\mathbb{F}_q[x]$ con $g(x)$ no nulo.
    \begin{enumerate}
        \item Realizar los siguientes pasos hasta que $r_n(x) = 0$ para algún $n$:
        $$f(x) = g(x) c_1(x) + r_1(x), \qquad \textit{donde } \gr(r_1(x)) < \gr(x),$$
        $$g(x) = r_1(x) c_2(x) + r_2(x), \qquad \textit{donde } \gr(r_2(x)) < \gr(r_1),$$
        $$r_1(x) = r_2(x) c_3(x) + r_3(x), \qquad \textit{donde } \gr(r_3(x)) < \gr(r_2),$$
        $$\vdots$$
        $$r_{n-3}(x) = r_{n-2}(x) c_{n-1}(x) + r_{n-1}(x), \qquad \textit{donde } \gr(r_{n-1}(x)) < \gr(r_{n-2}),$$
        $$r_{n-2}(x) = r_{n-1}(x) c_{n}(x) + r_{n}(x), \qquad \textit{donde } r_n(x) = 0.$$
        Entonces $\mcd(f(x), \; g(x)) = cr_{n-1}(x)$, donde $c \in \mathbb{F}_q$ es una constante para que $c r_{n-1}(x)$ sea mónico.
        \item Existen polinomios $a(x), b(x) \in \mathbb{F}_q[x]$ tales que 
        $$a(x) f(x) + b(x) g(x) = \mcd(f(x), \; g(x)).$$
    \end{enumerate}
\end{theorem}

En cada paso el grado del resto se decrementa al menos en $1$, por lo que podemos asegurar que la secuencia de pasos anterior terminará en algún momento.

A continuación se muestran algunos resultados relevantes.

\begin{proposition}
    Sean $f(x)$ y $g(x)$ polinomios en $\mathbb{F}_q[x]$.
    \begin{itemize}
        \item Si $k(x)$ es un divisor de $f(x)$ y $g(x)$, entonces $k(x)$ es un divisor de $a(x) f(x) + b(x) g(x)$ para algunos $a(x), b(x) \in \mathbb{F}_q[x]$.
        \item Si $k(x)$ es un divisor de $f(x)$ y $g(x)$, entonces $k(x)$ es un divisor de $\mcd(f(x), \; g(x))$.
    \end{itemize}
\end{proposition}

\begin{proposition}
    Sea $f(x)$ un polinomio en $\mathbb{F}_q[x]$ de grado $n$.
    \begin{itemize}
        \item Si $\alpha \in \mathbb{F}_q$ es una raíz de $f(x)$, entonces $x - \alpha$ es un factor de $f(x)$.
        \item El polinomio $f(x)$ tiene como mucho $n$ raíces en cualquier cuerpo que contenga a $\mathbb{F}_q$.
    \end{itemize}
\end{proposition}

\begin{theorem}
    Los elementos de $\mathbb{F}_q$ son las raíces de $x^q - x$.
\end{theorem}

\subsubsection{Construcción de cuerpos finitos}

Para realizar la construcción de cuerpos finitos, previamente necesitaremos conocer el siguiente concepto y algunos resultados relacionados.

\begin{definition}
    Sea $f(x) \in \mathbb{F}_q[x]$ un polinomio no constante, decimos que es \emph{irreducible} sobre $\mathbb{F}_q$ si no se factoriza como producto de dos polinomios en $\mathbb{F}_q[x]$ de menor grado.
\end{definition}

\begin{theorem}
    Sea $f(x)$ un polinomio no constante. Entonces

    $$f(x) = p_1(x)^{a_1} \cdots p_k(x)^{a_k},$$

    donde cada $p_i(x)$ es irreducible y único salvo orden, y los elementos $a_i$ son únicos.
\end{theorem}

Como consecuencia de este resultado, tenemos que $\mathbb{F}_q[x]$ es un \emph{dominio de factorización única}.

El siguiente resultado nos muestra cómo construir un cuerpo finito de característica $p$ a partir del cociente de anillos de polinomios por polinomios irreducibles.

\begin{proposition}
    Sea $p$ un número primo y sea el polinomio $f(x) \in \mathbb{F}_p[x]$ irreducible en $\mathbb{F}_p$ y de grado $m$. Tenemos que el anillo cociente $\mathbb{F}_q[x]/\left(f(x)\right)$ es un cuerpo finito con $q = p^m$ elementos, es decir, con característica $p$.
\end{proposition}

Escribiremos los elementos del anillo cociente, que son las clases laterales $g(x) + (f(x))$ como vectores en $\mathbb{F}_p^m$ con la siguiente correspondencia:

\begin{equation}
    \label{pr:correspondencia}
    g_{m-1} x^{m-1} + g_{m-2} x^{m-2} + \cdots + g_{1} x + g_0 + (f(x)) \leftrightarrow (g_{m-1}, g_{m-2}, ..., g_1, g_0).
\end{equation}

Esta notación facilita la operación de sumar dos elementos. Sin embargo, la multiplicación es algo más complicada. Supongamos que queremos multiplicar $g_1(x) + (f(x))$ por $g_2(x) + (f(x))$. Para ello, usaremos el resultado \ref{th:div_alg} para obtener

\begin{equation}
    \label{pr:multiplicacion_clase_lateral}
    g_1(x) g_2(x) = f(x) h(x) + r(x),
\end{equation}

donde $\gr(r(x)) \leq m - 1$ o $r(x) = 0$. Entonces 

$$(g_1(x) + (f(x))) (g_2(x) + (f(x))) = r(x) + (f(x)).$$

Podemos simplificar la notación si reemplazamos $x$ por $\alpha$ donde $f(\alpha) = 0$. Por \ref{pr:multiplicacion_clase_lateral}, se cumple que $g_1(\alpha) g_2(\alpha) = r(\alpha)$ y la correspondencia \ref{pr:correspondencia} queda como sigue

$$g_{m-1} g_{m-2} \cdots g_1 g_0 \leftrightarrow  g_{m-1} \alpha^{m-1} g_{m-2} \alpha^{m-2}\cdots g_1 \alpha g_0.$$

De esta forma, multiplicamos los polinomios en $\alpha$ de forma usual y aplicamos al resultado que $f(\alpha) = 0$ para reducir las potencias de $\alpha$ mayores que $m-1$ a polinomios en $\alpha$ de grado menor que $m$.

\section{Algoritmos genéticos}

%TODO
Añadir introducción a la sección: qué es y para qué sirve (tratar problemas que no son P, o que no se sabe que sean P). Didáctica!!!!

Los \emph{algoritmos genéticos} son algoritmos de optimización, búsqueda y aprendizaje inspirados en los procesos de evolución 
natural y evolución genética.

% TODO: esquema del ciclo de la evolución
En general, los algoritmos genéticos siguen el siguiente procedimiento (explicarlo mejor).

\begin{lstlisting}
    t = 0
    inicializar la poblacion P(t)
    evaluar la poblacion P(t)
    Mientras (no se cumpla la condicion de parada) hacer 
        t = t + 1
        seleccionar P' desde P(t-1)
        recombinar P'
        mutar P'
        reemplazar P(t) a partir de P(t-1) y P'
        evaluar P(t)
\end{lstlisting}

Existen dos modelos de algoritmos genéticos, el modelo generacional y el modelo estacionario.

En el modelo generacional, durante cada iteración se crea una población completa con nuevos individuos.
Así, la nueva población reemplaza directamente a la antigua.

En el modelo estacionario, durante cada iteración se escogen dos padres de la población y se les aplican los operadores genéticos.
De este modo, los descendientes reemplazan a los cromosomas de la población anterior.
Este modelo es elitista y produce una convergencia rápida cuando se reemplazan los peores cromosomas de la población.

%TODO
Añadir más cosas


\section{Clases de complejidad}

Explicar que es un problema NP-completo y demás.
% 1. Introducción a la teoría de códigos lineales. 
% Aquí cuanto más quieras leer del libro de Huffman y Pless, mejor. 
% Necesario sólo es el capítulo 1 y el 3 (no todo del 3). 
% Y, aunque no los utilicemos, puedes leerte el capítulo de códigos cíclicos. 
% Es importante centrarse en lo que es la distancia, y que es un problema NP-completo (archivo VARDY.pdf)) ya que es lo que le da seguridad a los criptosistemas.

\chapter{Introducción a la teoría de códigos lineales}

% TODO: reescribir introducción

\textcolor{red}{Hay que reescribir este párrafo}

El inicio de la teoría de códigos surgió a partir de la publicación de Claude Shannon sobre ``Una teoría matemática sobre la comunicación"\ en 1948 \cite{Shannon_1948}. En este artículo, Shannon explica que es posible transmitir mensajes fiables en un canal de comunicación que puede corromper la información enviada a través de él siempre y cuando no se supere la capacidad de dicho canal.

Con la teoría de códigos, podemos codificar datos antes de transmitirlos de tal forma que los datos alterados puedan ser decodificados al grado de precisión especificado. Así, el principal problema es determinar el mensaje que fue enviado a partir del recibido. El Teorema de Shannon nos garantiza que el mensaje recibido coincidirá con el que fue enviado un cierto porcentaje de las veces. Esto hace que el objetivo de la teoría de códigos sea crear códigos que cumplan las condiciones de este teorema.

En esta sección introduciremos los conceptos y resultados fundamentales sobre la teoría de códigos lineales. El desarrollo de este capítulo se ha basado en \cite{Huffman_Pless_2010}, \cite{Vardy_1997}, \cite{Wassermann_2006} y \cite{Podesta_2006}.

\section{Introducción}

Supongamos que queremos enviar un mensaje, por lo que habrá un emisor y un receptor que se comunican en una dirección. Este mensaje es enviado por un \emph{canal de comunicación}, cuyas características dependen de la naturaleza del mensaje a ser enviado. En general, hay que hacer una \emph{traducción} entre el mensaje original (o \emph{palabra fuente}) $x$ y el tipo de mensaje $c$ que el canal está capacitado para enviar (\emph{palabras código}). Este proceso se llama \emph{codificación}. Una vez codificado el mensaje, lo enviamos a través del canal, y nuestro intermediario (el receptor) recibe un mensaje codificado (\emph{palabra recibida}) posiblemente erróneo, ya que en todo proceso de comunicación hay ruido e interferencias. Una vez recibido, empieza el proceso llamado \emph{corrección de errores}, que consiste en recuperar el mensaje original corrigiendo los errores que se hubieran producido. El mensaje recibido $c'$ es traducido nuevamente a términos originales $x'$, es decir, es \emph{decodificado}. La siguiente figura representa un esquema de este proceso.

\begin{figure}[H]
	\center
	\includegraphics[scale=0.5]{figures/Diagrama_comunicacion.png}
	\caption{Esquema del modelo de comunicación}
\end{figure}

Las flechas indican que la comunicación es en un solo sentido.

En general, $x' \neq x$ y es deseable que este error sea detectado (lo cual permite pedir una retransmisión del mensaje) y en lo posible corregido.

La \emph{Teoría de Códigos Autocorrectores} se ocupa del segundo y cuarto pasos del esquema anterior, es decir, de la codificación y decodificación de mensajes, junto con el problema de detectar y corregir errores. A veces no es posible pedir retransmisión de mensajes y es por eso que los códigos autocorrectores son tan útiles y necesarios.

La calidad de un código con mensajes de longitud $k$ y palabras código de longitud $n$ vendrá dada por las siguientes características.

\begin{itemize}
    \item El cociente $\frac{k}{n}$, el \emph{ratio de información} del código, que mide el esfuerzo necesario para transmitir un mensaje codificado.
    \item La \emph{distancia mínima relativa} $d$ que es aproximadamente el doble de la proporción de errores que se pueden corregir en cada mensaje codificado.
    \item La \emph{complejidad} de los procedimientos de codificar y decodificar.
\end{itemize}

De esta forma, uno de los objetivos centrales de la teoría de códigos autocorrectores es construir códigos que sean de calidad. Esto es, códigos que permitan codificar muchos mensajes, que se puedan trasmitir rápida y eficientemente, que detecten y corrijan simultáneamente la mayor cantidad de errores posibles y que haya algoritmos de decodificación eficientes y efectivos. Por lo que habrá que encontrar un balance entre estas distintas metas, pues suelen ser contradictorias entre sí.

\section{Códigos lineales}

Sea $F_q$ el cuerpo finito con $q$ elementos, denotamos por $\mathbb{F}_q^n$ el espacio vectorial de las n-tuplas sobre el cuerpo finito $\mathbb{F}_q$. Generalmente los vectores $(a_1, ..., a_n)$ de $\mathbb{F}_q^n$ se denotarán por $a_1 \cdots a_n$.

\begin{definition}
    Un $(n, M)$ \emph{código} $\mathcal{C}$ sobre $\mathbb{F}_q$ es un subconjunto de 
    $\mathbb{F}_q^n$ de tamaño $M$. A los elementos de $\mathcal{C}$ los llamaremos \emph{palabras código}.
\end{definition}

\begin{exampleth}
    $ $
    \begin{itemize}
        \item Un código sobre $\mathbb{F}_2$ se llama \emph{código binario} y un ejemplo es $\mathcal{C} = \{00, 01, 10, 11\}$.
        \item Un código sobre $\mathbb{F}_3$ se llama \emph{código ternario} y un ejemplo es $\mathcal{C} = \{21, 02, 10, 20\}$.
    \end{itemize}
\end{exampleth}

Si $\mathcal{C}$ un subespacio k-dimensional de $\mathbb{F}_q^n$, entonces decimos que $\mathcal{C}$ es un $\left[ n, k \right]$ \emph{código lineal} sobre $\mathbb{F}_q$. De esta forma, los códigos lineales tendrán $q^k$ palabras código. Estos se pueden presentar con una matriz generadora o con una matriz de paridad.

\begin{definition}
    Una \emph{matriz generadora} para un $\left[ n,k \right]$ código $\mathcal{C}$ es una matriz $k \times n$ donde sus filas forman una base de $\mathcal{C}$.
\end{definition}

\begin{definition}
    Para cada conjunto de $k$ columnas independientes de una matriz generadora $G$, se dice que el conjunto de coordenadas correspondiente conforman un \emph{conjunto de información} de $\mathcal{C}$. Las $r = n-k$ restantes coordenadas se denominan \emph{conjunto de redundancia} y el número $r$ es la \emph{redundancia} de $\mathcal{C}$.
\end{definition}

En general, la matriz generadora no es única pues se puede obtener a partir de cualquier base. Sin embargo, si las $k$ primeras coordenadas conforman un conjunto de información, entonces el código tiene una única matriz generadora de la forma $( I_k | A)$, donde $I_k$ denota a la matriz identidad $k \times k$. Esta matriz se dice que está en \emph{forma estándar}.

Como un código lineal es un subespacio de un espacio vectorial, es el núcleo de alguna transformación lineal.

\begin{definition}
    Una \emph{matriz de paridad} $H$ de dimensión $(n-k) \times n$ de un $\left[ n,k \right]$ código $\mathcal{C}$ es una matriz que verifica que

    $$C = \left\lbrace \mathbf{x} \in \mathbb{F} _q^n : H\mathbf{x}^T = 0 \right\rbrace .$$
\end{definition}

Al igual que con la matriz generadora, la matriz de paridad no es única. Con el siguiente resultado podremos obtener una matriz de paridad cuando $\mathcal{C}$ tiene una matriz generadora en forma estándar.

\begin{theorem}
    \label{th:generadora-paridad}
    Si $G = \left( I_k | A \right)$ es una matriz generadora para el $\left[ n,k \right]$ código $\mathcal{C}$ en forma estándar, entonces $H = \left( -A^T | I_{n-k} \right)$ es una matriz de paridad de $\mathcal{C}$.
\end{theorem}

\begin{proof}
    Como $HG^T = -A^T + A^T = 0$, se tiene que $\mathcal{C}$ está contenido en el núcleo de la transformación lineal $x \mapsto Hx^T$. Esta transformación lineal tiene un núcleo de dimensión $k$, pues $H$ tiene rango $n-k$, que coincide con la dimensión de $\mathcal{C}$.
\end{proof}

\begin{exampleth}
    \label{ex:generadora-paridad}
    Sea la matriz $G = \left( I_4 | A \right)$, donde 
    
    \[
        G = \left( 
        \begin{array}{cccc|ccc}  
            1 & 0 & 0 & 0 & 0 & 1 & 1 \\
            0 & 1 & 0 & 0 & 1 & 0 & 1 \\
            0 & 0 & 1 & 0 & 1 & 1 & 0 \\
            0 & 0 & 0 & 1 & 1 & 1 & 1
        \end{array} 
        \right)
    \]

    es una matriz generadora en forma estándar para un $[7, 4]$ código binario que denotaremos por $\mathcal{H}_3$. Por el Teorema \ref{th:generadora-paridad}, una matriz de paridad de $\mathcal{H}_3$ es

    \[ 
        H = 
        \left( 
        \begin{array}{c|c}  
            -A^T & I_{7-4}
        \end{array} 
        \right)
        = 
        \left( 
        \begin{array}{c|c}  
            -A^T & I_{3}
        \end{array} 
        \right)
        =
        \left( 
        \begin{array}{cccc|ccc}  
            0 & 1 & 1 & 1 & 1 & 0 & 0 \\
            1 & 0 & 1 & 1 & 0 & 1 & 0 \\
            1 & 1 & 0 & 1 & 0 & 0 & 1 \\
        \end{array} 
        \right)
    \]

    Este código se denomina el $[7, 4]$ \emph{código de Hamming}.
\end{exampleth}

\section{Código dual}

Sabemos que $\mathcal{C}$ es un subespacio de un espacio vectorial, por lo que podemos calcular el subespacio ortogonal a dicho subespacio y así obtener lo que se denomina \emph{espacio dual u ortogonal} de $\mathcal{C}$, denotado por $\mathcal{C} ^{\perp}$. Se define este concepto con la operación del producto escalar como sigue.

\begin{definition}
    El \emph{espacio dual} de $\mathcal{C}$ viene dado por 
    
    $$\mathcal{C} ^{\perp} = \left\{ \mathbf{x} \in \mathbb{F}_q^n \; : \; \mathbf{x} \cdot \mathbf{c} = 0 \quad \forall \mathbf{c} \in \mathcal{C} \right\}$$
\end{definition}

Se observa que $\mathcal{C} ^{\perp}$ es un $[n,n-k]$ código.

El siguiente resultado nos muestra cómo obtener las matrices generadora y de paridad de $\mathcal{C} ^{\perp}$ a partir de las de $\mathcal{C}$.

\begin{proposition}
    Si tenemos una matriz generadora $G$ y una matriz de paridad $H$ de un código $\mathcal{C}$, entonces $H$ y $G$ son matrices generadoras y de paridad, respectivamente, de $\mathcal{C} ^{\perp}$.
\end{proposition}

Diremos que un código $\mathcal{C}$ es \emph{auto-ortogonal} si $\mathcal{C} \subseteq \mathcal{C} ^{\perp}$ y \emph{auto-dual} cuando $\mathcal{C} = \mathcal{C} ^{\perp}$.

\begin{exampleth}
    Una matriz generadora para el $[7, 4]$ código de Hamming $\mathcal{H}_3$ se presenta en el Ejemplo \ref{ex:generadora-paridad}. Sea $\mathcal{H'}_3$ el código de longitud 8 y dimensión 4 obtenido de $\mathcal{H}_3$ añadiendo una coordenada de verificación de paridad general a cada vector de G y por lo tanto a cada palabra código de $\mathcal{H}_3$. Entonces 

    \[
        G' = \left( 
        \begin{array}{cccc|cccc}  
            1 & 0 & 0 & 0 & 0 & 1 & 1 & 1 \\
            0 & 1 & 0 & 0 & 1 & 0 & 1 & 1\\
            0 & 0 & 1 & 0 & 1 & 1 & 0 & 1\\
            0 & 0 & 0 & 1 & 1 & 1 & 1 & 0
        \end{array} 
        \right)
    \]

    es una matriz generadora para $\mathcal{H'}_3$. Además, veamos que $\mathcal{H'}_3$ es un código auto-dual.

    Tenemos que $G' = (I_4 | A')$, donde

    \[
        A' = \left( 
        \begin{array}{cccc}  
            0 & 1 & 1 & 1 \\
            1 & 0 & 1 & 1\\
            1 & 1 & 0 & 1\\
            1 & 1 & 1 & 0
        \end{array} 
        \right) .
    \]

    Como $A' (A')^T = I_4$, entonces $\mathcal{H'}_3$ es auto-dual.
\end{exampleth}

\section{Pesos y distancias}

A la hora de corregir errores es importante establecer una medida que nos establezca cuánto de diferentes son las palabras enviadas y recibidas. En este apartado estudiaremos esta idea y cómo puede influir a la teoría de códigos.

\begin{definition}
    La \emph{distancia de Hamming} $\operatorname{d}(\mathbf{x},\mathbf{y})$ entre dos vectores $\mathbf{x}$, $\mathbf{y} \in \mathbb{F}_q^n$ se define como el número de coordenadas en las que $\mathbf{x}$ e $\mathbf{y}$ difieren.
\end{definition}

\begin{exampleth}
    Sean $\mathbf{x} = 012$, $\mathbf{y} = 210$ $\mathbf{x}, \mathbf{y} \in \mathbb{F}_3^4$. Entonces la distancia de Hamming entre los dos vectores es $d(\mathbf{x}, \mathbf{y}) = 2$.
\end{exampleth}

\begin{theorem}
    La función distancia $\operatorname{d}(\mathbf{x},\mathbf{y})$ satisface las siguientes propiedades.

    \begin{enumerate}
        \item No negatividad: $\operatorname{d}(\mathbf{x},\mathbf{y}) \geq 0 \quad \forall \mathbf{x},\mathbf{y} \in \mathbb{F}_q^n$.
        \item $\operatorname{d}(\mathbf{x},\mathbf{y}) = 0 \; \Leftrightarrow \; \mathbf{x} = \mathbf{y}$.
        \item Simetría: $\operatorname{d}(\mathbf{x},\mathbf{y}) = \operatorname{d}(\mathbf{y},\mathbf{x}) \quad \forall \mathbf{x},\mathbf{y} \in \mathbb{F}_q^n$.
        \item Desigualdad triangular: $\operatorname{d}(\mathbf{x},\mathbf{z}) \leq \operatorname{d}(\mathbf{x},\mathbf{y}) + \operatorname{d}(\mathbf{y},\mathbf{z}) \quad \forall \mathbf{x},\mathbf{y},\mathbf{z} \in \mathbb{F}_q^n$
    \end{enumerate}
\end{theorem}

\begin{proof}
    Las tres primeras afirmaciones se obtienen directamente a partir de la definición.
    La cuarta propiedad se obtiene a partir de la no negatividad. Esto es, sean $x,y,z \in \mathbb{F}_q^n$ distingamos dos casos. Si $x \neq z$
    tenemos que $y \neq x$ o $y \neq z$, entonces por la no negatividad se cumple la afirmación.
    En el caso en el que $x = z$, tendríamos que $d(x,z) = 0$ y también se da la 
    afirmación.
\end{proof}

Diremos que la \emph{distancia mínima} de un código $\mathcal{C}$ es la distancia más pequeña entre las distintas palabras código. Esta medida es fundamental a la hora de determinar la capacidad de corregir errores de $\mathcal{C}$.

\begin{exampleth}
    Sea $\mathcal{C} = \{010101, 212121, 111000\}$ un código ternario. Entonces

    $$d(010101, 212121) = 3, \qquad d(010101, 111000) = 4, \qquad d(212121, 111000) = 5.$$

    Por lo que la distancia mínima del código $\mathcal{C}$ es $d(\mathcal{C}) = 3$.
\end{exampleth}

\begin{theorem}[Decodificación de máxima verosimilitud]
    \label{th:decodificacion_maxima_verosimilitud}
    Es posible corregir hasta

    $$t := \left\lfloor \frac{d(\mathcal{C}) - 1}{2} \right\rfloor$$

    errores, donde $d(\mathcal{C})$ denota la distancia mínima del código $\mathcal{C}$.
\end{theorem}

\begin{proof}
    Usando la decodificación de máxima verosimilitud, un vector $y \in \mathbb{F}^n$ es decodificado en una palabra código $c  \in \mathcal{C}$, que es cercana a $y$ con respecto a la distancia de Hamming. Formalmente, $y$ es decodificado en una palabra código $c \in \mathcal{C}$ tal que $d(c,y) \leq d(c',y)$, $\forall c' \in \mathcal{C}$. Si hay varios $c \in \mathcal{C}$ con esta propiedad, se elige uno arbitrariamente.

    Si la palabra código $c \in \mathcal{C}$ fue enviada y no han ocurrido más de $t$ errores durante la transmisión, el vector recibido es 

    $$y = c + e \in \mathbb{F}^n,$$

    donde $e$ denota al vector error. Esto satisface 

    $$d(c,y) = d(e,0) \leq t,$$

    y por lo tanto $c$ es el único elemento de $\mathcal{C}$ que se encuentra en una bola de radio $t$ alrededor de $y$. Un decodificador de máxima verosimilitud produce este elemento $c$, y así se obtiene el código correcto.
\end{proof}

\begin{definition}
    El \emph{peso Hamming} $\operatorname{wt}(\mathbf{x})$ de un vector $\mathbf{x} \in \mathbb{F}_q^n$ se define como el número de coordenadas no nulas en $\mathbf{x}$.
\end{definition}

\begin{exampleth}
    Sea $\mathbf{x} = 2001021 \in \mathbb{F}_3^7$ un vector, entonces su peso Hamming es $wt(\mathbf{x}) = 4$.
\end{exampleth}

El siguiente resultado nos muestra la relación entre la distancia y el peso.

\begin{theorem}
    Si $\mathbf{x}, \mathbf{y} \in \mathbb{F}_q^n$, entonces $\operatorname{d}(\mathbf{x},\mathbf{y}) = \operatorname{wt}(\mathbf{x}-\mathbf{y})$. Si $\mathcal{C}$ es un código lineal, entonces la distancia mínima $d$ coincide con el peso mínimo de las palabras código no nulas de $\mathcal{C}$.
\end{theorem}

\begin{proof}
    Sean $x,y \in \mathbb{F}_q^n$, por la definición de distancia de Hamming tenemos que $d(x,y) = wt(x-y)$. Se supone ahora que $C$ es un código lineal, luego para todo $x,y \in \mathcal{C}$, $x-y \in \mathcal{C}$, luego para cualquier par de elementos $x,y \in \mathcal{C}$, existe $z \in \mathcal{C}$ tal que $d(x,y) = wt(z) \geq wt(\mathcal{C})$, donde $wt(\mathcal{C})$ es el peso mínimo de $\mathcal{C}$. Por tanto, $d \geq wt(\mathcal{C})$. Por otro lado, para todo $x \in \mathcal{C}$, se tiene que $wt(x) = d(x,0)$. Como $\mathcal{C}$ es lineal, $0 \in \mathcal{C}$, luego $d(x,0) \geq d$. Entonces, $wt(\mathcal{C}) \geq d$. Se concluye que $wt(\mathcal{C}) = d$, como se quería.
\end{proof}

Como consecuencia de este teorema, para códigos lineales, la distancia mínima también se denomina \emph{peso mínimo} de un código. Si se conoce el peso mínimo $d$ de un $[n,k]$ código, se dice entonces que es un $[n,k,d]_q$ código.

En el artículo \cite{Vardy_1997} se ha demostrado que el problema de calcular la distancia mínima de un código lineal binario es NP-difícil, y el problema de decisión correspondiente es NP-completo.

\begin{theorem}
    Dada una matriz binaria $H$ de dimensión $m \times n$ y un número entero $w > 0$, saber si existe un vector no nulo $x \in \mathbb{F}_2^n$ de peso menor que $w$ tal que $Hw^T = 0$ es un problema NP-completo. 
\end{theorem}

Veamos un esquema de la demostración de este resultado. Para ello, haremos uso de una transformación polinomial del problema de Decodificación de Máxima Verosimilitud al problema de Distancia Mínima.

El problema de Decodicificación de Máxima Verosimilitud es NP-completo y consiste en dados una matriz binaria $H$ de dimensión $m \times n$, un vector $s \in \mathbb{F}_2^m$ y un número entero $w > 0$. ¿Existe un vector $x \in \mathbb{F}_2^n$ de peso menor o igual que $w$ tal que $Hx^t = s$?

El problema de Decodificación de Máxima Verosimilitud sigue siendo NP-completo bajo ciertas restricciones, luego reformularemos este problema como la versión de campo finito de Suma de Subconjuntos, un problema NP-completo conocido. Además, calcular la distancia mínima para la clase de códigos lineales sobre un cuerpo de característica 2 es NP-difícil, y el problema de decisión correspondiente Distancia Mínima sobre $GF(2^m)$, abreviado $MD_{2^m}$, es NP-completo. Luego esta prueba se basa en una transformación polinomial de Decodificación de Máxima Verosimilitud a $MD_{2^m}$. Sin embargo, esto no prueba que Distancia Mínima sea NP-completo, ya que el posible conjunto de entradas a Distancia Mínima es un pequeño subconjunto del conjunto de posibles entradas a $MD_{2^m}$. Para ello, se construye una aplicación del código $\mathbb{C\#}$ sobre $GF(2^m)$ a un código binario $\mathbb{C}$, de tal forma que la distancia mínima de $\mathbb{C\#}$ puede determinarse a partir de la distancia mínima de $\mathbb{C}$. Dado que la longitud de $\mathbb{C}$ está acotada por la longitud de un polinomio de $\mathbb{C\#}$, y el mapeo en sí se puede lograr en tiempo polinomial, esto completa la prueba de la NP-completitud de Distancia Mínima.


\begin{definition}
    Sea $A_i$, también denotada por $A_i(\mathcal{C})$, el número de palabras código con peso $i$ en $\mathcal{C}$. Se dice que la lista $A_i$ para $0 \leq i \leq n$ es la distribución del peso o espectro del peso de $\mathcal{C}$.
\end{definition}

\section{Clasificación por isometría}

Como hemos visto, las propiedades de un código dependen principalmente de las distancias de Hamming entre sus palabras y entre palabras codificadas y no codificadas. Además, puede ser que un código pueda relacionarse con otro por medio de una aplicación que conserve las distancias de Hamming. De esta forma, podemos definir una relación de equivalencia entre dos códigos que preservan la distancia de Hamming.

Sean $C$ y $C'$ dos $[n,k]_q$ códigos, se dice que son de la misma cualidad si existe una aplicación

$$\iota : H(n,q) \rightarrow H(n,q)$$

con $\iota(C) = C'$ que preserva la distancia de Hamming, es decir, 

$$d(w,w') = d(\iota (w), \iota(w')), \qquad \forall w,w' \in H(n,q).$$

Los mapeos con la propiedad anterior se llaman \emph{isometrías}.

\begin{definition}
    Dos códigos lineales $C, C' \subseteq H(n,q)$ se llaman \emph{isométricos} si existe una isometría de $H(n,q)$ que aplica $C$ sobre $C'$.
\end{definition}

Las permutaciones de las coordenadas son isometrías, que se denominan \emph{isometrías permutacionales}.

\begin{definition}
    Sea $S_n$ el grupo isométrico en el conjunto $X = n = \left\{ 0,..., n-1 \right\}$. Dos códigos lineales $C, C' \subseteq H(n,q)$ son isométricos permutacionalmente si existe una isometría permutacional de $H(n,q)$ que aplica $C$ sobre $C'$. Esto es, hay una permutación $\pi$ en el grupo simétrico $S_n$ tal que 

    $$ C' = \pi (C) = \left\{ \pi(c) : c \in C \right\}, \quad \text{ and } \quad d(c, \tilde{c}) = d(\pi(c), \pi(\tilde{c})), \qquad \forall c,\tilde{c} \in C,$$ 

    donde
    
    $$\pi(c) = \pi(c_0,...,c_{n-1}) := \left( c_{\pi ^{-1} (0)}, ..., c_{\pi ^{-1} (n-1)} \right)$$
\end{definition}

\section{Algoritmo para el cálculo de la distancia}

Como hemos visto, la distancia mínima es importante en un código lineal. Sin embargo, calcular este parámetro para un código dado puede resultar realmente duro. A continuación presentaremos el algoritmo de BZ para el cálculo de la distancia, que tiene eficiencia exponencial.

\begin{Ualgorithm}[htbp]
    \DontPrintSemicolon
    \KwIn{matriz generadora $G_1 = (I_k | A_1)$ de $\mathcal{C}$}
    \KwOut{distancia mínima $\bar{d}_i$}
    $m \longleftarrow 2$\;
    $k_1 \longleftarrow k$\;
    \While{rank $(A_m) \neq 0$}{
        Aplicar la eliminación Gaussiana y posibles permutaciones de las columnas de la matriz $A_{m-1}$ desde $G_{m-1} = \left( \begin{array}{c|c} A_{m-1}' & \begin{array}{c|c} I_{k_{m-1}} & A_{m-1} \\ \hline  0 & 0 \end{array} \end{array} \right)$ para obtener la matriz generadora $G_{m} = \left( \begin{array}{c|c}  A_{m}' & \begin{array}{c|c}   I_{k_{m}} & A_{m} \\  \hline 0 & 0  \end{array}  \end{array} \right)$\;
    }
    $C_0 \longleftarrow \left\lbrace 0 \right\rbrace$\;
    $i \longleftarrow 0$\;
    \While{$\bar{d}_i > d_i'$}{
        $i \longleftarrow i + 1$\;
        $C_i \longleftarrow C_{i-1} \cup \bigcup_{j=1}^m \left\lbrace v \cdot G_j : v \in \mathbb{F}(q)^k, \; wt(v) = i \right\rbrace$\;
        $\bar{d}_i \longleftarrow min \lbrace wt(c) : c \in C_i, \; c \neq 0 \rbrace$\;
        $d_i' \longleftarrow \sum_{j=1, k-k_j \leq i}^m (i+1)-(k-k_j)$\;
    }
    \caption{Algoritmo de Brouwer-Zimmermann: cálculo de la distancia mínima de un $[n,k]$ código lineal $\mathcal{C}$.}
\end{Ualgorithm}


% 2. Códigos de Goppa.
% Esto lo puedes encontrar en el capítulo 13 de Huffman y Pless. 
% La decodificación  de estos códigos, es similar a la de los códigos BCH (capítulo 5) utilizando el algoritmo de Sugiyama. 
% El artículo donde se definen y se decodifican es el de Goppa.pdf. 
% Además, ahi se explica de forma elemental, sin utilizar geometría algebraica.

\chapter{Códigos de Goppa}


% TODO: introduccion


\section{Espacio afín, espacio proyectivo y homogeneización}

Los códigos de geometría algebraica se definen con respecto a curvas tanto en el espacio afín como en el espacio proyectivo.

Sea $\mathbb{F}$ un cuerpo, posiblemente infinito. Se define el $\emph{espacio afín n-dimensional sobre } \mathbb{F}$, denotado por $\mathbb{A}^n (\mathbb{F})$, como el espacio vectorial n-dimensional ordinario $\mathbb{F}^n$. Los puntos en $\mathbb{A}^n (\mathbb{F})$ son $(x_1,...,x_n)$ donde $x_i \in \mathbb{F}$.

% deberia definir el espacio proyecto y tal?? espero q no...



\section{Algunos códigos clásicos}

% TODO: codigos BCH

\subsection{Códigos Reed-Solomon generalizados}

Para $k \geq 0$, $\mathcal{P}_k$ denota el conjunto de polinomios de grado menor que $k$, incluyendo el polinomio nulo, en $\mathbb{F}_q[x]$. Sea $n$ un número entero tal que $1 \leq n \leq q$, $\gamma = (\gamma _0,..., \gamma _{n-1})$ una n-tupla de elementos distintos de $\mathbb{F}_q$, y $\textbf{v} = (v_0,...,v_{n-1})$ una n-tupla de elementos no nulos de $\mathbb{F}_q$. Sea $k$ un número entero tal que $1 \leq k \leq n$. Entonces los códigos

$$GRS_k (\gamma, \textbf{v}) = \left\{ \left( v_0 f(\gamma_0), ..., v_{n-1}f(\gamma_{n-1}) \right) : f \in \mathcal{P}_k \right\}$$

son los códigos Reed-Solomon generalizados (códigos GRS).

\subsection{Códigos clásicos de Goppa}

Los códigos clásicos de Goppa se introdujeron por V. D. Goppa en 1970. Estos códigos son generalizaciones de códigos BCH y subcódigos de subcuerpos de ciertos códigos GRS.

Para motivar la definición de los códigos Goppa, se introduce una construcción de los códigos BCH de longitud $n$ sobre $\mathbb{F}_q$. Sea $t = ord_q(n)$ y sea $\beta$ la raíz enésima primitiva de la unidad en $\mathbb{F}_{q^t}$. Se elige $\delta > 1$ y sea $\mathcal{C}$ el código BCH de longitud $n$ y distancia $\delta$. Entonces $c(x) = c_0 + c_1x + \cdots + c_{n-1}x^{n-1} \in \mathbb{F}_q [x] / (x^n - 1)$ está en $\mathcal{C}$ si y solo si $c(\beta^j) = 0$ para $1 \leq j \leq \delta - 1$. Tenemos que 

$$(x^n - 1) \sum_{i=0}^{n-1} \frac{c_i}{x - \beta ^{-i}} = \sum_{i=0}^{n-1} c_i \sum_{l=0}^{n-1} x^l (\beta ^{-i})^{n-1-l} = \sum_{l=0}^{n-1} x^l \sum_{i=0}^{n-1} c_i (\beta^{l+1})^i.$$

Como $c(\beta^{l+1}) = 0$ para $0 \leq l \leq \delta - 2$, el lado derecho de la ecuación es un polinomio cuyo término de menor grado tiene grado al menos $\delta - 1$. Por lo tanto, el lado derecho se puede escribir como $x^{\delta - 1} p(x)$, donde $p(x)$ es un polinomio en $\mathbb{F}_{q^t}[x]$. Así, se puede decir que $c(x) \in \mathbb{F}_q[x] / (x^n - 1)$ está en $\mathcal{C}$ si y solo si 

$$\sum_{i=0}^{n-1} \frac{c_i}{x - \beta ^{-i}} = \frac{x^{\delta - 1} p(x)}{x^n - 1}$$

o equivalentemente

$$\sum_{i=0}^{n-1} \frac{c_i}{x - \beta ^{-i}} \equiv 0 (mod x^{\delta - 1})$$

La última equivalencia es la base para la definición de los códigos clásicos de Goppa.

% continuar con la pagina 540, se fija un cuerpo de extension...
% 4. Criptografía post-cuántica basada en códigos. 
% Aquí depende de si vas a matricularte en la asignatura de Criptografía (la imparte Javier Lobillo). 
% Allí te explicarán los sistemas asimétricos. Si no te matriculas, lo puedes encontrar en el libro Introduction to Cryptography (capítulos 1,2,3).
% No te hace falta todo lo que hay en esos capítulos, pero, bueno, es interesante para una matemática/informática el conocer este tipo de cosas.
% Para la parte post-cuántica, tenemos el archivo 9783540887010-c1.pdf y el libro Post-QuantumCryptography 
% (capítulos Introduction to post-quantum cryptography y Code-based cryptography, también Quantum computing, si te apetece). 
% Para el criptosistema de McEliece (que es que implementaremos) está el archivo sander-report-s15.pdf, 
% donde se explica este y el criptosistema de Niederreiter, que es equivalente en seguridad.

% buscar la referencia original del 78 de McEliece que ahi viene mejor explicado

\chapter{Criptografía post-cuántica basada en códigos}

El principal objetivo de la criptografía es proporcionar confidencialidad mediante métodos de cifrado. Cuando queremos enviar un mensaje a un destinatario, el canal de comunicación puede ser inseguro y otras personas podrían leerlo o incluso modificarlo de tal forma que el destinatario no se diera cuenta. Para prevenir estos ataques nos será de utilidad la criptografía.

En este capítulo introduciremos las nociones básicas de la criptografía, junto a sus objetivos, y estudiaremos los dos principales tipos de criptosistemas (simétricos y asimétricos) \cite{Introduction_to_cryptography}. Definiremos el primer y más utilizado algoritmo de los sistemas criptográficos asimétricos, el RSA. Finalmente, discutiremos la seguridad de los criptosistemas ante la existencia de grandes ordenadores cuánticos y mostraremos algunas alternativas que sean capaces de resistir sus ataques \cite{Post-Quantum_Cryptography_2009}. En concreto, estudiaremos el criptosistema de McEliece y el de Niederreiter \cite{Sander}.

\section{Introducción}

En general, los métodos de cifrado consisten en encriptar el mensaje, llamado \emph{texto plano}, antes de ser transmitido, de esta forma obtenemos un \emph{texto cifrado} o \emph{criptograma}. Este texto cifrado se transmite al destinatario, quien lo \emph{desencripta} mediante una \emph{clave de descifrado}, la cual solo conocen el receptor y el emisor y previamente la intercambiaron.

Formalmente, dado un conjunto de los mensajes $\mathcal{M}$, un conjunto de los criptogramas $\mathcal{C}$ y el espacio de claves $\mathcal{K}_p \times \mathcal{K}_s$, un \emph{criptosistema} viene definido por dos aplicaciones
\[
    E : \mathcal{K}_p \times \mathcal{M} \rightarrow \mathcal{C},
\]
\[
    D : \mathcal{K}_s \times \mathcal{C} \rightarrow \mathcal{M},
\]
tales que para cualquier clave $k_p \in \mathcal{K}_p$, existe una clave $k_s \in \mathcal{K}_s$ de manera que dado cualquier mensaje $m \in \mathcal{M}$,
\begin{equation}
    \label{def:criptosistema}
    D(k_s, E(k_p, m)) = m.
\end{equation}

Para simplificar la notación de las funciones de cifrado y descifrado usaremos, fijadas las claves $k_p \in \mathcal{K}_p$ y su correspondiente $k_s \in \mathcal{K}_s$:
\begin{align*} 
    E_{k_p} &: \mathcal{M} \rightarrow \mathcal{C}, \quad [ E_{k_p} (m) = E(k_p, m) ]\\ 
    D_{k_s} &: \mathcal{C} \rightarrow \mathcal{M}, \quad [ D_{k_s} (c) = D(k_s, c) ]
\end{align*}

La propiedad \ref{def:criptosistema} se transforma en
\[
    D_{k_s} \left( E_{k_p}(m) \right) = m.
\]
Para encriptar y desencriptar existen \emph{algoritmos de cifrado} y \emph{de descifrado}, respectivamente, y cada uno usará una clave secreta. Si esta clave es la misma en ambos algoritmos, diremos que los métodos de encriptación son \emph{simétricos}. Algunos ejemplos importantes de estos métodos son DES (\emph{Data Encryption Standard}) y AES (\emph{Advanced Encryption Standard}).

En 1976, Diffie y Hellman \cite{Diffie_Hellman_1976} introdujeron un concepto revolucionario, la \emph{criptografía de Clave Pública}, también llamada \emph{criptografía asimétrica}, que permitió dar una solución al antiguo problema del intercambio de claves e indicar el camino a la firma digital. Los métodos de cifrado de \emph{clave pública} son \emph{asimétricos}. Cada receptor tiene una clave personal $k = (k_p, k_s)$, que consiste en dos partes: $k_p$ es la clave de cifrado y es pública, y $k_s$ es la clave de descifrado, que es privada. De esta forma, si queremos enviar un mensaje, lo encriptaremos mediante la clave pública $k_p$ del receptor. Así, el receptor poddrá descifrar el texto cifrado usando su clave privada $k_s$, que solo la conoce él. Al ser la clave pública, cualquiera puede encriptar un mensaje usándola, sin embargo descifrarlo sin saber la clave privada será casi imposible.

\section{Objetivos de la criptografía}

Además de proporcionar confidencialidad, la criptografía proporciona soluciones para otros problemas:

\begin{enumerate}
    \item \emph{Confidencialidad}. La información solo puede ser accesible por las entidades autorizadas.
    \item \emph{Integridad de datos}. El receptor de un mensaje debería ser capaz de determinar que el mensaje no ha sido modificado durante la transmisión.
    \item \emph{Autenticidad}. El receptor de un mensaje debería ser capaz de verificar su origen.
    \item \emph{No repudio}. El emisor de un mensaje debería ser incapaz de negar posteriormente que envió el mensaje.
\end{enumerate}

Para garantizar la integridad de los datos, hay métodos simétricos y de clave pública. El mensaje $m$ es aumentado por un \emph{código de autenticación de mensaje} (MAC). Este código es generado por un algoritmo que depende de la clave secreta. Así, el mensaje aumentado $(m, MAC(k,m))$ está protegido contra modificaciones. El receptor ahora puede comprobar la integridad del mensaje $(m, \bar{m})$ verificando que $MAC(k, m) = \bar{m}$.

Las firmas digitales requieren métodos de clave pública y proporcionan autenticación y no repudiablidad. Cualquier persona puede verificar si una firma digital es válida con la clave pública del firmante. Esto es, si firmamos con nuestra clave privada $k$, obtenemos la firma $Sign(k_s, m)$. El receptor recibe la firma $s$ del mensaje $m$ y comprueba con el algoritmo de verificación \emph{Verify} que se cumple que \emph{Verify}$(k_p, s, m) = ok$, siendo $k_p$ la clave pública del emisor.

\section{Criptografía simétrica}

Un criptosistema \emph{simétrico}, como hemos visto, viene determinado por dos aplicaciones
\[
    E : \mathcal{K} \times \mathcal{M} \rightarrow \mathcal{C},
\]
\[
    D : \mathcal{K} \times \mathcal{C} \rightarrow \mathcal{M},
\]
tales que para cualquier clave $k \in \mathcal{K}$ y cualquier mensaje $m \in \mathcal{M}$,
\begin{equation}
    \label{def:criptosistema_simetrico}
    D(k, E(k, m)) = m.
\end{equation}

Fijada $k \in \mathcal{K}$, usaremos la notación 
\[
    E_k : \mathcal{M} \rightarrow \mathcal{C},
\]  
\[
    D_k : \mathcal{C} \rightarrow \mathcal{M},
\]
para las funciones de cifrado y descifrado. La propiedad \ref{def:criptosistema_simetrico} se transforma en
\[
    D_k \left( E_k (m) \right) = m.
\]
Observamos que la clave que se usa para cifrar es la misma que se usa para descifrar.

Algunos ejemplos importantes de sistemas simétricos son DES (\emph{Data Encryption Standard}) y AES (\emph{Advanced Encryption Standard}) \cite[Sección 2.1]{Introduction_to_cryptography}.

\section{Criptografía asimétrica}

A diferencia de la criptografía simétrica, en la criptografía asimétrica los participantes en la comunicación no comparten una clave secreta. Cada uno tiene un par de claves: la \emph{clave secreta} $k_s$ conocida solo por él y una \emph{clave pública} conocida por todos.

Supongamos que Bob tiene un par de claves $(k_p, k_s)$ y Alice quiere encriptar un mensaje $m$ para Bob. Alice, como cualquier otra persona, conoce la clave pública $k_p$ de Bob. Alice usa una función de encriptación $E$ con la clave pública $k_p$ de Bob para obtener el texto cifrado $c = E_{k_p}(m)$. Esto solo puede ser seguro si es prácticamente inviable calcular $m$ de $c = E_{k_p}(m)$. Sin embargo, Bob sí es capaz de calcular el mensaje $m$, ya que puede usar su clave secreta. La función de encriptación $E_{k_p}$ debe tener la propiedad de que su pre-imagen $m$ del texto cifrado $c = E_{k_p}(m)$ sea fácil de calcular usando la clave secreta $k_s$ de Bob, quien es el único que puede descifrar el mensaje encriptado.

En la criptografía de clave pública, necesitamos unas funciones $\left( E_{k_p} \right)_{k_p \in K_P}$ tales que cada función $E_{k_p}$ se pueda cacular con un algoritmo eficiente. Sin embargo, su pre-imagen debería ser prácticamente inviable de calcular. Estas familias $\left( E_{k_p} \right)_{k_p \in K_P}$ se denominan \emph{funciones de una sola dirección}. En cada función $E_{k_p}$ de la familia, tiene que haber una información secreta $k_s$ para que el algoritmo sea eficiente y calcule la inversa de $E_{k_p}$. Las funciones con esa propiedad se denominan \emph{funciones con trampilla}.

En 1976, Diffie y Hellman \cite{Diffie_Hellman_1976} presentaron la idea de la criptografía de clave pública, es decir, introdujeron métodos de clave pública para el acuerdo de clave y, además, describieron cómo las firmas digitales funcionarían. El primer criptosistema de clave pública que podía servir como un mecanismo de acuerdo de clave y como una firma digital fue el criptosistema RSA, que actualmente es el más conocido y usado. Este criptosistema lleva el nombre de sus inventores: Rivest, Shamir y Adleman. El criptosistema RSA se basa en la dificultad de factorizar grandes números, lo que le permite construir funciones de una sola dirección con una trampilla. Otras funciones de una sola dirección se basan en la dificultad de extraer logaritmos discretos. Estos dos problemas de la teoría de números son los cimientos de los criptosistemas de clave pública más usados actualmente.

Cada participante en un criptosistema de clave pública tiene una clave $k = (k_p, k_s)$, que consiste enu na clave pública ($k_p$) y una clave privada o secreta ($k_s$). Para garantizar la seguridad del criptosistema, debería ser inviable obtener la clave privada $k_s$ a partir de la clave pública $k_p$. Un algoritmo eficiente debería ser el encargado de elegir aleatoriamente ambas claves en un gran espacio de parámetros. Así, cualquiera puede usar $k_p$ para encriptar mensajes, pero solo quien posea $k_s$ podrá desencriptarlos.

En cuanto a las firmas digitales, supongamos que tenemos una familia de funciones con trampilla $\left( E_{k_p} \right)_{k_p \in K_P}$ donde cada función $E_{k_p}$ es biyectiva. Sea $k_p$ la clave pública de Alice, quien es la única de calcular la inversa $E_{k_p}^{-1}$ de $E_{k_p}$ pues para ello se necesita la clave privada $k_s$ de Alice. De esta forma, si Alice quiere firmar un mensaje $m$, tiene que calcular $E_{k_p}^{-1}(m)$, que será el valor de la firma $s$ de $m$. Todo el mundo puede verificar la firma de Alice $s$ pues cualquiera puede usar su clave pública $k_p$ y calcular $E_{k_p}(s)$. Si $E_{k_p} (s) = m$, entonces podemos asegurarnos de que Alice realmente firmó $m$ porque es la única que es capaz de calcular $E_{k_p}^{-1}(m)$.

Una importante aplicación de los criptosistemas de clave pública es que permiten intercambiar claves en sistemas de clave secreta. Si Alice conoce la clave pública de Bob, ella puede generar una clave de sesión, cifrarla con la clave pública de Bob y enviársela.  

Algunos sistemas conocidos de clave pública son:

\begin{itemize}
    \item \emph{RSA}: está basado en el problema de factorización de enteros.
    \item \emph{ElGamal}: está basado en el problema del logaritmo discreto.
    \item \emph{McEliece}: está basado en la teoría de los códigos Goppa.
    \item \emph{Curvas Elípticas}: son una generalización del sistema ElGamal y se basan en el problema de calcular logaritmos discretos en curvas elípticas.
\end{itemize}

\subsection{RSA}

El criptosistema RSA consiste en multiplicar dos números primos muy grandes y publicar su producto $n$. Una parte de la clave pública la conformará $n$, mientras que los factores de $n$ se mantienen en secreto y se usarán como clave privada. La idea básica es que los factores de $n$ no puedan recuperarse de $n$. Por lo que la seguridad de RSA radica en la dificultad del problema de factorización de enteros.

\subsubsection{Generación de claves}

Cada usuario del criptosistema RSA posee una clave pública y otra privada. Para generar este par de claves, se siguen los siguientes tres pasos:

\begin{enumerate}
    \item Se eligen aleatoriamente dos grandes números primos distintos $p$ y $q$ y se calcula su producto $n = p \cdot q$. También calculamos $\phi(n) = n + 1 - (p + q)$.
    \item Se elige un entero $e$ tal que $1 < e < \phi(n)$ y sea primo con $\phi(n)$.
    \item Se calcula $d$ que verifique $ed \equiv 1 \pmod{\phi(n)}$, es decir, $d \equiv e^{-1} \pmod{\phi(n)}$. 
    \item La clave privada es $(p, q, d)$.
    \item La clave pública es $(n, e)$.
\end{enumerate}

Los números $n$, $e$ y $d$ se denominan \emph{módulo}, \emph{exponente de cifrado} y \emph{exponente de descifrado}, respectivamente. El exponente de descifrado $d$ se puede obtener con el algoritmo extendido de Euclides. Con este exponente es posible descifrar un texto cifrado y generar una firma digital.

\subsubsection{Cifrado y descifrado}

Supongamos que queremos cifrar un mensaje. Para ello, usaremos la clave pública $(n, e)$. Para cifrar un texto plano $m$ podemos usar la función de cifrado:
\[
    RSA_{n, e}(m) = m^e \pmod{n},
\]
es decir, el texto cifrado $c$ es $m^e$ módulo $n$.

Para descifrar el criptograma $c$, se usa la función de descifrado:
\[
    RSA_{n, d}(c) = c^d \pmod{n}.
\]
De esta forma, se puede recuperar el texto plano, es decir, $m = c^d$ módulo $n$, pues las funciones de cifrado y descifrado $RSA_{n, e}$ y $RSA_{n, d}$ son inversas entre sí.

Con este procedimiento de cifrado, podemos cifrar secuencias de bit hasta $k := \lfloor \log_2{n} \rfloor$ bits. Si los mensajes son más largos, podemos dividirlos en bloques de longitud $k$ y cifrar cada uno por separado.

\begin{exampleth}
    Consideremos dos números primos $p = 7$ y $q = 11$, tenemos que $n = p \cdot q = 77$ y $\phi(n) = 60$. Busquemos ahora un número $e$ mayor que $1$ y menor que $60$ y que sea primo con $\phi(n) = 60$, elegimos $e = 13$. Por lo que el exponente de descifrado es
    \[
        d \equiv e^{-1} \equiv 37 \mod{\phi(n)}.
    \]
    La clave pública es $(e, n) = (13, 77)$ y la clave privada es $(d, n) = (37, 77)$. Vamos a cifrar el mensaje $m = 42$ con la función de cifrado:
    \[
        RSA_{n, e}(m) = m^e \equiv 42^{13} \equiv 14 \pmod{n}.
    \]
    Luego el mensaje cifrado $m$ es $14$. Ahora, si queremos descifrarlo, calcularemos lo siguiente:
    \[
        RSA_{n, d}(c) = c^d \equiv 14^{37} \equiv 42 \pmod{n}.
    \]
    Observamos que efectivamente coincide con el mensaje original $m$.
\end{exampleth}

\subsubsection{Firma digital}

El criptosistema RSA también se puede usar para realizar firmas digitales. Sea $(n, e)$ la clave pública y $d$ el exponente de descifrado, si queremos firmar un mensaje $m$, le aplicamos el algoritmo de descifrado y obtenemos la \emph{firma} de $m$, $\sigma = m^d$. Decimos que $(m, \sigma)$ es un \emph{mensaje firmado}. Para verificar ese mensaje, basta con calcular $\sigma ^e$, donde $e$ es la clave pública del firmante, y comprobar que coincide con $m$.

\section{Criptografía post-cuántica}

En \cite{Post-Quantum_Cryptography_2009} se comenta la creencia de que algunos sistemas criptográficos, tales como RSA, resisten los ataques de grandes ordenadores clásicos, pero no de los grandes ordenadores cuánticos. Sin embargo, se cree que algunas alternativas, como el sistema de McEliece con una clave de cuatro millones de bits, son capaces de resistir los ataques de grandes ordenadores clásicos y cuánticos.

Es por esto que surgen diversas preocupaciones, ante la amenaza de los ordenadores cuánticos se duda sobre si seguir usando RSA o simplemente cambiar a otros sistemas criptográficos que sean resistentes a dichos ordenadores. Sin embargo, esto no es tan sencillo pues necesitamos tiempo para mejorar la eficiencia, fomentar la confanza y mejorar de la usabilidad de la criptografía post-cuántica. En breve, todavía no estamos preparados para que el mundo cambie a la criptografía post-cuántica.

En esta sección estudiaremos los sistemas criptográficos basados en códigos, criptosistemas que usan un código de corrección de errores $\mathcal{C}$. Para ello, agregan un error a una palabra de $\mathcal{C}$ o al calcular un síndrome relativo a la matriz de paridad de $\mathcal{C}$.

El primero de esos sistemas es un sistema de cifrado de clave pública y fue propuesto por Robert J. McEliece en 1978 \cite{McEliece_1978}. Este sistema usa como clave privada un código de Goppa binario aleatorio y, como clave pública, una matriz generadora aleatoria de una versión permutada aleatoriamente de ese código. El texto cifrado es una palabra código a la que se le han añadido algunos errores, y solo el que posee la clave privada puede eliminar esos errores. Actualmente, no se conoce ningún ataque que presente una amenaza grave a este sistema, ni los ordenadores cuánticos. Cabe destacar la seguridad y la rapidez del criptosistema de McEliece, ya que tanto los procedimientos de cifrado y descifrado son de baja complejidad.

\subsection{Criptosistema de McEliece}

En 1978, R. J. McEliece propuso un criptosistema de clave pública \cite{McEliece_1978}. El criptosistema de McEliece se basa en códigos lineales de corrección de errores y hasta ahora es el criptosistema más exitoso basado en nociones de teoría de codificación.

La construcción original en \cite{McEliece_1978} usa códigos de Goppa binarios para cifrar y descifrar mensajes. Sim embargo, han surgido otras variantes de este criptosistema que usan otros códigos lineales, pero la mayoría han resultado ser inseguros. A día de hoy, la construcción original de 1978 se considera segura con la correcta elección de parámetros.

Por esta razón, el criptosistemade McEliece se pone a la altura del RSA. Existen algunas diferencias entre ambos criptosistemas, el de McEliece es capaz de cifrar y descifrar mensajes más rápido. Sin embargo, los tamaños de las claves son mayores que los del RSA, que es la razón por la que el criptosistema de McEliece apenas se usa. La principal razón por la que está creciendo el interés en el criptosistema de McEliece es porque es uno de los mejores candidatos para criptosistemas de clave pública seguros post-cuánticos. El esquema Niederreiter \cite{Niederreiter_1986}, una variante del criptosistema de McEliece, también permite construir un esquema de firma digital segura.

En esta sección presentaremos el criptosistema de McEliece incluyendo algunas de sus variantes y estudiaremos su seguridad y los ataques más conocidos.

\subsubsection{Original construcción}

El criptosistema de McEliece usa códigos lineales de corrección de errores para cifrar mensajes. Este criptosistema posee una clave privada, que se elige al generar la clave y contiene la descripción del código lineal estructurado, y una clave pública que se basa en una versión suficientemente aleatorizada de ese mismo código. De esta forma, será difícil descifrar un mensaje sin conocer la estructura del código lineal (clave privada), pues es lo que proporciona un descifrado rápido.

Los códigos que usa la construcción original son los códigos de Goppa binarios irreducibles. Estos códigos son muy adecuados pues poseen altas capacidades de corrección de erorres y matrices generadoras densas, que son difíciles de distinguir de matrices binarias aleatorias.

\subsubsection{Generación de claves}

La generación de la clave de este criptosistema se obtiene a partir de los siguientes pasos:

\begin{enumerate}
    \item Se elige un $[n, k, 2t + 1]$-código lineal aleatorio $\mathcal{C}$ sobre $\mathbb{F}_2$ que tenga un algoritmo de decodificación eficiente $\mathcal{D}$ que sea capaz de corregir hasta $t$ errores.
    \item Se calcula la matriz generadora $G$ de dimensión $k \times n$ para $\mathcal{C}$.
    \item Se genera una matriz no singular binaria aleatoria $S$ de dimensión $k \times k$.
    \item Se genera una matriz de permutaciones aleatoria $P$ de dimensión $n \times n$.
    \item Se calcula la matriz $G' = SGP$ de dimensión $k \times n$. La clave pública es $(G', t)$ y la clave privada es $(S, G, P, \mathcal{D})$.
\end{enumerate}

Es decir, la matriz $G'$ se obtiene al permutar las columnas de $G$ a partir de la matriz $P$ y luego realizar un cambio de base con la matriz $S$. De esta forma, la matriz $G'$ corresponde a un $[n, k, 2t + 1]$-código lineal que es equivalente permutacionalmente a la clave privada elegida. Llamaremos a $G'$ la \emph{matriz generadora pública}.

\subsubsection{Cifrado y descifrado}

Una vez tenemos la clave, podemos \textbf{cifrar} un texto plano $\textbf{m} \in \{ 0, 1 \}^k$ eligiendo un vector aleatorio $\textbf{e} \in \{ 0, 1 \}^n$ de peso $t$ y calcular el texto cifrado como
\[
    \textbf{c} = \textbf{m} G' + \textbf{e}.
\]
Para recuperar $\textbf{m}$ eficientemente, podemos usar el siguiente algoritmo de \textbf{descifrado}. Sea un criptograma $\textbf{c} \in \{ 0, 1 \}^n$, primero calculamos
\[
    \textbf{c} P^{-1} = (\textbf{m} S) G + \textbf{e} P^{-1}.
\]
Ahora ya podemos aplicar el algoritmo de descifrado $\mathcal{D}$ a $\textbf{c} P^{-1}$ para obtener $\textbf{c'} = \textbf{m} S$, pues $(\textbf{m} S) G$ es una palabra código válida para el código lineal elegido y $\textbf{e} P^{-1}$ tiene peso $t$. Finalmente, podemos calcular el mensaje $\textbf{m}$ con
\[
    \textbf{m} = \textbf{c'} S^{-1}.
\]
Por otra parte, a la hora de aplicar el algoritmo de Sugiyama, sabemos que este algoritmo de descifrado necesita saber el polinomio generador. De este modo consideraremos la clave privada como $(S, G, P, g(x))$, donde $g(x)$ es el polilnomio de Goppa para el código elegido.

% TODO
\textcolor{red}{Poner ejemplos usando la implementación de este criptosistema.}


\subsection{Criptosistema de Niederreiter}

En 1986, H. Niederreiter \cite{Niederreiter_1986} propuso una variante relevante del criptosistema de McEliece. A diferencia del criptosistema de McEliece, este criptosistema usa una matriz de paridad en vez de una matriz generadora.

Antes de introducir este criptosistema, necesitamos definir el concepto de síndrome de una palabra en $\mathbb{F}^n$.

\begin{definition}
    Sea $\mathcal{C}$ un $[n, k, d]$-código lineal sobre $\mathbb{F}$ y sea $H$ su matriz de paridad. El \emph{síndrome} de una palabra $\textbf{y} \in \mathbb{F}^n$ se define como
    \[
        \textbf{s} = \textbf{y} H^T.
    \]
\end{definition}

Sabemos que una palabra código de $\mathcal{C}$ tiene síndrome igual a $\textbf{0}$, según la definición de la matriz de paridad. Sean $\textbf{y}_1$, $\textbf{y}_2 \in \mathbb{F}^n$ dos vectores, entonces
\[
    \textbf{y}_1 - \textbf{y}_2 \in \mathcal{C} \Leftrightarrow (\textbf{y}_1 - \textbf{y}_2) H^T = \textbf{0} \Leftrightarrow \textbf{y}_1 H^T = \textbf{y}_2 H^T.
\]
El hecho de que $\textbf{y}_1 - \textbf{y}_2$ sea una palabra código de $\mathcal{C}$ si y solo si los síndromes de $\textbf{y}_1$ e $\textbf{y}_2$ son iguales es la base para un método eficiente para implementar la decodificación de la palabra código más cercana que se llama \emph{decodificación del síndrome}.

Sea una palabra $\textbf{y} \in \mathbb{F}^n$, un algoritmo de decodificación de síndrome $\mathcal{D}$ encuentra una palabra con el mínimo peso $\textbf{e} \in \mathbb{F}^n$ tal que
\[
    \textbf{y} H^T = \textbf{e} H^T.
\]
Si $\textbf{y}$ tiene la forma $\textbf{y} = \textbf{c} + \textbf{e'}$, donde $\textbf{c} \in \mathcal{C}$ y $\textbf{w}(\textbf{e'}) \leq t$, entonces $\textbf{e} = \textbf{e'}$, esto es, el algoritmo de decodificación de síndrome encuentra exactamente el vector error que se le añadió a la palabra código.

Ahora, ya podemos definir el criptosistema de Niederreiter, que se basa en la idea de la decodificación del síndrome.

\subsubsection{Generación de claves}

La generación de la clave de este criptosistema se obtiene a partir de los siguientes pasos:

\begin{enumerate}
    \item Se elige un $[n, k, 2t + 1]$-código lineal aleatorio $\mathcal{C}$ sobre $\mathbb{F}_2$ que tenga un algoritmo de decodificación del síndrome eficiente $\mathcal{D}$ que sea capaz de corregir hasta $t$ errores.
    \item Se calcula la matriz de paridad $H$ de dimensión $(n - k) \times n$ para $\mathcal{C}$.
    \item Se genera una matriz no singular binaria aleatoria $S$ de dimensión $(n - k) \times (n - k)$.
    \item Se genera una matriz de permutaciones aleatoria $P$ de dimensión $n \times n$.
    \item Se calcula la matriz $H' = SHP$ de dimensión $(n - k) \times n$. La clave pública es $(H', t)$ y la clave privada es $(S, H, P, \mathcal{D})$.
\end{enumerate}

\subsubsection{Cifrado y descifrado}

Una vez tenemos la clave, podemos \textbf{cifrar} un texto plano $\textbf{m} \in \{ 0, 1 \}^n$ con peso $t$, calculando el texto cifrado como el síndrome de $\textbf{m}$
\[
    \textbf{c} = \textbf{m} H^{'T}.
\]
Para recuperar $\textbf{m}$ eficientemente, podemos usar el siguiente algoritmo de \textbf{descifrado}. Sea un criptograma $\textbf{c}$, primero calculamos
\[
    S^{-1} \textbf{c}^T = H P \textbf{m}^T.
\]
Luego, buscamos un vector $\textbf{z} \in \mathbb{F}^n$ tal que $H \textbf{z}^T = HP \textbf{m}^T$. Entonces $\textbf{z} - (P\textbf{m}^T)^T = \textbf{z} - \textbf{m}P^T$ es una palabra código válida en $\mathcal{C}$ debido a que los síndromes de las palabras son iguales. Como $\textbf{m}P^T$ tiene peso $t$, podemos aplicar $\mathcal{D}$ a $\textbf{z}$ para encontrar el vector error $\textbf{m}P^T$ y por lo tanto $\textbf{m}$.

% TODO
\textcolor{red}{Hay que desarrollar y explicar mejor el párrafo de arriba.}

Observemos que el mensaje en texto plano se representa como el error de la palabra código en lugar de la palabra con la información original. De este modo, el mensaje en texto plano debe codificarse adicionalmente en un vector de peso $t$ para el cifrado.

% TODO
\textcolor{red}{No entiendo qué dice de hacerle al mensaje en texto plano para poder cifrarlo.}

A diferencia con el criptosistema de McEliece, el algoritmo de decodificación del síndrome es más eficiente que el algoritmo de decodificación del criptosistema de McEliece. Además, tambien se puede usar para construir un esquema de firma digital.

Sin embargo, si un ataque es capaz de romper el criptosistema de McEliece, también puede romper el esquema Niederreiter y viceversa. Esto se debe a que los criptosistemas de McEliece y Niederreiter se basan en el uso de códigos lineales que son duales entre si y la matriz generadora se puede obtener eficientemente a partir de la matriz de paridad y viceversa.

% TODO
\textcolor{red}{Poner ejemplos usando la implementación de este criptosistema.}


\subsection{Seguridad del criptosistema de McEliece}

En esta sección estudiaremos la seguridad del criptosistema de McEliece original y analizaremos los ataques más conocidos ante este criptosistema. Como la seguridad de los esquemas de McEliece y Niederreiter es equivalente \cite{Equivalence_McEliece_Niederreiter}, todos los ataques que analicemos también estarán relacionados implícitamente con el esquema Niederreiter.

Fijemos un código de Goppa $\Gamma(L, g) \subset \mathbb{F}_2^n$, con $g(x) \in \mathbb{F}_{2^m}[x]$ y $L$ una tupla de $n$ elementos distintos de $\mathbb{F}_{2^m}$, capaz de corregir hasta $t$ errores. Tenemos que la dimensión del código es $k = n - tm$. Sea $G$ la matriz generadora binaria de dimensión $k \times n$ de $\Gamma$ y definimos $G' = SGP$ como la clave pública de McEliece, donde $S$ es la matriz binaria no singular de dimensión $k \times k$ y $P$ es la matriz de permutaciones de dimensión $n \times n$.

Para estudiar la seguridad de este criptosistema, tenemos que determinar cómo de difícil es determinar el mensaje $\textbf{m}$ a partir de conocer $G'$ y haber interceptado $\textbf{c}$. Como se indica en \cite{McEliece_1978}, pueden darse dos ataques básicos:

\begin{enumerate}
    \item Intentar recuperar la clave secreta $G$ a partir de $G'$ y así descifrar el mensaje.
    \item Intentar recuperar el mensaje original $\textbf{m}$ a partir del criptograma $\textbf{c}$ sin conocer la clave privada $G$.
\end{enumerate}

El primer ataque parece inviable si $n$ y $t$ son suficientemente grandes, pues existen demasiadas posibilidades tanto para $G$ como para $S$ y $P$.

El segundo ataque puede ser más prometedor para el adversario pues puede aproximarse mediante la decodificación de conjuntos de información.

La seguridad del criptosistema de McEliece se puede insinuar por la intratabilidad de los siguientes problemas fundamentales en la teoría de la codificación.

\begin{problemth}[Problema general de decodificación para códigos lineales]
    Sea $\mathcal{C}$ un $[n, k]$-código lineal sobre $\mathbb{F}$ e $\textbf{y} \in \mathbb{F}^n$. Encontrar una palabra código $\textbf{c} \in \mathcal{C}$ tal que la distancia $d(\textbf{y}, \textbf{c})$ sea mínima.
\end{problemth}

\begin{problemth}[Problema de encontrar una palabra código dado un peso]
    Sea $\mathcal{C}$ un $[n, k]$-código lineal sobre $\mathbb{F}$ y $w \in \mathbb{N}$. Encontrar una palabra código $\textbf{c} \in \mathcal{C}$ tal que $\textbf{w}(\textbf{c}) = w$.
\end{problemth}

Se ha demostrado que ambos problemas son NP-duro \cite{Intractability_coding_problems}. No obstante, esto no implica que romper el criptosistema de McEliece sea NP-duro, pues los códigos binarios de Goppa solo cubren una fracción de todos los códigos lineales posibles. Es por esto que la seguridad del criptosistema de McEliece se basa en la suposición de que la clave pública es indistinguible de cualquier matriz aleatoria.

\subsubsection{Seguridad post-cuántica}

El criptosistema de McEliece es inmune ante el algoritmo de Shor \cite{Shor_1997}, que si se implementara en un ordenador cuántico podría romper otros criptosistemas de clave pública tales como RSA. 
% 6. Ataque usando algoritmos genéticos. 
% El ataque está basado en el cálculo de la distancia de un código. 
% Lo puedes encontrar en el archivo Cuellar_etal_2021.pdf (es un algoritmo genético muy sencillo). 
% Implementación en Sagemath de un supuesto ataque, a ver hasta que longitud puedes cargártelo (como mucho 512, la verdad)

\chapter{Ataque usando algoritmos genéticos}

% TODO introducción

\textcolor{red}{Añadir introducción }

\cite{Cuellar_etal}

\section{Esquema basado en permutaciones}

Sea $\mathbb{F}_q$ un cuerpo finito con $q$ elementos y sean $k,n$ dos números enteros tales que $0 < k \leq n$. Denotamos por $\mathcal{S}_n$ al conjunto de permutaciones de $n$ símbolos. Decimos que dos $[n, k]_q$-códigos lineales $\mathcal{C}_1$ y $\mathcal{C}_2$ son \emph{equivalentes permutacionalmente} si son iguales con una permutación fija de las coordenadas de una palabra código, esto es, si existe una permutación $x \in \mathcal{S}_n$ tal que $(c_0, ..., c_{n-1}) \in \mathcal{C}_1$ si y solo si $(c_{x(0)}, ..., c_{x(n-1)}) \in \mathcal{C}_2$. Observemos que $G$ es la matriz generadora de $\mathcal{C}_1$ si y solo si $GP$ es la matriz generadora de $\mathcal{C}_2$, donde $P$ es la matriz de permutaciones de $x$. Los códigos equivalentes permutacionalmente tienen en común la distancia mínima, ya que la permutación de sus componentes no modifica el peso del vector.

El siguiente resultado nos proporciona un método para calcular la distancia mínima a partir de la matriz escalonada (\emph{reduced row echelon form}) de la matriz generadora del código equivalente.

\begin{theorem}
    Sea $G$ una matriz generadora de dimensión $k \times n$ de un $[n, k]_q$-código lineal $\mathcal{C}$ sobre el cuerpo finito $\mathbb{F}_q$. Existe una permutación $x \in \mathcal{S}_n$ tal que la matriz escalonada, $R$, de $GP_x$, donde $P_x$ es la matriz de permutaciones de $x$, cumple que el peso de alguna de sus filas alcanza la mínima distancia de $\mathcal{C}$. Por lo tanto, si $b$ es una fila de $R$ verificando esa propiedad, entonces $bP_x^{-1}$ es una palabra código no nula de $\mathcal{C}$ con peso mínimo.
\end{theorem}

\begin{proof}
    Ver \cite[Página 4]{Cuellar_etal}.
\end{proof}

Este teorema expone que encontrar la distancia mínima de un $[n, k]_q$-código lineal se reduce a encontrar el mínimo de la aplicación $\mathfrak d : \mathcal{S}_n \rightarrow \mathbb{N}$ definida por
\[
    \mathfrak d (x) = \min \left\{ \text{w}(b) \; : \; b \text{ es una fila de la matriz escalonada de } FP_x \right\}.
\]
para cualquier $x \in \mathcal{S}_n$, donde $G$ es la matriz generadora del código y $P_x$ representa la matriz de permutaciones de $x$.

Sin embargo, existen pocas permutaciones que nos proporcionarán el mínimo de $\mathfrak{d}$. En concreto, supongamos que solo hay una palabra código que alcanza el mínimo peso $d$ de $\mathcal{C}$ salvo multiplicación escalar.

En \cite{Cuellar_etal} se obtiene que la probabilidad de encontrar el mínimo de $\mathfrak{d}$ por una búsqueda aleatoria es
\[
    \frac{d! (n-d)!}{n!} = {n \choose d}^{-1}.
\]

\begin{exampleth}
    Vamos a ilustrar el esquema descrito para obtener el peso mínimo de un código a partir de su matriz escalonada. Sea $\mathcal{C}$ un $[8, 4]_4$-código lineal sobre el cuerpo finito $\mathbb{F}_4 = \{ 0, 1, a, a + 1 \}$ con matriz generadora
    \[ G =
        \left(
        \begin{array}{cccccccc} 
            1 & 0 & a+1 & a+1 & a+1 & 0 & 0 & a  \\
            a+1 & 0 & 0 & a+1 & a & a+1 & 1 & a+1  \\
            1 & a+1 & 1 & a+1 & a & 0 & a & 0  \\
            0 & 0 & 0 & a & a & a & a+1 & a
        \end{array}
        \right).
    \]
    Sea $x \in \mathcal{S}_8$ la permutación dada por $x(0) = 1$, $x(1) = 0$, $x(2) = 3$, $x(3) = 2$, $x(4) = 5$, $x(5) = 4$, $x(6) = 7$ y $x(7) = 6$. La matriz de permutaciones de $x$ es la siguiente:
    \[ P_x =
        \left(
        \begin{array}{cccccccc} 
            0 & 1 & 0 & 0 & 0 & 0 & 0 & 0  \\
            1 & 0 & 0 & 0 & 0 & 0 & 0 & 0  \\
            0 & 0 & 0 & 1 & 0 & 0 & 0 & 0  \\
            0 & 0 & 1 & 0 & 0 & 0 & 0 & 0  \\
            0 & 0 & 0 & 0 & 0 & 1 & 0 & 0  \\
            0 & 0 & 0 & 0 & 1 & 0 & 0 & 0  \\
            0 & 0 & 0 & 0 & 0 & 0 & 0 & 1  \\
            0 & 0 & 0 & 0 & 0 & 0 & 1 & 0 
        \end{array}
        \right).
    \]
    Aplicamos esta permutación a la matriz generadora $G$ para obtener la matriz generadora $G^x$ del código equivalente permutacionalmente $\mathcal{C}_x$:
    \[ G^x =
        \left(
        \begin{array}{cccccccc} 
            0 & 1 & a+1 & a+1 & 0 & a+1 & a & 0  \\
            0 & a+1 & a+1 & 0 & a+1 & a & a+1 & 1  \\
            a+1 & 1 & a+1 & 1 & 0 & a & 0 & a  \\
            0 & 0 & a & 0 & a & a & a & a+1
        \end{array}
        \right).
    \]
    La matriz escalonada de $G^x$ es
    \[ G^x =
        \left(
        \begin{array}{cccccccc} 
            1 & 0 & 0 & 0 & a+1 & 0 & a & a  \\
            0 & 1 & 0 & 0 & 0 & a & 0 & 0  \\
            0 & 0 & 1 & 0 & 1 & 1 & 1 & a  \\
            0 & 0 & 0 & 1 & 1 & a+1 & a & a
        \end{array}
        \right),
    \]
    donde el peso de la primera fila es $4$, el de la segunda es $2$ y el de la tercera y cuarta es $5$. Como la segunda fila tiene peso mínimo entre todas las filas, tenemos que $\mathfrak{d} (x) = 2$. Realmente, la distancia de $\mathcal{C}$ es $2$, luego $x$ alcanza el mínimo de $\mathfrak{d}$.
\end{exampleth}

\section{Algoritmos}

% TODO introducción

\textcolor{red}{Añadir introducción}

En ambos algoritmos, la función fitness consistirá en minimizar el peso que producen las soluciones que componen la población.

\subsection{Algoritmo Genético Generacional (GGA)}

El algoritmo comienza con la inicialización de la población $P(t)$ con $N$ soluciones aleatorias y la evalúa en la iteración $t = 0$. Luego, el principal bucle del algoritmo se ejecuta hasta que se cumpla la condición de parada, que en este caso será un número máximo de generaciones. 

El bucle principal empieza seleccionando $N$ padres según el operador de selección de torneo binario. Este operador consiste en seleccionar dos individuos de la población aleatorios y seleccionar como padre al individuo que tenga mejor fitness.

Después, comienza la etapa de generación de una nueva población. De esta forma, se le aplica el operador de cruce a dos padres elegidos aleatoriamente para generar un nuevo par de soluciones con probabilidad $p_c$. Este operador consiste en componer ambos padres, es decir, sean $p_1$ y $p_2$ dos padres, el operador de cruce generará dos descencientes $d_1$ y $d_2$ a partir de la composición de cada uno en distintos órdenes:
\[
    d_1 = p_1 \circ p_2,
\]
\[
    d_2 = p_2 \circ p_1.
\]
Si no se combinan, se le aplica el operador de mutación a cada padre para generar una solución mutada. Este operador elige una columna de las primeras $k$ columnas y otra de las $n-k$ columnas restantes de la matriz generadora y las permuta para crear un nuevo descendiente.

En ambos casos, habrá que comprobar que las nuevas soluciones son válidas. Esto es, una solución no válida en el algoritmo GGA es la solución nula $(0,...,0)$. De esta forma, todas las nuevas soluciones $N$ se habrán generado por cruce o mutación y formarán la población de la siguiente iteración $P(t+1)$. Finalmente, se evalúan las soluciones de $P(t+1)$.

Se ha incluído una componente elitista antes de que comience la siguiente iteración: si la solución $P(t+1)$ no tiene un fitness igual o superior que la mejor en $P(t)$, entonces la peor solución de $P(t+1)$ es reemplazada por la mejor solución de $P(t)$. Además, si se alcanza un número fijo de evaluaciones de las soluciones que no producen mejora en el fitness de la mejor solución encontrada, se reinicializará $P(t+1)$ con $N$ nuevas soluciones aleatorias.

En resumen, el Algoritmo Genético Generacional (GGA) consiste en:

\begin{Ualgorithm}[H]
    \DontPrintSemicolon
    \KwIn{$N$: número par con el tamaño de la población}
    \KwIn{$p_c$: probabilidad de cruce}
    \KwIn{$MaxReinit$: número de evaluaciones de las soluciones sin mejorar el fitness antes de la reinicialización}
    \KwOut{Mejor solución de $P(t)$}
    $t \longleftarrow 0$\;
    Inicializar la población $P(t)$ con $N$ soluciones aleatorias válidas\;
    Evaluar las soluciones de $P(t)$\;
    \While{no se cumpla la condición de parada}{
        $P(t+1) \longleftarrow \emptyset$\;
        $parents(1..N) \longleftarrow$ Seleccionar $N$ soluciones de $P(t)$ con selección de torneo binario\;
        \For{$i$ in $0..N/2 - 1$}{
            \eIf{número aleatorio de la distribución uniforme $[0,1]$ es menor que $p_c$]}{
                $c_1, c_2 \longleftarrow$ soluciones generadar a partir del cruce de los padres $parents(2i + 1)$ y $parents(2i + 2)$\;
            }
            {
                $c_1 \longleftarrow$ mutación del padre $parents(2i+1)$\;
                $c_2 \longleftarrow$ mutación del padre $parents(2i+2)$\;
            }
            \If{$c_1$ (resp. $c_2$) no es válido}{
                reemplazar $c_1$ (resp. $c_2$) con una solución aleatoria válida\;
            }
            $P(t+1) \longleftarrow P(t+1) \bigcup \{ c_1, c_2 \}$\;
        }
        Evaluar las soluciones de $P(t+1)$\;
        \If{ningún fitness de $P(t+1)$ es igual o superior que el mejor fitness de $P(t)$}{
            Reemplazar la peor solución de $P(t+1)$ con la mejor solución de $P(t)$\;
        }
        \If{MaxReinit soluciones han sido evaluadas sin mejorar la mejor solución de $P(t+1)$}{
            Reemplazar las soluciones de $P(t+1)$ con $N-1$ soluciones aleatorias y la mejor solución de $P(t)$\;
        }
        $t \longleftarrow t + 1$\;
    }
    \caption{Algoritmo Genético Generacional (GGA).}
\end{Ualgorithm}


\subsection{Algoritmo de Mutación Clataclísmico (CHC)}

\textcolor{red}{Añadir}


% TODO: Conclusión
\newpage
\null
\thispagestyle{empty}

\chapter*{Conclusión}
\addcontentsline{toc}{chapter}{Conclusión} 

% TODO: poner aquí la conclusión
Añadir conclusión

\newpage
\null
\thispagestyle{empty}

\appendix
\chapter[Implementación en SageMath de los códigos de Goppa]{Implementación en SageMath de los códigos de Goppa}
\label{annex:sage-Goppa}

En este anexo describimos la documentación de las clases desarrolladas para implementar en SageMath los códigos de Goppa. Para ello, primero se ha implementado la clase \texttt{Goppa} que hereda de la clase \texttt{AbstractLinearCode} de SageMath, que permite representar los aspectos básicos de los códigos de Goppa, tales como la matriz de paridad y la matriz generadora. Para representar la codificación de los códigos de Goppa ha sido necesario crear la clase \texttt{GoppaEncoder}, que es capaz de proporcionar una palabra código a partir de un mensaje. Finalmente, se ha implementado la clase \texttt{GoppaDecoder} para simular la decodificación de los códigos de Goppa, esto es, se han desarrollado métodos para calcular el síndrome de una palabra codificada y el algoritmo de Sugiyama para transformar una palabra codificada con posibles errores en la palabra código original.

% TODO: GoppaEncoder y GoppaDecoder heredan, hay que mencionarlo

\section{Clase para códigos de Goppa}

Esta clase simula el comportamiento básico de los códigos de Goppa, es decir, proporciona métodos para calcular el código de Goppa asociado a un conjunto de definición y un polinomio sobre un cuerpo finito. Además, proporciona métodos para obtener las matrices de paridad y generadora.

\begin{description}[leftmargin=1em, font=\normalfont\ttfamily, style=nextline]
    \item[class Goppa(self, defining\_set, generating\_pol, field)]
    
    \emph{Hereda de:} \texttt{AbstractLinearCode}
  
    Representación de un código de Goppa como un código lineal.
  
    \textsc{Argumentos}
    \begin{description}[font=\normalfont\ttfamily]
        \item[field] Cuerpo finito sobre el que se define el código de Goppa.
        \item[generating\_pol] Polinomio mónico con coeficientes en un cuerpo finito $GF(p^m)$ que extiende de \texttt{field}.
        \item[defining\_set] Tupla de $n$ elementos distintos de $GF(p^m)$ que no son las raíces de \texttt{generating\_pol}.
    \end{description}

    \textsc{Ejemplos}
    \begin{lstlisting}[gobble=4]
        % TODO
    \end{lstlisting}
    \textcolor{red}{Completar ejemplo}

    \begin{description}[font=\ttfamily, style=nextline]
        \item[get\_generating\_pol(self)] Devuelve el polinomio generador del código.
        
        \textsc{Ejemplos}
        \begin{lstlisting}[gobble=4]
            % TODO
        \end{lstlisting}
        \textcolor{red}{Completar ejemplo}

        \item[get\_defining\_set(self)] Devuelve el conjunto de definición del código.

        \textsc{Ejemplos}
        \begin{lstlisting}[gobble=4]
            % TODO
        \end{lstlisting}
        \textcolor{red}{Completar ejemplo}

        \item[get\_parity\_pol(self)] Devuelve el polinomio de paridad del código.

        \textsc{Ejemplos}
        \begin{lstlisting}[gobble=4]
            % TODO
        \end{lstlisting}
        \textcolor{red}{Completar ejemplo}

        \item[get\_parity\_check\_matrix(self)] Devuelve la matriz de paridad del código.

        \textsc{Ejemplos}
        \begin{lstlisting}[gobble=4]
            % TODO
        \end{lstlisting}
        \textcolor{red}{Completar ejemplo}

        \item[get\_generator\_matrix(self)] Devuelve la matriz generadora del código.

        \textsc{Ejemplos}
        \begin{lstlisting}[gobble=4]
            % TODO
        \end{lstlisting}
        \textcolor{red}{Completar ejemplo}

        \item[get\_dimension(self)] Devuelve la dimensión del código.

        \textsc{Ejemplos}
        \begin{lstlisting}[gobble=4]
            % TODO
        \end{lstlisting}
        \textcolor{red}{Completar ejemplo}
    \end{description}
\end{description}

\section{Codificador para códigos de Goppa}

En esta sección presentaremos la implementación del codificador para los códigos de Goppa.

\begin{description}[leftmargin=1em, font=\normalfont\ttfamily, style=nextline]
    \item[class GoppaEncoder(self, code)]
    
    \emph{Hereda de:} \texttt{Encoder}
  
    Representación de un codificador para un código de Goppa usando su matriz generadora.
  
    \textsc{Argumentos}
    \begin{description}[font=\normalfont\ttfamily]
        \item[code] Código asociado a este codificador.
    \end{description}

    \textsc{Ejemplos}
    \begin{lstlisting}[gobble=4]
        % TODO
    \end{lstlisting}
    \textcolor{red}{Completar ejemplo}

    \begin{description}[font=\ttfamily, style=nextline]
        \item[get\_generator\_matrix(self)] Devuelve la matriz generadora del código asociado a \texttt{self}.

        \textsc{Ejemplos}
        \begin{lstlisting}[gobble=4]
            % TODO
        \end{lstlisting}
        \textcolor{red}{Completar ejemplo}

        \item[encode(self, m)] Transforma \texttt{m} en una palabra código del código asociado a \texttt{self}.

        \textsc{Argumentos}
        \begin{description}[font=\normalfont\ttfamily]
            \item[m] Vector asociado a un mensaje de \texttt{self}.
        \end{description}

        \textsc{Salida}
        \begin{description}[font=\normalfont\ttfamily]
            \item[] Vector asociado a una palabra código del código asociado a \texttt{self}.
        \end{description}

        \textsc{Ejemplos}
        \begin{lstlisting}[gobble=4]
            % TODO
        \end{lstlisting}
        \textcolor{red}{Completar ejemplo}
    \end{description}
\end{description}

\section{Decodificador para códigos de Goppa}

En esta sección presentaremos la implementación del decodificador para los códigos de Goppa. Este decodificador usa el algoritmo de Sugiyama descrito en \ref{th:alg-Sugiyama}.

\begin{description}[leftmargin=1em, font=\normalfont\ttfamily, style=nextline]
    \item[class GoppaDecoder(self, code)]
    
    \emph{Hereda de:} \texttt{Decoder}
  
    Representación de un decodificador para un código de Goppa usando el algoritmo de Sugiyama.
  
    \textsc{Argumentos}
    \begin{description}[font=\normalfont\ttfamily]
        \item[code] Código asociado a este decodificador.
    \end{description}

    \textsc{Ejemplos}
    \begin{lstlisting}[gobble=4]
        % TODO
    \end{lstlisting}
    \textcolor{red}{Completar ejemplo}

    \begin{description}[font=\ttfamily, style=nextline]
        \item[get\_syndrome(self, c)] Devuelve el polinomio síndrome asociado a la palabra código \texttt{c}.
        
        \textsc{Argumentos}
        \begin{description}[font=\normalfont\ttfamily]
            \item[c] Vector del espacio de entrada del código asociado a \texttt{self}.
        \end{description}

        \textsc{Salida}
        \begin{description}[font=\normalfont\ttfamily]
            \item[] Polinomio síndrome asociado al elemento \texttt{c}.
        \end{description}

        \textsc{Ejemplos}
        \begin{lstlisting}[gobble=4]
            % TODO
        \end{lstlisting}
        \textcolor{red}{Completar ejemplo}

        \item[get\_generating\_pol(self)] Devuelve el polinomio generador del código asociado a \texttt{self}.
        
        \textsc{Ejemplos}
        \begin{lstlisting}[gobble=4]
            % TODO
        \end{lstlisting}
        \textcolor{red}{Completar ejemplo}

        \item[decote\_to\_code(self, word)] Corrije los errores de \texttt{word} y devuelve una palabra código del código asociado a \texttt{self}.
         
        \textsc{Argumentos}
        \begin{description}[font=\normalfont\ttfamily]
            \item[word] Vector del espacio de entrada del código asociado a \texttt{self}.
        \end{description}

        \textsc{Salida}
        \begin{description}[font=\normalfont\ttfamily]
            \item[] Vector asociado a una palabra código del código asociado a \texttt{self}.
        \end{description}

        \textsc{Ejemplos}
        \begin{lstlisting}[gobble=4]
            % TODO
        \end{lstlisting}
        \textcolor{red}{Completar ejemplo}
    \end{description}
\end{description}
\chapter[Implementación en SageMath del criptosistema de McEliece]{Implementación en SageMath del criptosistema de McEliece}
\label{annex:sage-McEliece}

\section{Clase para el criptosistema McEliece}

\begin{description}[leftmargin=1em, font=\normalfont\ttfamily, style=nextline]
    \item[class McEliece(self, n, p, q, g)]
  
    Representación del criptosistema de McEliece
  
    \textsc{Argumentos}
    \begin{description}[font=\normalfont\ttfamily]
        % TODO
        \item[]
    \end{description}

    \textsc{Ejemplos}
    \begin{lstlisting}[gobble=4]
        % TODO
    \end{lstlisting}

    \begin{description}[font=\ttfamily, style=nextline]
        \item[get\_public\_key(self)] Devuelve la matriz generadora pública que es parte de la clave pública.

        \textsc{Ejemplos}
        \begin{lstlisting}[gobble=4]
            % TODO
        \end{lstlisting}

        \item[encrypt(self, m)] Devuelve el criptograma asociado al texto plano \texttt{m}.

        \textsc{Argumentos}
        \begin{description}[font=\normalfont\ttfamily]
            \item[m] Texto plano que se va a encriptar.
        \end{description}
        
        \textsc{Ejemplos}
        \begin{lstlisting}[gobble=4]
            % TODO
        \end{lstlisting}

        \item[decrypt(self, c)] Devuelve el texto plano asociado al criptograma \texttt{c}.

        \textsc{Argumentos}
        \begin{description}[font=\normalfont\ttfamily]
            \item[c] Criptograma que se va a desencriptar.
        \end{description}
        
        \textsc{Ejemplos}
        \begin{lstlisting}[gobble=4]
            % TODO
        \end{lstlisting}
    \end{description}
\end{description}

\section{Funciones auxiliares}

\begin{description}[leftmargin=1em, font=\normalfont\ttfamily, style=nextline]
    \item[get\_weight(c)]
  
    Calcula el peso del vector \texttt{c}.
  
    \textsc{Argumentos}
    \begin{description}[font=\normalfont\ttfamily]
        \item[c] Vector al que se le va a calcular su peso.
    \end{description}

    \textsc{Ejemplos}
    \begin{lstlisting}[gobble=4]
        % TODO
    \end{lstlisting}
\end{description}
\chapter[Implementación en SageMath del criptosistema de Niederreiter]{Implementación en SageMath del criptosistema de Niederreiter}
\label{annex:sage-Niederreiter}
\chapter[Implementación en SageMath de los algoritmos genéticos]{Implementación en SageMath de los algoritmos genéticos}
\label{annex:sage-geneticos}

% TODO:

El código desarrollado se encuentra en
\begin{center}
    \url{https://github.com/paula1999/TFG/tree/main/src}.
\end{center}


\section{Algoritmo genético GGA}

\section{Algoritmo genético CHC}



% ----------------------- %
% BIBLIOGRAFÍA
% ----------------------- %

% Estilo de cita.
\bibliographystyle{unsrtnat}
\renewcommand\bibname{Reference}

%[citestyle=numeric]

\renewcommand\bibname{Bibliografía}
\bibliography{bib/library}

% Añadimos la bibliografía al índice
\phantomsection
\addcontentsline{toc}{chapter}{Bibliografía}

\end{document}
