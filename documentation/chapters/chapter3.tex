% 2. Códigos de Goppa.
% Esto lo puedes encontrar en el capítulo 13 de Huffman y Pless. 
% La decodificación  de estos códigos, es similar a la de los códigos BCH (capítulo 5) utilizando el algoritmo de Sugiyama. 
% El artículo donde se definen y se decodifican es el de Goppa.pdf. 
% Además, ahi se explica de forma elemental, sin utilizar geometría algebraica.

\chapter{Códigos de Goppa}

% TODO: introduccion

Goppa describió una nueva clase de códigos de corrección de errores lineales no cíclicos. Lo más importante es que algunos de estos códigos excedían el límite asintótico de Gilbert-Varshamov, una hazaña que muchos teóricos códigos pensaban que nunca podría lograrse. En este capítulo vamos a hablar sobre los códigos de Goppa y sus propiedades más importantes.


\section{Algunos códigos clásicos}

Antes de estudiar los códigos de Goppa, introduciremos algunos códigos clásicos para motivar el objetivo de este capítulo.

\subsection{Códigos Reed-Solomon generalizados}

Para $k \geq 0$, $\mathcal{P}_k$ denota el conjunto de polinomios de grado menor que $k$, incluyendo el polinomio nulo, en $\mathbb{F}_q[x]$. Sea $n$ un número entero tal que $1 \leq n \leq q$, $\gamma = (\gamma _0,..., \gamma _{n-1})$ una n-tupla de elementos distintos de $\mathbb{F}_q$, y $\textbf{v} = (v_0,...,v_{n-1})$ una n-tupla de elementos no nulos de $\mathbb{F}_q$. Sea $k$ un número entero tal que $1 \leq k \leq n$. Entonces los códigos

$$GRS_k (\gamma, \textbf{v}) = \left\{ \left( v_0 f(\gamma_0), ..., v_{n-1}f(\gamma_{n-1}) \right) : f \in \mathcal{P}_k \right\}$$

son los códigos Reed-Solomon generalizados (códigos GRS).

\subsection{Códigos clásicos de Goppa}

Los códigos clásicos de Goppa se introdujeron por V. D. Goppa en 1970. Estos códigos son generalizaciones de códigos BCH y subcódigos de subcuerpos de ciertos códigos GRS.

Para motivar la definición de los códigos de Goppa, se introduce una construcción de los códigos BCH de longitud $n$ sobre $\mathbb{F}_q$. Sea $t = ord_q(n)$ y sea $\beta$ la raíz enésima primitiva de la unidad en $\mathbb{F}_{q^t}$. Se elige $\delta > 1$ y sea $\mathcal{C}$ el código BCH de longitud $n$ y distancia $\delta$. Entonces $c(x) = c_0 + c_1x + \cdots + c_{n-1}x^{n-1} \in \mathbb{F}_q [x] / (x^n - 1)$ está en $\mathcal{C}$ si y solo si $c(\beta^j) = 0$ para $1 \leq j \leq \delta - 1$. Tenemos que 

$$(x^n - 1) \sum_{i=0}^{n-1} \frac{c_i}{x - \beta ^{-i}} = \sum_{i=0}^{n-1} c_i \sum_{l=0}^{n-1} x^l \left( \beta ^{-i} \right) ^{n-1-l} = \sum_{l=0}^{n-1} x^l \sum_{i=0}^{n-1} c_i \left( \beta^{l+1} \right) ^i.$$

Como $c(\beta^{l+1}) = 0$ para $0 \leq l \leq \delta - 2$, el lado derecho de la ecuación es un polinomio cuyo término de menor grado tiene grado al menos $\delta - 1$. Por lo tanto, el lado derecho se puede escribir como $x^{\delta - 1} p(x)$, donde $p(x)$ es un polinomio en $\mathbb{F}_{q^t}[x]$. Así, se puede decir que $c(x) \in \mathbb{F}_q[x] / (x^n - 1)$ está en $\mathcal{C}$ si y solo si 

$$\sum_{i=0}^{n-1} \frac{c_i}{x - \beta ^{-i}} = \frac{x^{\delta - 1} p(x)}{x^n - 1}$$

o equivalentemente

$$\sum_{i=0}^{n-1} \frac{c_i}{x - \beta ^{-i}} \equiv 0 \pmod{x^{\delta - 1}}$$

La última equivalencia es la base para la definición de los códigos clásicos de Goppa.

Fijado el cuerpo de extensión $\mathbb{F}_{q^t}$ de $\mathbb{F}_q$, sea $L = \{ \gamma_0, ..., \gamma_{n-1} \}$ $n$ elementos distintos de $\mathbb{F}_{q^t}$ y sea $G(x) \in \mathbb{F}_{q^t}[x]$ con $G(\gamma_i) \neq 0$ para $0 \leq i \leq n - 1$. Entonces el \emph{código de Goppa} $\Gamma (L,G)$ es el conjunto de vectores $c_0 \cdots c_{n-1} \in \mathbb{F}_q^n$ tal que 

\begin{equation}
    \label{definicion_goppa}
    \sum_{i=0}^{n-1} \frac{c_i}{x - \gamma_i} \equiv 0 \pmod{G(x)}
\end{equation}

De esta forma, cuando la parte de la izquierda está escrita como una función racional, significa que el numerador es un múltiplo de $G(x)$. Además, trabajar con módulo $G(x)$ es como trabajar en el anillo $\mathbb{F}_{q^t}[x]/(G(x))$, y la hipótesis $G(\gamma_i) \neq 0$ garantiza que $x - \gamma_i$ es invertible en este anillo. Se llama a $G(x)$ el \emph{polinomio de Goppa} de $\Gamma (L,G)$. Nótese que el código BCH de longitud $n$ y la distancia elegida $\delta$ es el código de Goppa $\Gamma (L,G)$ con $L = \{ 1, \beta^{-1}, ..., \beta^{1-n} \}$ y $G(x) = x^{\delta - 1}$.

A continuación, se busca una matriz de paridad para $\Gamma (L,G)$. Para ello, se observa que

$$\frac{1}{x - \gamma_i} \equiv - \frac{1}{G(\gamma_i)} \frac{G(x) - G(\gamma_i)}{x - \gamma_i} \pmod{ G(x)}$$

ya que, comparando numeradores, $1 \equiv - G(\gamma_i)^{-1} \left( G(x) - G(\gamma_i) \right) \pmod{G(x)}$. Así que por \eqref{definicion_goppa} $\textbf{c} = c_0 \cdots c_{n-1} \in \Gamma (L,G)$ si y solo si

\begin{equation}
    \label{congruencia_goppa}
    \sum_{i=0}^{n-1} c_i \frac{G(x) - G(\gamma_i)}{x - \gamma_i} G(\gamma_i)^{-1} \equiv 0 \pmod{G(x)}
\end{equation}

Supongamos que $G(x) = \sum_{j=0}^w g_j x^j$ con $g_j \in \mathbb{F}_{q^t}$, donde $w = \deg{(G(x))}$. Entonces

$$\frac{G(x) - G(\gamma_i)}{x - \gamma_i} G(\gamma_i)^{-1} = G(\gamma_i)^{-1} \sum_{j=1}^w g_j \sum_{k=0}^{j-1} x^k \gamma_i^{j-1-k} = G(\gamma_i)^{-1} \sum_{k=0}^{w-1} x^k \left( \sum_{j=k+1}^w g_j \gamma_i^{j-1-k} \right)$$

Por lo tanto, por \eqref{congruencia_goppa}, estableciendo los coeficientes de $x^k$ iguales a $0$, en el orden $k = w - 1, w - 2, ..., 0$, tenemos que $\textbf{c} \in \Gamma (L,G)$ si y solo si $Hc^T = 0$, donde 

\begin{equation}
    H = \left(
        \begin{array}{ccc} 
            h_0 g_w & \cdots & h_{n-1} g_w  \\
            h_0 (g_{w-1} + g_w \gamma_0) & \cdots & h_{n-1} (g_{w-1} + g_w \gamma_{n-1}) \\
            & \vdots & \\
            h_0 \sum_{j=1}^w \left( g_j + \gamma_0^{j-1} \right) & \cdots & h_{n-1} \sum_{j=1}^w \left( g_{j} + \gamma_{n-1}^{j-1} \right) \\
        \end{array}
        \right)
\end{equation}

con $h_i = G(\gamma_i)^{-1}$.

\begin{proposition}
    La matriz $H$ se puede reducir a una matriz $H'$ de dimensión $w \times n$, donde 

    \begin{equation}
        H = \left(
            \begin{array}{ccc} 
                G(\gamma_0)^{-1} & \cdots & G(\gamma_{n-1})^{-1}  \\
                G(\gamma_0)^{-1} \gamma_0 & \cdots & G(\gamma_{n-1})^{-1} \gamma_{n-1} \\
                & \vdots & \\
                G(\gamma_0)^{-1} \gamma_0^{w-1} & \cdots & G(\gamma_{n-1})^{-1} \gamma_{n-1}^{w-1} \\
            \end{array}
            \right)
    \end{equation}
\end{proposition}

\begin{proof}
    % TODO
\end{proof}

\section{Decodificación de los códigos de Goppa}

