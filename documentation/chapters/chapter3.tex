% 2. Códigos de Goppa.
% Esto lo puedes encontrar en el capítulo 13 de Huffman y Pless. 
% La decodificación  de estos códigos, es similar a la de los códigos BCH (capítulo 5) utilizando el algoritmo de Sugiyama. 
% El artículo donde se definen y se decodifican es el de Goppa.pdf. 
% Además, ahi se explica de forma elemental, sin utilizar geometría algebraica.

\chapter{Códigos de Goppa}

En \cite{Goppa_codes_1973}, Goppa describió una nueva clase de códigos de corrección de errores lineales. Lo más importante es que algunos de estos códigos excedían el límite asintótico de Gilbert-Varshamov \cite{Varshamov-Gilbert_bound_1982}, una hazaña que muchos teóricos códigos pensaban que nunca podría lograrse. En este capítulo vamos a estudiar los códigos de Goppa \cite[Sección 13.2.2]{Huffman_Pless_2010} y sus propiedades más importantes, así como su codificación y decodificación \cite{Goppa_codes_1973}.

\section{Códigos clásicos de Goppa}

Fijado el cuerpo de extensión $\mathbb{F}_{q^t}$ de $\mathbb{F}_q$, sea $L = \{ \gamma_0, ..., \gamma_{n-1} \}$ una tupla de $n$ elementos distintos de $\mathbb{F}_{q^t}$ y sea $g(x) \in \mathbb{F}_{q^t}[x]$ con $g(\gamma_i) \neq 0$ para $0 \leq i \leq n - 1$. Entonces el \emph{código de Goppa} $\Gamma(L,g)$ es el conjunto de vectores $c_0 \cdots c_{n-1} \in \mathbb{F}_q^n$ tal que 
\begin{equation}
    \label{def:goppa}
    \sum_{i=0}^{n-1} \frac{c_i}{x - \gamma_i} \equiv 0 \pmod{g(x)}.
\end{equation}

De esta forma, cuando la parte de la izquierda está escrita como una función racional, significa que el numerador es un múltiplo de $g(x)$. Además, trabajar módulo $g(x)$ es como trabajar en el anillo $\mathbb{F}_{q^t}[x]/(g(x))$, y la hipótesis $g(\gamma_i) \neq 0$ garantiza que $x - \gamma_i$ es invertible en este anillo. Se llama a $g(x)$ el \emph{polinomio de Goppa} de $\Gamma(L,g)$.

A continuación, buscaremos una matriz de paridad para $\Gamma(L,g)$. Para ello, observamos que

\[
    \frac{1}{x - \gamma_i} \equiv - \frac{1}{g(\gamma_i)} \frac{g(x) - g(\gamma_i)}{x - \gamma_i} \pmod{ g(x)}
\]

ya que, comparando numeradores, $1 \equiv - g(\gamma_i)^{-1} \left( g(x) - g(\gamma_i) \right) \pmod{g(x)}$. Así que por \eqref{def:goppa} $\textbf{c} = c_0 \cdots c_{n-1} \in \Gamma(L,g)$ si y solo si
\begin{equation}
    \label{congruencia_goppa}
    \sum_{i=0}^{n-1} c_i \frac{g(x) - g(\gamma_i)}{x - \gamma_i} g(\gamma_i)^{-1} \equiv 0 \pmod{g(x)}.
\end{equation}

Supongamos que $g(x) = \sum_{j=0}^w g_j x^j$ con $g_j \in \mathbb{F}_{q^t}$, donde $w = \gr(g(x))$. Entonces

\[
    \frac{g(x) - g(\gamma_i)}{x - \gamma_i} g(\gamma_i)^{-1} = g(\gamma_i)^{-1} \sum_{j=1}^w g_j \sum_{k=0}^{j-1} x^k \gamma_i^{j-1-k} = g(\gamma_i)^{-1} \sum_{k=0}^{w-1} x^k \left( \sum_{j=k+1}^w g_j \gamma_i^{j-1-k} \right)
\]

Por lo tanto, por \eqref{congruencia_goppa}, estableciendo los coeficientes de $x^k$ iguales a $0$, en el orden $k = w - 1, w - 2, ..., 0$, tenemos que $\textbf{c} \in \Gamma(L,g)$ si y solo si $Hc^T = 0$, donde 
\begin{equation}
    H = \left(
        \begin{array}{ccc} 
            h_0 g_w & \cdots & h_{n-1} g_w  \\
            h_0 (g_{w-1} + g_w \gamma_0) & \cdots & h_{n-1} (g_{w-1} + g_w \gamma_{n-1}) \\
            & \vdots & \\
            h_0 \sum_{j=1}^w \left( g_j + \gamma_0^{j-1} \right) & \cdots & h_{n-1} \sum_{j=1}^w \left( g_{j} + \gamma_{n-1}^{j-1} \right) \\
        \end{array}
        \right),
\end{equation}

con $h_i = g(\gamma_i)^{-1}$.

% TODO
\textcolor{red}{Formalizar la otra forma de construir la matriz de paridad a partir de los coeficientes de los polinomios de paridad.}

\begin{proposition}
    La matriz $H$ se puede reducir a una matriz $H'$ de dimensión $w \times n$, donde 
    \begin{equation}
        H' = \left(
            \begin{array}{ccc} 
                g(\gamma_0)^{-1} & \cdots & g(\gamma_{n-1})^{-1}  \\
                g(\gamma_0)^{-1} \gamma_0 & \cdots & g(\gamma_{n-1})^{-1} \gamma_{n-1} \\
                & \vdots & \\
                g(\gamma_0)^{-1} \gamma_0^{w-1} & \cdots & g(\gamma_{n-1})^{-1} \gamma_{n-1}^{w-1} \\
            \end{array}
            \right).
    \end{equation}
\end{proposition}

Las entradas de $H'$ están en $\mathbb{F}_{q^t}$. Eligiendo una base de $\mathbb{F}_{q^t}$ sobre $\mathbb{F}_q$, cada elemento de $\mathbb{F}_{q^t}$ se puede representar como un vector columna $t \times 1$ sobre $\mathbb{F}_q$. Reemplazando cada entrada de $H'$ por su correspondiente vector columna, obtenemos una matriz $H''$ de dimensión $tw \times n$ sobre $\mathbb{F}_{q}$ que tiene la propiedad de que $\textbf{c} \in \mathbb{F}_q^n$ está en $\Gamma(L,g)$ si y solo si $H''c^T = 0$.

El siguiente resultado nos muestra los límites en la dimensión y la distancia mínima de un código de Goppa.

\begin{theorem}
    \label{th:dist_min_Goppa}
    Con la notación de esta sección, sea $\Gamma(L,g)$ un código de Goppa tal que $\gr(g(x)) = w$ entonces es un $[n, k, d]$ código con $k \geq n - wt$ y $d \geq w + 1$.
\end{theorem}

\begin{proof}
    Las filas de $H''$ pueden ser dependientes, luego esta matriz tiene rango como máximo $wt$. Por lo que $\Gamma(L,g)$ tiene dimensión al menos $n - wt$. Si una palabra código $\textbf{c} \in \Gamma(L,g)$ tiene peso $w$ o menos, entonces el lado izquierdo de \ref{def:goppa} es una función racional, donde el numerador tiene grado $w - 1$ o menos; pero este numerador tiene que ser múltiplo de $g(x)$, lo cual es una contradicción pues el grado de $g$ es $w$.
\end{proof}

\begin{corollary}
    Si $\Gamma(L,g)$ es un código de Goppa tal que $\gr(g(x)) = w$, entonces puede corregir hasta
    \[
        \left\lfloor \frac{w}{2} \right\rfloor
    \]
    errores.
\end{corollary}

\begin{proof}
    El teorema \ref{th:decodificacion_maxima_verosimilitud} afirma que es posible corregir hasta 
    \[
        \left\lfloor \frac{d(\Gamma(L,g)) - 1}{2} \right\rfloor
    \]
    errores. Además, por el teorema \ref{th:dist_min_Goppa} sabemos que la distancia mínima de un código de Goppa $\Gamma(L,g)$ tal que $\gr(g(x)) = w$ es $d(\Gamma(L,g)) = w + 1$. Por lo que
    \[
        \left\lfloor \frac{d(\Gamma(L,g)) - 1}{2} \right\rfloor = \left\lfloor \frac{(w + 1) - 1}{2} \right\rfloor = \left\lfloor \frac{w}{2} \right\rfloor .
    \]
\end{proof}

Para ilustrar la teoría que acabamos de ver sobre los códigos de Goppa, usaremos el código descrito en el apéndice \ref{annex:sage-Goppa} en los siguientes ejemplos.

\begin{exampleth}
    Sea $\mathbb{F}_{2^4}$ el cuerpo de extensión sobre $\mathbb{F}_{2^2}$. Definimos el polinomio $g(x) = x^3 + ax^2 + 1 \in \mathbb{F}_{2^4}[x]$ y tomamos $n = 10$ como longitud del conjunto de definición. Podemos obtener el código de Goppa definido por dicho conjunto de definición y polinomio $g$ escribiendo lo siguiente:

    \begin{lstlisting}[gobble=4]
        sage: F = GF(2^2)
        sage: L = GF(2^4)
        sage: a = L.gen()
        sage: b = F.gen()
        sage: R.<x> = L[]
        sage: g = x^3 + a*x^2 + 1
        sage: n = 10
        sage: defining_set = get_defining_set(n, g, L)
        sage: C = Goppa(defining_set, g, F)
        sage: C
        > [10, 4] Goppa code
    \end{lstlisting}

    Observamos que el código es un $[10, 4]$-código de Goppa, esto es, la longitud de este código es $10$ y la dimensión es $4$. Calculemos ahora la matriz de paridad y la matriz generadora asociada a este código.

    \begin{lstlisting}[gobble=4]
        sage: H = C.parity_check_matrix()
        sage: show(H)
    \end{lstlisting}

    La matriz de paridad $H$ que nos devuelve el código anterior es la siguiente:
    \[
        H = \left(\begin{array}{rrrrrrrrrr}
            0 & z_{2} & 0 & z_{2} + 1 & 1 & z_{2} + 1 & 1 & 0 & 0 & 1 \\
            1 & 1 & z_{2} + 1 & 1 & z_{2} + 1 & z_{2} + 1 & 0 & 0 & z_{2} + 1 & 0 \\
            1 & 1 & 1 & z_{2} + 1 & z_{2} & z_{2} & z_{2} + 1 & 0 & 0 & z_{2} \\
            1 & z_{2} & z_{2} + 1 & 0 & 1 & 0 & 1 & 0 & z_{2} & z_{2} \\
            0 & 1 & z_{2} & 1 & z_{2} & z_{2} & z_{2} & 1 & 1 & 1 \\
            z_{2} & z_{2} & 0 & z_{2} & z_{2} & 1 & z_{2} & 0 & 1 & z_{2}
            \end{array}
        \right).
    \]
    Análogamente, podemos calcular la matriz generadora $G$ como se indica a continuación.

    \begin{lstlisting}[gobble=4]
        sage: G = C.get_generator_matrix()
        sage: show(G)
    \end{lstlisting}

    Este código devuelve la siguiente matriz generadora:
    \[
        G = \left(\begin{array}{rrrrrrrrrr}
        1 & 0 & 0 & z_{2} + 1 & 0 & z_{2} + 1 & z_{2} & z_{2} + 1 & 0 & z_{2} \\
        0 & 1 & 0 & 0 & 0 & z_{2} + 1 & 0 & 1 & 1 & 0 \\
        0 & 0 & 1 & z_{2} & 0 & z_{2} + 1 & 0 & z_{2} + 1 & 1 & z_{2} + 1 \\
        0 & 0 & 0 & 0 & 1 & 1 & z_{2} + 1 & 0 & 0 & 1
        \end{array}\right).
    \]
    Observamos que efectivamente el rango de la matriz generadora coincide con la dimensión del código, que era $4$.

    Además, podemos comprobar efectivamente que se da la igualdad $H \cdot G^T = 0$:

    \begin{lstlisting}[gobble=4]
        sage: H*G.T == 0
        > True
    \end{lstlisting}
\end{exampleth}

\begin{exampleth}
    \label{ex:goppa-3_3}
    Sea $\mathbb{F}_{3^6}$ el cuerpo de extensión sobre $\mathbb{F}_{3^3}$. Definimos el polinomio $g(x) = x^2 + ax + 1 + a^2 \in \mathbb{F}_{3^6}[x]$ y tomamos $n = 10$ como longitud del conjunto de definición. Obtengamos el código de Goppa definidor por dicho conjunto de definición y polinomio $g$.

    \begin{lstlisting}[gobble=4]
        sage: F = GF(3^3)
        sage: L = GF(3^6)
        sage: a = L.gen()
        sage: b = F.gen()
        sage: R.<x> = L[]
        sage: g = x^2 + a*x + 1 + a^2
        sage: n = 10
        sage: defining_set = get_defining_set(n, g, L)
        sage: C = Goppa(defining_set, g, F)
        sage: C
        > [10, 6] Goppa code
    \end{lstlisting}
    
    Observamos que el código es un $[10, 6]$-código de Goppa, esto es, la longitud de este código es $10$ y la dimensión es $6$. Calculemos ahora la matriz de paridad y la matriz generadora asociada a este código.

    \begin{lstlisting}[gobble=4]
        sage: H = C.parity_check_matrix()
        sage: show(H)
    \end{lstlisting}

    La matriz de paridad $H$ que nos devuelve el código anterior es la siguiente:
    \[\scriptsize
        H = 
        \left(\begin{array}{rrrrrrrrrr}
        2 z_{3}^{2} + 2 z_{3} + 2 & z_{3}^{2} + 1 & 1 & z_{3}^{2} & 2 z_{3}^{2} + 2 z_{3} + 1 & 1 & 2 z_{3}^{2} + 2 z_{3} + 2 & z_{3} + 2 & z_{3}^{2} + 1 & z_{3}^{2} + 2 z_{3} + 1 \\
        z_{3}^{2} + 2 z_{3} + 2 & z_{3}^{2} + 1 & 2 & z_{3}^{2} + z_{3} & z_{3}^{2} + 2 z_{3} + 1 & z_{3}^{2} & z_{3}^{2} + 2 z_{3} + 2 & 1 & z_{3}^{2} & z_{3}^{2} + 1 \\
        2 z_{3}^{2} + z_{3} + 1 & z_{3} + 1 & 1 & z_{3}^{2} + 2 z_{3} & z_{3}^{2} + 1 & z_{3}^{2} + z_{3} & z_{3}^{2} + z_{3} + 1 & 2 z_{3} + 2 & z_{3}^{2} & 2 \\
        0 & z_{3} + 2 & 0 & z_{3}^{2} & z_{3}^{2} + 2 z_{3} & z_{3}^{2} + 2 z_{3} + 2 & z_{3} + 1 & 2 z_{3}^{2} + 2 z_{3} & z_{3}^{2} + z_{3} & 2 z_{3}^{2} + 2 z_{3} + 2
        \end{array}\right).
    \]
    Análogamente, podemos calcular la matriz generadora $G$ como se indica a continuación.

    \begin{lstlisting}[gobble=4]
        sage: G = C.get_generator_matrix()
        sage: show(G)
    \end{lstlisting}

    Este código devuelve la siguiente matriz generadora:
    \[ 
        \footnotesize
        G = 
        \left(\begin{array}{rrrrrrrrrr}
        1 & 0 & 0 & 0 & 0 & 0 & 2 z_{3} + 2 & 2 z_{3}^{2} + 1 & z_{3} + 1 & 2 z_{3}^{2} + 2 z_{3} + 1 \\
        0 & 1 & 0 & 0 & 0 & 0 & 2 z_{3}^{2} + 2 z_{3} + 1 & z_{3}^{2} + 2 z_{3} + 2 & 2 z_{3}^{2} + 2 & 2 z_{3}^{2} + 2 z_{3} \\
        0 & 0 & 1 & 0 & 0 & 0 & 2 z_{3}^{2} + z_{3} + 2 & 2 z_{3}^{2} + 2 z_{3} & 2 z_{3} + 2 & 2 z_{3}^{2} + 2 z_{3} \\
        0 & 0 & 0 & 1 & 0 & 0 & 2 z_{3}^{2} + z_{3} & z_{3} & 2 z_{3}^{2} + z_{3} & z_{3}^{2} + 2 z_{3} + 2 \\
        0 & 0 & 0 & 0 & 1 & 0 & z_{3} + 1 & 2 z_{3}^{2} + 2 z_{3} & z_{3} + 2 & z_{3}^{2} + 2 \\
        0 & 0 & 0 & 0 & 0 & 1 & z_{3}^{2} + 2 z_{3} + 1 & z_{3}^{2} + 1 & z_{3}^{2} + 2 & 2 z_{3}^{2} + 1
        \end{array}\right).
    \]
    Observamos que efectivamente el rango de la matriz generadora coincide con la dimensión del código, que era $6$.

    Además, podemos comprobar efectivamente que se da la igualdad $H \cdot G^T = 0$:

    \begin{lstlisting}[gobble=4]
        sage: H*G.T == 0
        > True
    \end{lstlisting}
\end{exampleth}

\begin{comment}
    Sea el cuerpo de extensión $\mathbb{F}_{2^3}$ de $\mathbb{F}_2$ y sea $g(x) = x^2 + x + 1 \in \mathbb{F}_{2^3}[x]$. Definimos $L$ como una tupla de $n$ elementos distintos de $\mathbb{F}_{q^t}$ tales que sus elementos no son raíces de $g$. 

    El código de Goppa asociado al conjunto de definición $L$ y al polinomio $g$ es un $[8, 2]$ código de Goppa. Este código tiene distancia mínima $5$ y su matriz de paridad es la siguiente:

    \[
        H = \left(
        \begin{array}{cccccccc} 
            1 & 1 & 1 & 1 & 1 & 1 & 1 & 0 \\
            0 & 1 & 1 & 1 & 0 & 1 & 0 & 0 \\
            0 & 1 & 0 & 0 & 1 & 1 & 1 & 0 \\
            1 & 0 & 0 & 0 & 0 & 0 & 0 & 1 \\
            0 & 0 & 1 & 0 & 1 & 1 & 1 & 0 \\
            0 & 1 & 1 & 1 & 0 & 0 & 1 & 0 \\
        \end{array}
        \right)    
    \]

    El codificador asociado al $[8, 2]$ código de Goppa tiene la siguiente matriz generadora:

    \[
        G = \left(
        \begin{array}{cccccccc} 
            1 & 0 & 0 & 1 & 0 & 1 & 1 & 1 \\
            0 & 1 & 1 & 1 & 1 & 1 & 1 & 0 \\
        \end{array}
        \right)    
    \]

    Sea $(0, 1) \in \mathbb{F}_2$ una palabra, la podemos codificar con la matriz anterior:

    $$x = (0, 1, 1, 1, 1, 1, 1, 0).$$

    Ahora, queremos enviar la palabra codificada $x$ a nuestro destinatario, pero para ello le añadimos un error en una posición aleatoria. Sea $e$ el error definido como:

    $$e = (0, 0, 0, 0, 1, 0, 0, 0).$$

    La palabra que enviaremos será:

    $$y = x + e = (0, 1, 1, 1, 0, 1, 1, 0).$$

    El destinatario decodificará esta palabra con el algoritmo de Sugiyama obteniendo:

    $$(0, 1, 1, 1, 1, 1, 1, 0),$$

    que coincide con $x$.
\end{comment}

\subsection{Códigos binarios de Goppa}

Los códigos binarios de Goppa son códigos de corrección de errores que pertenecen a la clase de los códigos de Goppa que acabamos de estudiar. La estructura binaria le da más ventajas matemáticas sobre variantes no binarias y, además, tienen propiedades interesantes adecuadas para la construcción del criptosistema McEliece.

\begin{definition}
    Fijada la extensión de cuerpos $\mathbb{F}_{2^m}$ de $\mathbb{F}_2$, sea $L = \{ \gamma_0, ..., \gamma_{n-1} \} \in \mathbb{F}_{2^m}^n$ una tupla de $n$ elementos distintos de $\mathbb{F}_{2^m}$ y sea $g(x) \in \mathbb{F}_{2^m}[x]$ con $g(\gamma_i) \neq 0$ para $0 \leq i \leq n - 1$. Entonces el \emph{código binario de Goppa} $\Gamma(L,g)$ es el conjunto de vectores $c_0 \cdots c_{n-1} \in \{ 0, 1 \}^n$ tal que 
    \begin{equation}
        \sum_{i=0}^{n-1} \frac{c_i}{x - \gamma_i} \equiv 0 \pmod{g(x)}
    \end{equation}
\end{definition}

Observemos que si $g(x)$ es un polinomio irreducible, todos los elementos $\gamma \in \mathbb{F}_{2^m}$ satisfacen $g(\gamma) \neq 0$. A los códigos que cumplan esta propiedad los llamaremos \emph{códigos binarios de Goppa irreducibles}.

La importancia de estos códigos se debe a que pueden doblar la capacidad correctora de los códigos generales de Goppa.

\subsection{Codificación de los códigos de Goppa}

La codificación de un mensaje consiste en escribirlo como una palabra código de un código. Fijado un código de Goppa $\Gamma (L, g)$ sobre $\mathbb{F}_q$ con matriz generadora $G$. Dado el mensaje $\textbf{m} \in \mathbb{F}_q$, lo podemos codificar realizando la operación
\[
    \textbf{c} = \textbf{m} G.  
\]
De esta forma obtenemos la palabra código $\textbf{c}$ que está en $\Gamma (L, g)$.

En los siguientes ejemplos vamos a usar el código descrito en el apéndice \ref{annex:sage-Goppa} para mostrar la codificación de algunos mensajes.

\begin{exampleth}
    Sea $\mathbb{F}_{3^6}$ el cuerpo de extensión sobre $\mathbb{F}_{3^3}$. Definimos el polinomio $g(x) = x^2 + ax + 1 + a^2 \in \mathbb{F}_{3^6}[x]$ y tomamos $n = 10$ como longitud del conjunto de definición. Sabemos que, por el ejemplo \ref{ex:goppa-3_3}, los mensajes deben tener longitud $6$. Dada la palabra
    \[
        word := (1, 0, b, b + 1, b, 0, b + 1) \in \mathbb{F}_2^2,
    \]
    la podemos transformar en una palabra código del código Goppa definido por el conjunto de definición y el polinomio $g$:

    \begin{lstlisting}[gobble=4]
        sage: F = GF(3^3)
        sage: L = GF(3^6)
        sage: a = L.gen()
        sage: b = F.gen()
        sage: R.<x> = L[]
        sage: g = x^2 + a*x + 1 + a^2
        sage: n = 10
        sage: defining_set = get_defining_set(n, g, L)
        sage: C = Goppa(defining_set, g, F)
        sage: E = GoppaEncoder(C)
        sage: word = vector(F, (1, 0, b + 1, b, 0, b + 1))
        sage: x = E.encode(word)
        sage: x
        > (1, 0, z3 + 1, z3, 0, z3 + 1, 2*z3, z3^2, z3 + 1, 0)
    \end{lstlisting}

    Por lo que la palabra $word$ codificada es la siguiente:
    \[
        (1, 0, z_3 + 1, z_3, 0, z_3 + 1, 2z_3, z_3^2, z_3 + 1, 0).
    \]
\end{exampleth}

\subsection{Decodificación de los códigos de Goppa}

Como hemos visto, al transmitir una palabra código a un receptor, éste podría recibir la palabra alterada. Para que el receptor pueda determinar el mensaje original, necesitaremos decodificar el mensaje recibido, \cite{Goppa_codes_1973}. Sea $\Gamma(L,g)$ un código de Goppa, donde $L = \{ \gamma_0, ..., \gamma_{n-1} \}$ es una tupla de $n$ elementos distintos de $\mathbb{F}_{q^t}$. Supongamos que $e$ es el vector de errores que se añade a la palabra código $c$ transmitida, entonces la palabra recibida $y$ está dada por
\[
    y = c + e,
\]
de donde 
\[
    \sum_{\gamma \in L} \frac{y_\gamma}{x - \gamma} = \sum_{\gamma \in L} \frac{c_\gamma}{x - \gamma} + \sum_{\gamma \in L} \frac{e_\gamma}{x - \gamma}.
\]
Como $c$ es una palabra código, la primera sumatoria de la parte derecha desaparece al aplicar el módulo $g(x)$, y tenemos que
\[
    \sum_{\gamma \in L} \frac{y_\gamma}{x - \gamma} = \sum_{\gamma \in L} \frac{e_\gamma}{x - \gamma} \pmod{g(x)}.
\]
Definiremos su síndrome como el polinomio $S(x)$ de grado menos que $\gr(g(x))$ tal que 
\[
    S(x) = \sum_{\gamma \in L} \frac{y_\gamma}{x - \gamma} \pmod{g(x)}.
\]
Acabamos de ver que 
\[
    S(x) = \sum_{\gamma \in L} \frac{e_\gamma}{x - \gamma} \pmod{g(x)}.
\]
Sea $M$ un subconjunto de $L$ tal que $e_{\gamma} \neq 0$ si y solo si $\gamma \in M$. Entonces
\begin{equation}
    \label{def:sindrome}
    S(x) = \sum_{\gamma \in M} \frac{e_\gamma}{x - \gamma} \pmod{g(x)}.
\end{equation}

De esta forma, ahora podemos introducir el polinomio cuyas raíces son las ubicaciones de los errores,
\begin{equation}
    \label{def:localizaciones}
    \sigma (x) = \prod_{\gamma \in M} (x - \gamma).
\end{equation}

Sin embargo, para los códigos de Goppa es más conveniente definir una variante de este polinomio de la siguiente forma.
\begin{equation}
    \label{def:eta}
    \eta (x) = \sum_{\gamma \in M} e_\gamma \prod_{\partial \in M \setminus \{ \gamma \} } (x - \partial)
\end{equation}

Observemos que de esta forma $\sigma(x)$ y $\eta(x)$ deben ser primos relativos.

Derivando la expresión de $\sigma(x)$, tenemos que 
\begin{equation}
    \label{def:localizaciones_derivada}
    \sigma'(x) = \sum_{\gamma \in M} \prod_{\partial \in M \setminus \{ \gamma \} } (x - \partial),
\end{equation}

de donde, para cada $\gamma \in M$,
\[
    \eta (\gamma) = e_\gamma \prod_{\partial \in M \setminus \{ \gamma \} } (\gamma - \partial) = e_\gamma \sigma'(\gamma),
\]

por lo que $e_\gamma = \frac{\eta(\gamma)}{\sigma'(\gamma)}$. De esta forma, una vez que hemos calculado los polinomios $\sigma$ y $\eta$, las coordenadas del vector error vienen dadas por 
\[
    e_\gamma = \left\{ \begin{array}{lcc}
    0 &   \text{si}  & \sigma(\gamma) \neq 0 \\
    \\ \frac{\eta(\gamma)}{\sigma'(\gamma)} &  \text{si} & \sigma(\gamma) = 0
    \end{array}
    \right.
\]
donde $\sigma'(x)$ es la derivada de $\sigma(x)$.

Lo esencial para decodificar los códigos de Goppa es determinar los coeficientes de los polinomios $\sigma$ y $\eta$. Para ello, tenemos que relacionar $\sigma$ y $\eta$ al síndrome de la ecuación \eqref{def:sindrome}. Esto se consigue multiplicando las ecuaciones \eqref{def:sindrome} y \eqref{def:localizaciones}, obteniendo
\begin{equation}
    \label{prop:key_equation}
    S(x) \cdot \sigma(x) \equiv \eta(x) \pmod{g(x)}.
\end{equation}

La ecuación \eqref{prop:key_equation} es la \emph{ecuación clave} para decodificar los códigos de Goppa. Dado $g(x)$ y $S(x)$, el problema de decodificar consiste en encontrar polinomios de grado bajo $\sigma(x)$ y $\eta(x)$ que satisfacen \eqref{prop:key_equation}.

Reduciendo cada potencia de $x \pmod{g(x)}$ e igualando coeficientes de $1, x, ..., x^{\gr(g) - 1}$, tenemos que \eqref{prop:key_equation} es un sistema de $\gr(G)$ ecuaciones lineales donde las incógnitas son los coeficientes de $\sigma$ y $\eta$. Por lo tanto, para probar que el decodificador es capaz de corregir todos los patrones hasta $t$ errores, basta con probar que \eqref{prop:key_equation} tiene una única solución con grados de $\sigma$ y de $\eta$ suficientemente pequeños. Esto equivale a que el conjunto de ecuaciones lineales correspondientes sean linealmente independientes.

Supongamos que existen dos pares diferentes de soluciones a \eqref{prop:key_equation}:
\begin{equation}
    \label{prop:key_equation_1}
    S(x) \sigma^{(1)}(x) \equiv \eta^{(1)}(x) \pmod{g(x)},
\end{equation}
\begin{equation}
    \label{prop:key_equation_2}
    S(x) \sigma^{(2)}(x) \equiv \eta^{(2)}(x) \pmod{g(x)},
\end{equation}

donde $\sigma^{(1)}(x)$ y $\eta^{(1)}(x)$ son primos relativos, al igual que $\sigma^{(2)}(x)$ y $\eta^{(2)}(x)$. Además, $\sigma^{(1)}(x)$ y $g(x)$ no pueden tener ningún factor en común, pues en ese caso ese factor podría dividir a $\eta^{(1)}(x)$, contradiciendo que $\sigma^{(1)}(x)$ y $\eta^{(1)}(x)$ son primos relativos. Así, podemos dividir \eqref{prop:key_equation_1} por $\sigma^{(1)}(x)$ y obtenemos
\[
    S(x) \equiv \frac{\eta^{(1)}(x)}{\sigma^{(1)}(x)} \pmod{g(x)}.
\]
De la misma forma para \eqref{prop:key_equation_2},
\[
    S(x) \equiv \frac{\eta^{(2)}(x)}{\sigma^{(2)}(x)} \pmod{g(x)}
\]
de donde,
\begin{equation}
    \label{prop:key_equation_12}
    \sigma^{(1)}(x) \eta^{(2)}(x) \equiv \sigma^{(2)}(x) \eta^{(1)}(x) \pmod{g(x)}.
\end{equation}

Si $\gr(G) = 2t$ y $\gr(\sigma^{(1)}) \leq t$, $\gr(\sigma^{(2)}) \leq t$, $\gr(\eta^{(2)}) < t$ y $\gr(\eta^{(1)}) < t$, entonces se da la siguiente igualdad
\begin{equation}
    \label{prop:key_equation_caso1}
    \sigma^{(1)}(x) \eta^{(2)}(x) = \sigma^{(2)}(x) \eta^{(1)}(x).
\end{equation}

Así, $\sigma^{(1)}$ divide a $\sigma^{(2)} \eta^{(1)}$, y como $\sigma^{(1)}$ y $\eta^{(1)}$ son primos relativos, $\sigma^{(1)}$ tiene que dividir a $\sigma^{(2)}$. Análogamente, $\sigma^{(2)}$ tiene que dividir a $\sigma^{(1)}$. Como ambos son mónicos, se tiene que $\sigma^{(1)} = \sigma^{(2)}$ y así, $\eta^{(1)} = \eta^{(2)}$. Con esto hemos probado que si el grado de $G$ es $2t$, entonces \eqref{prop:key_equation} tiene una única solución cuando $\gr(\eta) < \gr(\sigma) \leq t$, luego el correspondiente sistema de ecuaciones lineales donde las incógnitas son los coeficientes de $\sigma$ y $\eta$ tiene que ser no singular. En el siguiente teorema se concluye este resultado.

\begin{theorem}
    Si $\gr(g(x)) = 2t$, entonces hay un algoritmo de decodificación algebraica de corrección de $t$ errores para el código q-ario de Goppa con el polinomio de Goppa $g(x)$.
\end{theorem}

Estudiemos ahora este resultado en el caso binario, primero observamos que ya que todos los $e_\gamma$ distintos de cero son iguales a 1, entonces \eqref{def:eta} y \eqref{def:localizaciones_derivada} coincidan. De esta forma, la ecuación \eqref{prop:key_equation_12} ahora pasa a ser
\[
    \sigma^{(1)} \left( \sigma^{(2)} \right) ' \equiv \sigma^{(2)} \left( \sigma^{(1)} \right) ' \pmod{g(x)}
\]
Ahora, cuando $\sigma$ sea par escribiremos en su lugar $\hat{\sigma}$, mientras que cuando $\sigma$ sea impar escribiremos en su lugar $x \sigma '$. Así, tenemos que
\begin{align*} 
    \left( \hat{\sigma}^{(1)} + x \sigma^{(1)'} \right) \sigma^{(2)'} &\equiv \left( \hat{\sigma}^{(2)} + x \sigma^{(2)'} \right) \sigma^{(1)'}\\ 
    \hat{\sigma}^{(1)} \sigma^{(2)'} + \hat{\sigma}^{(2)} \sigma^{(1)'} &\equiv 0 \pmod{g(x)}.
\end{align*}

El lado izquierdo es un cuadrado perfecto, pues todos los polinomios de ese lado son pares. Esto implica que
\[
    \hat{\sigma}^{(1)} \sigma^{(2)'} + \hat{\sigma}^{(2)} \sigma^{(1)'} \equiv 0 \pmod{\bar{G}(x)}
\]
donde $\bar{G}(x)$ es múltiplo de $g(x)$ de menor grado ya que $\bar{G}$ es un cuadrado perfecto. Por lo que, si $\gr(\bar{G}) = 2t$, $\gr(\sigma^{(1)}) \leq t$ y $\gr(\sigma^{(2)}) \leq t$, entonces 
\[
    \hat{\sigma}^{(1)} \left( \sigma^{(2)} \right) ' = \hat{\sigma}^{(2)} \sigma^{(1)'}.
\]
Por la primalidad relativa, $\sigma^{(1)} = \sigma^{(2)}$. En el siguiente teorema se concluye este resultado.

\begin{theorem}
    Si $\gr(g(x)) = t$ y si $g(x)$ no tiene factores irreducibles repetidos, entonces hay un algoritmo de decodificación algebraica de corrección de $t$ errores para el código binario de Goppa con el polinomio de Goppa $g(x)$.
\end{theorem}

\subsubsection{Algoritmo de decodificación de Sugiyama}

El algoritmo de Sugiyama, descrito en \cite{Sugiyama_1975}, es una aplicación simple del algoritmo de Euclides para determinar el polinomio localizador de errores de una manera más eficiente.

A continuación vamos a describir alguna de las propiedades el algoritmo de Euclides que están relacionadas con el método para resolver la ecuación clave para decodificar los códigos de Goppa. El algoritmo de Euclides que presentamos en \ref{th:alg-euclides}, en este caso aplicado a los polinomios $r_{-1}(x)$ y $r_0(x)$ donde $\gr(r_{-1}) > \gr(r_0)$, se puede reescribir en forma de matriz como sigue:
\begin{equation}
    \label{alg:euclides-matriz}
    \left( 
        \begin{array}{c}  
            r_{i-2}(x) \\
            r_{i-1}(x) \\
        \end{array} 
    \right)
    = 
    \left( 
        \begin{array}{cc}  
            q_{i}(x) & 1 \\
            1 & 0 \\
        \end{array} 
    \right)
    \left( 
        \begin{array}{c}  
            r_{i-1}(x) \\
            r_{i}(x) \\
        \end{array} 
    \right).
\end{equation}

Definimos los polinomios $U_i(x)$ y $V_i(x)$ como
\begin{equation}
    \label{alg:euclides-u}
    U_i(x) = q_i(x)U_{i-1}(x) + U_{i-2}(x)
\end{equation}
y
\[
    V_i(x) = q_i(x)V_{i-1}(x) + V_{i-2}(x),
\]
donde $U_0(x) = 1$, $U_{-1}(x) = 0$, $V_0(x) = 0$ y $V_{-1}(x) = 1$, tenemos que
\begin{equation}
    \label{alg:euclides-u-v}
    \left( 
        \begin{array}{cc}  
            U_i(x) & U_{i-1} \\
            V_i(x) & V_{i-1} \\
        \end{array} 
    \right)
    = 
    \left( 
        \begin{array}{cc}  
            q_{1}(x) & 1 \\
            1 & 0 \\
        \end{array} 
    \right)
    \cdots
    \left( 
        \begin{array}{cc}  
            q_{i}(x) & 1 \\
            1 & 0 \\
        \end{array} 
    \right).
\end{equation}

De \eqref{alg:euclides-matriz} y \eqref{alg:euclides-u-v}, obtenemos
\[
    \left( 
        \begin{array}{c}  
            r_{-1}(x) \\
            r_{0}(x) \\
        \end{array} 
    \right)
    = 
    \left( 
        \begin{array}{cc}  
            U_i(x) & U_{i-1}(x) \\
            V_i(x) & V_{i-1}(x) \\
        \end{array} 
    \right)
    \left( 
        \begin{array}{c}  
            r_{i-1}(x) \\
            r_{i}(x) \\
        \end{array} 
    \right).
\]
Como el determinante de la matriz de la izquierda de \ref{alg:euclides-u-v} es $(-1)^i$, tenemos que
\[
    \left( 
        \begin{array}{c}  
            r_{-1}(x) \\
            r_{0}(x) \\
        \end{array} 
    \right)
    = 
    (-1)^i
    \left( 
        \begin{array}{cc}  
            V_{i-1}(x) & -U_{i-1}(x) \\
            -V_i(x) & U_{i}(x) \\
        \end{array} 
    \right)
    \left( 
        \begin{array}{c}  
            r_{-1}(x) \\
            r_{0}(x) \\
        \end{array} 
    \right).
\]
Esta ecuación nos permite relacionar el resto $r_i(x)$ de la iteración $i$-ésima con los polinomios $r_{-1}(x)$ y $r_0(x)$ de la siguiente manera
\begin{equation}
    \label{prop:relacion-r_i}
    r_i(x) = (-1)^i \left( -V_i(x) r_{-1}(x) + U_i(x) r_0(x) \right).
\end{equation}

Como consecuencia, observamos que $r_{-1}(x)$ divide al polinomio $\left( -r_{-1}(x) + (-1)^i U_i(x) r_0(x) \right)$. Esto conlleva a la siguiente relación:
\begin{equation}
    \label{prop:euclides-resto}
    r_i(x) \equiv (-1)^i U_i(x) r_0(x) \pmod{r_{-1}(x)}.
\end{equation}

Observamos que esta relación es similar a la ecuación clave para decodificar los códigos de Goppa. Veamos otras propiedades del algoritmo de Euclides que también están relacionadas con dicha ecuación clave.

Aplicando \ref{th:alg-euclides} a los polinomios $r_{-1}(x)$ y $r_0(x)$, sabemos que en cada iteración $i$-ésima se cumple la siguiente relación:
\begin{equation}
    \label{prop:euclides-grados}
    \gr(r_i) < \gr(r_{i-1}).
\end{equation}

Por \eqref{alg:euclides-u}, se tiene que $\gr(U_i) = \sum_{j=1}^i \gr(q_j)$. Como $\gr(r_{i-1}) = \gr(r_{-1}) - \sum_{j=1}^i \gr(q_j)$, aplicando \ref{th:alg-euclides}, es claro que
\begin{equation}
    \label{prop:euclides-grado-u}
    \gr(U_i) = \gr(r_{-1}) - \gr(r_{i-1}).
\end{equation}

A partir de \eqref{alg:euclides-u-v}, obtenemos
\begin{equation}
    \label{prop:euclides-primos-rel}
    U_i(x) V_{i-1}(x) = U_{i-1}(x) V_i(x) = (-1)^i.
\end{equation}

De esta forma, hemos probado que los polinomios $U_i(x)$ y $V_i(x)$ son primos relativos.

A continuación se describe el teorema fundamental para resolver la ecuación clave para decodificar los códigos de Goppa, usando el algoritmo de Euclides.

\begin{theorem}[Algoritmo de Sugiyama]
    \label{th:alg-Sugiyama}
    Fijado un código de Goppa $\Gamma(L, g)$ sobre el cuerpo $\mathbb{F}_q$ y sean $S(x)$ su síndrome, $\sigma(x)$ el polinomio localizador de errores y $\eta(x)$ el polinomio evaluador de errores. Definimos $r_{-1}(x) = g(x)$ y $r_0(x) = S(x)$. Empezamos las divisiones \ref{th:alg-euclides} del algoritmo de Euclides para calcular el máximo común divisor de $r_{-1}(x)$ y $r_0(x)$ desde $i = 0$ hasta $i = k$, de tal forma que se cumpla que $\gr(r_{k-1}) \geq t$ y $\gr(r_k) \leq t - 1$. Entonces la solución viene dada por
    \begin{equation}
        \label{th:sol-Goppa}
        \begin{split}
            \sigma_s(x) &= \delta U_k (x), \\
            \eta_s(x) &= (-1)^k \delta r_k(x)
        \end{split}
    \end{equation}

    donde $\delta$ es una constante no nula que hace que $\sigma_s(x)$ sea mónico.
\end{theorem}

\begin{proof}
    Comenzamos viendo que el número de iteraciones $k$ cuando se satisfacen $\gr(r_{k-1}) \geq t$ y $\gr(r_k) \leq t - 1$ es único. Tenemos que $\gr(r_0) = \gr(S) \leq t$ \cite[Lema 1]{Sugiyama_1975}. Sea $p(x)$ el máximo común divisor de $g(x)$ y $S(x)$, existe la iteración $i$ en la que se cumple que $\gr(p) \leq \gr(r_i) \leq t - 1$. Por \eqref{prop:euclides-grados}, es claro que la secuencia de $\gr(r_i)$ para $i = 0, 1, 2$ decrece monótonamente conforme $i$ se incrementa. Concluimos que el número de iteraciones $k$ es único.

    Por otra parte, como $U_k(x) S(x) \equiv (-1)^k r_k(x) \pmod{g(x)}$ se satisface por \eqref{prop:euclides-resto}, tenemos que $\sigma_s(x)$ y $\eta_s(x)$ dados por \eqref{th:sol-Goppa} satisfacen la relación
    \[
        \sigma_s(x) S(x) \equiv \eta_s(x) \pmod{g(x)}.
    \]
    Deduzcamos ahora el grado de los polinomios $\sigma_s$ y $\eta_s$. Sabemos que $\gr(r_k) \leq t - 1$, luego $\eta$ cumple que $\gr(\eta_s) \leq t - 1$. En cuanto a $\sigma$, por \eqref{prop:euclides-grado-u} tenemos que $\gr(U_k) = \gr(g) - \gr(r_{k-1})$ y como $\gr(r_{k-1}) \geq t$, entonces $\gr(\sigma_s) \leq t$.

    Veamos ahora que $\sigma_s(x)$ y $\eta_s(x)$ son primos relativos. Para ello, supongamos que no lo son. Entonces estos polinomios satisfacen
    \begin{equation}\label{prop:sigma-eta-no-primos-relativos}
        \begin{split}
            \sigma_s(x)     &=   \mu(x) \sigma(x)\\
            \eta_s(x)       &= \mu(x) \eta(x),\\
        \end{split}
    \end{equation}

    donde $\mu(x)$ es un polinomio conveniente \cite[Página 92]{Sugiyama_1975}.
    
    Por \eqref{prop:relacion-r_i} y \eqref{prop:sigma-eta-no-primos-relativos}, tenemos que
    \begin{align*}
        \mu(x) \eta(x)  &=   \mu(x) \sigma(x) S(x) - \delta V_k(x) g(x)\\
        \eta(x)         &=  \sigma(x) S(x) + \phi(x) g(x),
    \end{align*}
    
    donde $\phi(x)$ es un polinomio conveniente. A partir de estas relaciones, es evidente que $\mu(x)$ divide a $V_k(x)$. El polinomio $\mu(x)$ también divide a $U_k(x)$. Pero esto contradice a \eqref{prop:euclides-primos-rel} que implica que $U_k(x)$ y $V_k(x)$ son primos relativos. Por lo que $\sigma_s(x)$ y $\eta_s(x)$ son primos relativos.

    Finalmente, el factor $\delta$ hace que $\sigma_s(x)$ sea mónico. Por lo tanto, $\sigma_s(x)$ y $\eta_s(x)$ dados por \eqref{th:sol-Goppa} son la solución del problema.
\end{proof}

\begin{corollary}
    Sea $e$ el número de errores que realmente se produjeron y sea $k$ el número de iteraciones del algoritmo descrito en el teorema, entonces $k \leq e$.
\end{corollary}

\begin{proof}
    Como el número de errores que se produjeron es $e$, entonces $\gr(\sigma) = e$. Por \eqref{alg:euclides-u} y \eqref{th:sol-Goppa},
    \[
        \gr(\sigma_s) = \sum_{i=1}^k \gr(q_i).
    \]
    Como $\gr(q_i) \geq 1$ para todo $i$, concluimos que $k \leq e$.
\end{proof}

En resumen, el algoritmo de Sugiyama consiste en:

\begin{itemize}
    \item[I.] Calcular el síndrome $S(x)$.
    \item[II.] Sean $r_{-1}(x) = g(x)$, $r_0(x) = S(x)$, $U_{-1}(x) = 0$, $U_0(x) = 1$, $V_0(x)$ y $V_{-1}(x) = 1$.
    \item[III.] Buscar $U_i(x)$, $V_i(x)$, $q_i(x)$ y $r_i(x)$ aplicando el algoritmo de Euclides para encontrar el máximo común divisor de $r_{-1}(x)$ y $r_0(x)$ para $i = 1,..., k$, hasta que $k$ cumpla que $\gr(r_{k-1}(x)) \geq t$ y $\gr(r_k(x)) < t$:
        \begin{equation*}
            \begin{split}
                r_{i-2}(x) &= r_{i-1}(x) q_i(x) + r_i(x), \qquad \gr(r_i(x)) < \gr(r_{i-1})(x)\\
                U_i(x) &= q_i(x) U_{i-1}(x) + U_{i-2}(x)\\
                V_i(x) &= q_i(x) V_{i-1}(x) + V_{i-2}(x)
            \end{split}
        \end{equation*}
    \item[IV.] La solución viene dada por:
        \begin{equation*}
            \begin{split}
                \sigma_s(x) &= \delta U_k (x), \\
                \eta_s(x) &= (-1)^k \delta r_k(x)
            \end{split}
        \end{equation*}  
    \item[V.] Calcular las raíces de $\sigma(x)_s$.
    \item[VI.] Para cada raíz $P$, buscar la posición en la que se encuentra en el conjunto de definición y el error en esa posición tendrá el valor:
        $$\frac{\eta(P)}{\sigma'(P)}.$$
    \item[VII.] Restar a la palabra codificada el error para obtener el mensaje original codificado.
\end{itemize}

\begin{exampleth}
    % TODO
\end{exampleth}

% TODO: más ejemplos
\textcolor{red}{Completar y añadir más ejemplos}


\begin{comment}
El algoritmo de Sugiyama es el siguiente:

\begin{itemize}
    \item[I.] Sean $f(x) = x^{2t}$, $s(x) = S(x)$, $r_{-1}(x) = f(x)$, $r_0(x) = s(x)$, $b_{-1}(x) = 0$ y $b_0(x) = 1$.
    \item[II.] Iterar incrementando en una unidad desde $i = 1$ hasta $I$, de forma que se cumpla que $\gr(r_{I-1}(x)) \geq t$ y $\gr(r_I(x)) < t$. En cada iteración habrá que determinar $h_i(x)$, $r_i(x)$ y $b_i(x)$:
        $$r_{i-2}(x) = r_{i-1}(x) h_i(x) + r_i (x), \qquad \text{ donde } \gr(r_i(x)) < \gr(r_{i-1}(x)),$$
        $$b_i(x) = b_{i-2}(x) - h_i(x) b_{i-1}(x).$$
    \item[III.] $\sigma(x)$ es un escalar no nulo múltiplo de $b_I(x)$. 
\end{itemize}

Para verificar que este algoritmo funciona, el siguiente lema nos será de utilidad.

\begin{lemma}
    \label{lm:dem_sugiyama}
    Con la notación del algoritmo de Sugiyama, sea $a_{-1}(x) = 1$, $a_0(x) = 0$ y $a_i(x) = a_{i-2}(x) - h_i(x) a_{i-1}(x)$ para $i \geq 1$. Las siguientes afirmaciones son ciertas.

    \begin{itemize}
        \item $a_i(x) f(x) + b_i(x) s(x) = r_i (x)$ para $i \geq -1$.
        \item $b_i(x) r_{i-1}(x) - b_{i-1} r_i(x) = (-1)^i f(x)$ para $i \geq 0$.
        \item $a_i(x) b_{i-1}(x) - a_{i-1} b_i (x) = (-1)^{i+1}$ para $i \geq 0$.
        \item $\gr(b_i(x)) + \gr(r_{i-1}(x)) = \gr(f(x))$ para $i \geq 0$.
    \end{itemize}
\end{lemma}

\begin{proof}
    Ver Lema 5.4.11 en la página 191, \cite{Huffman_Pless_2010}.
\end{proof}

Comprobemos ahora que el algoritmo de Sugiyama funciona. Por el Lema \ref{lm:dem_sugiyama} (i) tenemos que

\begin{equation}
    \label{dem:sugiyama_relacion}
    a_I(x) x^{2t} + b_I(x) S(x) = r_I(x).
\end{equation}

Además, por la ecuación clave sabemos que

\begin{equation}
    \label{dem:sugiyama_key_equation}
    a(x) x^{2t} + \sigma(x) S(x) = \omega(x)
\end{equation}

para algún polinomio $a(x)$. Multiplicando \ref{dem:sugiyama_relacion} por $\sigma(x)$ y \ref{dem:sugiyama_key_equation} por $b_I(x)$ obtenemos 

\begin{equation}
    \label{dem:sugiyama_relacion_sigma}
    a_I(x) \sigma(x) x^{2t}  + b_I(x) \sigma(x) S(x) = r_I(x) \sigma(x) \quad \text{ y }
\end{equation}
\begin{equation}
    \label{dem:sugiyama_key_equation_b_I}
    a(x) b_I(x) x^{2t} + \sigma(x) b_I(x) S(x) = \omega(x) b_I(x).
\end{equation}

Aplicando módulo $x^{2t}$ a ambas ecuaciones tenemos que 

\begin{align*}
    b_I(x) \sigma(x) S(x) &\equiv r_I(x) \sigma(x) \pmod{x^{2t}} \quad \text{ y }\\
    \sigma(x) b_I(x) S(x) &\equiv \omega(x) b_I(x) \pmod{x^{2t}}.
\end{align*}

Por lo tanto,

\begin{equation}
    \label{dem:sugiyama_modulo}
    r_I(x) \sigma(x) \equiv \omega(x) b_I(x) \pmod{x^{2t}}.
\end{equation}

Como $\gr(\sigma(x)) \leq t$, por la elección de $I$,

$$\gr(r_I(x) \sigma(x)) = \gr(r_I(x)) + \gr(\sigma(x)) < t + t = 2t.$$ 

Por el Lema \ref{lm:dem_sugiyama} (iv), la elección de $I$ y como $\gr(\omega(x)) < t$, entonces se cumple que 

$$\gr(\omega(x) b_I(x)) = \gr(\omega(x)) + \gr(b_I(x)) < t + \gr(b_I(x)) = t + (\gr(x^{2t}) - \gr(r_{I-1}(x))) \leq 3t - t = 2t.$$

Por \ref{dem:sugiyama_modulo} tenemos que $r_I(x) \sigma(x) = \omega(x) b_I(x)$. Esto, junto con \ref{dem:sugiyama_relacion_sigma} y \ref{dem:sugiyama_key_equation_b_I}, implica que 

\begin{equation}
    \label{dem:sugiyama_implicacion_ecuaciones}
    r_I(x) \sigma(x) \equiv \omega(x) b_I(x) \pmod{x^{2t}}.
\end{equation}

Sin embargo, el Lema \ref{lm:dem_sugiyama} (iii) afirma que $a_I(x)$ y $b_I(x)$ son primos relativos, y por \ref{dem:sugiyama_implicacion_ecuaciones} se cumple que $a(x) = \lambda(x) a_I(x)$. Sustituyendo esta relación en \ref{dem:sugiyama_implicacion_ecuaciones},

\begin{equation}
    \label{dem:sugiyama_sustitucion}
    \sigma(x) = \lambda(x) b_I(x).
\end{equation}

Ahora, sustituyendo estas dos últimas relaciones en \ref{dem:sugiyama_key_equation} obtenemos

$$\lambda(x) a_I(x) x^{2t} + \lambda(x) b_I(x) S(x) = \omega(x).$$ 

La ecuación \ref{dem:sugiyama_relacion} implica que

\begin{equation}
    \label{dem:sugiyama_relacion_omega}
    \omega(x) = \lambda(x) a_I(x) x^{2t} + \lambda(x) b_I(x) S(x) = \lambda(x) \cdot \left( a_I(x) x^{2t} + b_I(x) S(x) \right) = \lambda(x) r_I(x).
\end{equation}

Teniendo en cuenta que los polinomios $\sigma(x)$ y $\omega(x)$ son primos relativos y por \ref{dem:sugiyama_sustitucion} y \ref{dem:sugiyama_relacion_omega}, $\lambda(x)$ tiene que ser una constante no nula, verificando el paso III del algoritmo de Sugiyama.

Como solo estamos interesados en las raíces de $\sigma(x)$, es suficiente con determinar las raíces de $b_I(x)$ obtenidos en el paso II; lo que nos dará los $X_j$. 

Más adelante veremos que este algoritmo sigue funcionando con otras elecciones de $f(x)$ y $s(x)$, con las modificaciones apropiadas de las condiciones sobre las que el algoritmo termina en el paso II. Una de estas modificaciones nos será útil para decodificar los códigos Goppa que veremos a continuación.

\end{comment}