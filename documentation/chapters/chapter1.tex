% 0. Preliminares. 
% Esto es voluntario. 
% Consiste en poner lo que ya sabías, porque lo has dado en el grado, y que te haga falta 
% (por ejemplo, algebra lineal, cuerpos finitos, operaciones con polinómios, algoritmos genéticos, etc).

% TODO
\chapter{Preliminares}

En este capítulo se desarrollarán las herramientas necesarias para poder afrontar el criptosistema de McEliece 
que precisa este trabajo. Se abordarán conceptos relacionados con el álgebra lineal, anillos, cuerpos finitos, polinomios, algoritmos genéticos, etc.

\section{Anillos}

En esta sección introduciremos el concepto de anillo para poder definir el concepto de cuerpo.

\begin{definition}
    Un \emph{anillo} $(A, +, \cdot )$ es un conjunto $A$ junto con dos operaciones binarias $A \times A \rightarrow A$ denotadas
    por la suma (denotada por $+$) y producto (denotado por $\cdot$) que verifican los siguientes axiomas:

    \begin{itemize}
        \item Propiedad asociativa de la suma: 
        $$  a + (b + c) = (a + b) + c \qquad \forall a,b,c \in A$$
        \item Existencia del elemento neutro para la suma:
        $$ 0 + a = a = a + 0 \qquad \forall a \in A$$
        \item Existencia del elemento inverso para la suma:
        $$ \forall a \in A \; \; \exists -a \in A \quad a + (-a) = 0 = (-a) + a $$
        \item Propiedad conmutativa de la suma:
        $$ a + b = b + a \qquad \forall a,b \in A $$
        \item Propiedad asociativa del producto:
        $$ a(bc) = (ab)c \qquad \forall a,b,c \in A $$
        \item Propiedad distributiva del producto:
        $$ a(b + c) = ab + ac, \quad (b + c)a = ba + ca \qquad \forall a,b,c \in A $$
        \item Existencia del elemento neutro para el producto:
        $$ 1a = a = a1 \qquad \forall a \in A $$
        
    \end{itemize}

    Un anillo de llama \emph{conmutativo o abeliano} si se verifica la propiedad conmutativa del producto 
    $$ ab = ba \qquad \forall a,b \in A $$
\end{definition}

%TODO
Añadir algo de ideales.

% No sé si hace falta
%\section{Dominios de integridad}
%Sea $A$ un anillo conmutativo.

%\begin{definition}
%    Un elemento $a \in A$ se llama \emph{divisor de cero} si existe un $b \in A$ con $b \neq 0$ tal que $ab = 0$.
    
%    Un \emph{dominio de integridad} es un anillo conmutativo $A$ no trivial sin divisores de cero no nulos, es decir, 
%    si $1 \neq 0$ y si $ab = 0$, entonces $a = 0$ o $b = 0$.
%\end{definition}

%TODO
%Añadir propiedades de los dominios de integridad.

\section{Cuerpos finitos}

Con estos conceptos previos podemos ya definir el de cuerpo.

\begin{definition}
    Un \emph{cuerpo} $(A, +, \cdot )$ es un anillo conmutativo no trivial en el que todo elemento no nulo tiene un inverso multiplicativo.
    Se dice que un cuerpo es \emph{finito} si tiene un número finito de elementos.
\end{definition}

En los códigos lineales son comunes los siguientes cuerpos: \emph{cuerpo binario} con dos elementos, 
\emph{cuerpo ternario} con tres elementos y \emph{cuerpo cuaternario} con cuatro elementos.

Diremos que la \emph{característica} de un cuerpo es el número de elementos que tiene.

Todos los cuerpos finitos tienen un número de elementos $q = p^n$, para algún número primo $p$ y algún entero positivo $n$.
Denotaremos por $\mathbb{F}_q$ a los cuerpos finitos con característica $q$.


\section{Polinomios}

En esta sección vamos a introducir el concepto de polinomio junto con sus operaciones.

\begin{definition}
    Sea $A$ un anillo conmutativo. El \emph{conjunto de polinomios} en la indeterminada $X$ con coeficientes en $A$ es el conjunto de 
    todas las sumas formales finitas
    $$f = a_n X^n + a_{n-1} X^{n-1} + \cdots + a_1 X + a_0$$
    Este conjunto de representa por $A[X]$.
\end{definition}

En el conjunto de polinomios definimos una suma y un producto. 

Sean $f = a_n X^n + a_{n-1} X^{n-1} + \cdots + a_1 X + a_0$ y $g = b_m X^m + b_{m-1} X^{m-1} + \cdots + b_1 X + b_0$
dos polinomios. Supongamos que $m \leq n$, tomando $b_i = 0$ para todo $n \geq i > m$, definimos las operaciones de suma y producto de polinomios
$$f + g = (a_n + b_n)X^n + \cdots + (a_1 + b_1)X + (a_0 + b_0)$$
$$f \cdot g = a_n b_m X^{n+m} + (a_n b_{m+1} + a_{n-1} b_m) X^{n+m-1} + \cdots + (a_1 b_0 + a_0 b_1)X + a_0 b_0$$

De esta forma, diremos que el conjunto $A[X]$ con las operaciones anteriores es un \emph{anillo de polinomios en X con coeficientes en A}.

\begin{definition}
    Para un polinomio $f = a_n + a_{n-1} X^{n-1} + \cdots + a_1 X + a_0 \neq 0$ el mayor índice $n$ tal que $a_n \neq 0$ se llama \emph{grado de f}
    y se representa por $gr(f)$. Si $f = 0$ definimos $gr(f) = - \infty$.

    Llamaremos \emph{término (de grado i)} a cada uno de los sumandos $a_i X^i$ del polinomio $f$. El \emph{término líder}
    es el término no nulo de mayor grado. El coeficiente $a_n \neq 0$ del término líder se llama \emph{coeficiente líder} 
    y el término de grado cero $a_0$ se llama \emph{término constante}.
\end{definition}

A continuación tenemos algunas propiedades de los polinomios.

\begin{proposition}
    Sea $A$ un anillo conmutativo y sean $f,g \in A[X]$ dos polinomios, tenemos que 
    $$gr(f + g) \leq \max{(gr(f), gr(g))},$$
    $$gr(f \cdot g) \leq gr(f) + gr(g)$$
    Si $gr(f) \neq gr(g)$, se verifica 
    $$gr(f + g) = \max{(gr(f), gr(g))}$$
    Si $A$ es un dominio de integridad, entonces
    $$gr(f \cdot g) = gr(f) + gr(g)$$
\end{proposition}

% TODO: añadir teorema del resto?
Añadir más propiedades


\section{Algoritmos genéticos}

%TODO
Añadir introducción a la sección: qué es y para qué sirve (tratar problemas que no son P, o que no se sabe que sean P). Didáctica!!!!

Los \emph{algoritmos genéticos} son algoritmos de optimización, búsqueda y aprendizaje inspirados en los procesos de evolución 
natural y evolución genética.

% TODO: esquema del ciclo de la evolución
En general, los algoritmos genéticos siguen el siguiente procedimiento (explicarlo mejor).

\begin{lstlisting}
    t = 0
    inicializar la poblacion P(t)
    evaluar la poblacion P(t)
    Mientras (no se cumpla la condicion de parada) hacer 
        t = t + 1
        seleccionar P' desde P(t-1)
        recombinar P'
        mutar P'
        reemplazar P(t) a partir de P(t-1) y P'
        evaluar P(t)
\end{lstlisting}

Existen dos modelos de algoritmos genéticos, el modelo generacional y el modelo estacionario.

En el modelo generacional, durante cada iteración se crea una población completa con nuevos individuos.
Así, la nueva población reemplaza directamente a la antigua.

En el modelo estacionario, durante cada iteración se escogen dos padres de la población y se les aplican los operadores genéticos.
De este modo, los descendientes reemplazan a los cromosomas de la población anterior.
Este modelo es elitista y produce una convergencia rápida cuando se reemplazan los peores cromosomas de la población.

%TODO
Añadir más cosas


\section{Clases de complejidad}

Explicar que es un problema NP-completo y demás.