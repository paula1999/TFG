% 0. Preliminares. 
% Esto es voluntario. 
% Consiste en poner lo que ya sabías, porque lo has dado en el grado, y que te haga falta 
% (por ejemplo, algebra lineal, cuerpos finitos, operaciones con polinomios, algoritmos genéticos, etc).

% TODO

\chapter{Preliminares}

En este capítulo se desarrollarán las herramientas necesarias para poder afrontar el criptosistema de McEliece que precisa este trabajo. Se abordarán conceptos relacionados con el álgebra lineal, anillos, cuerpos finitos, polinomios, algoritmos genéticos, etc.

\section{Anillos}

En esta sección introduciremos el concepto de anillo para poder definir el concepto de cuerpo, así como las principales propiedades de esta estructura algebraica.

\begin{definition}
    Un \emph{anillo} $(A, +, \cdot )$ es un conjunto $A$ junto con dos operaciones binarias $A \times A \rightarrow A$ denotadas por la suma (denotada por $+$) y producto (denotado por $\cdot$) que verifican los siguientes axiomas:

    \begin{itemize}
        \item Propiedad asociativa de la suma: 
        $$  a + (b + c) = (a + b) + c \qquad \forall a,b,c \in A$$
        \item Existencia del elemento neutro para la suma:
        $$ 0 + a = a = a + 0 \qquad \forall a \in A$$
        \item Existencia del elemento inverso para la suma:
        $$ \forall a \in A \; \; \exists -a \in A \quad a + (-a) = 0 = (-a) + a $$
        \item Propiedad conmutativa de la suma:
        $$ a + b = b + a \qquad \forall a,b \in A $$
        \item Propiedad asociativa del producto:
        $$ a \cdot (b \cdot c) = (a \cdot b) \cdot c \qquad \forall a,b,c \in A $$
        \item Propiedad distributiva del producto:
        $$ a \cdot (b + c) = a \cdot b + a \cdot c, \quad (b + c) \cdot a = b \cdot a + c \cdot a \qquad \forall a,b,c \in A $$
        
    \end{itemize}

    Un anillo de llama \emph{conmutativo o abeliano} si se verifica la propiedad conmutativa del producto 
    $$ ab = ba \qquad \forall a,b \in A $$
\end{definition}

Es fácil comprobar que en todo anillo se verifica $0 \cdot x = x \cdot 0 = 0$ (obsérvese que $0 \cdot = (0 + 0) \cdot x = 0 \cdot x + 0 \cdot x$ y simplifíquese por el simétrico de $0 \cdot x$).

Además del anillo conmutativo, existen otros casos particulares de anillos. Diremos que un anillo es \emph{unitario} si es un anillo cuyo producto tiene elemento neutro, esto es, $\exists 1 \in A \; : \; x \cdot 1 = 1 \cdot x = x$, para todo $x \in A$.

Diremos que un elemento $a$ del anillo $A$ es \emph{invertible} si existe un elemento $a'$ en el anillo $A$ tal que $a \cdot a' = a' \cdot a = 1$. Este elemento $a'$ es único, lo llamaremos \emph{elemento inverso} y lo denotaremos por $a^{-1}$.

Cuando se da la igualdad $1 = 0$, diremos que el anillo es \emph{trivial} y tendrá un solo elemento.

\begin{definition}
    Sea $A$ un anillo, $1 \in A$ el elemento neutro del producto y $n \geq 1$ un número natural, definimos la cardinalidad de $A$ como:

    \[
        Car(A) = \left\{ \begin{array}{lc}
        0 &   \textit{si } n \cdot 1 \neq 0 \textit{ para cualquier } n \geq 1 \\
        \\ n & \textit{si n es el menor número natural no nulo para el que } n \cdot 1 = 0
        \end{array}
        \right.
    \]
\end{definition}

%TODO
% - Añadir homomorfismo de anillos? (apuntes de álgebra I)
% - Añadir subestructuras de anillos e ideales? (apuntes de álgebra I)
% - Añadir propiedades de los dominios de integridad? (apuntes de álgebra I)

\section{Cuerpos finitos}

Para presentar el concepto de cuerpo finito, necesitaremos definir previamente el concepto de cuerpo junto con algunas de sus propiedades más relevantes.

\begin{definition}
    Un \emph{cuerpo} $(K, +, \cdot)$ es un anillo conmutativo no trivial en el que todo elemento no nulo tiene un inverso multiplicativo. Se dice que un cuerpo es \emph{finito} si tiene un número finito de elementos.
\end{definition}

Sea $(K, +, \cdot)$ un cuerpo y $E \subset K$, diremos que $E$ es un \emph{subcuerpo} de $K$ o $K$ es una \emph{extensión} de $E$ si se cumple que $(E, +, \cdot)$ es un cuerpo cuando las operaciones $+$ y $\cdot$ se restringen a $E$.

Diremos que la \emph{característica} de un cuerpo es el número de elementos que tiene.

Todos los cuerpos finitos tienen un número de elementos $q = p^n$, para algún número primo $p$ y algún entero positivo $n$. Denotaremos por $\mathbb{F}_q$ a los cuerpos finitos con característica $q$, aunque otra común notación es $GF(q)$.

Observemos que si $p$ un número primo y $q$ es un número entero tal que $q = p^n$, entonces $\mathbb{F}_q$ es un espacio vectorial sobre $\mathbb{F}_p$ de dimensión $n$. Además, hay $q$ vectores en el espacio vectorial de dimensión $n$ sobre $\mathbb{F}_p$.

Notemos que todos los cuerpos finitos de orden $q$ son isomorfos, aunque cada cuerpo puede tener diferentes representaciones.

\begin{proposition}
    Sea $\mathbb{F}_q$ un cuerpo finito con $q = p^n$ elementos, entonces 

    $$p \cdot \alpha = 0, \qquad \forall \alpha \in \mathbb{F}_q.$$
\end{proposition}

\begin{proposition}
    Sea $\mathbb{F}_q$ un cuerpo finito con característica $p$, se cumple que

    $$( \alpha + \beta )^p = \alpha^p + \beta^p, \qquad \forall \alpha, \beta \in \mathbb{F}_q.$$
\end{proposition}

% TODO: homomorfismo de cuerpos


\subsection{Anillo de polinomios sobre cuerpos finitos}

En esta sección vamos a introducir el concepto de polinomio junto con sus operaciones.

\begin{definition}
    Sea $A$ un anillo conmutativo. El \emph{conjunto de polinomios} en la variable $x$ con coeficientes en $A$ está compuesto por el siguiente conjunto
    $$\{ a_n x^n  + a_{n-1} x^{n-1} + \cdots + a_1 x + a_0 \; : \; a_0, ..., a_n \in A \}.$$
    Este conjunto se representa por $A[X]$.
\end{definition}

En el conjunto de polinomios definimos una suma y un producto. 

Sean $f = a_n x^n + a_{n-1} x^{n-1} + \cdots + a_1 x + a_0$ y $g = b_m x^m + b_{m-1} x^{m-1} + \cdots + b_1 x + b_0$ dos polinomios. Supongamos que $m \leq n$, tomando $b_i = 0$ para todo $n \geq i > m$, definimos las operaciones de suma y producto de polinomios:

$$f + g = (a_n + b_n)x^n + (a_{n-1} + b_{n-1})x^{n-1} + \cdots + (a_1 + b_1)x + (a_0 + b_0).$$
$$f \cdot g = a_n b_m x^{n+m} + (a_n b_{m+1} + a_{n-1} b_m) x^{n+m-1} + \cdots + (a_1 b_0 + a_0 b_1)x + a_0 b_0.$$

De esta forma, diremos que el conjunto $A[X]$ con las operaciones anteriores es un \emph{anillo de polinomios en X con coeficientes en A}.

\begin{definition}
    Para un polinomio $f = a_n + a_{n-1} x^{n-1} + \cdots + a_1 x + a_0 \neq 0$ el mayor índice $n$ tal que $a_n \neq 0$ se llama \emph{grado de f} y se representa por $\gr(f)$. Si $f = 0$ definimos $\gr(f) = - \infty$.

    Llamaremos \emph{término (de grado i)} a cada uno de los sumandos $a_i X^i$ del polinomio $f$. El \emph{término líder} es el término no nulo de mayor grado. El coeficiente $a_n \neq 0$ del término líder se llama \emph{coeficiente líder} y el término de grado cero $a_0$ se llama \emph{término constante}. Si el coeficiente líder es $1$, diremos que el polinomio es \emph{mónico}.
\end{definition}

A continuación tenemos algunas propiedades de los polinomios.

\begin{proposition}
    Sea $A$ un anillo conmutativo y sean $f,g \in A[X]$ dos polinomios, tenemos que 
    $$\gr(f + g) \leq \max{(\gr(f), \; \gr(g))},$$
    $$\gr(f \cdot g) \leq \gr(f) + \gr(g)$$
    Si $\gr(f) \neq \gr(g)$, se verifica 
    $$\gr(f + g) = \max{(\gr(f), \; \gr(g))}$$
\end{proposition}

Podemos trasladar estos resultados al caso de los cuerpos finitos. Sea $\mathbb{F}_q$ un cuerpo finito, un polinomio $f(x)$ estará definido sobre dicho cuerpo si es de la forma $f(x) = \sum_{i = 0}^{n} a_i \cdot x^i$, donde $a_i \in \mathbb{F}_q$ para todo $i = 0, ..., n$. Análogamente, diremos que $f(x) \in \mathbb{F}_q$.

Sean $f(x)$ y $g(x)$ polinomios en $\mathbb{F}_q[x]$, diremos que $f(x)$ divide a $g(x)$ si existe un polinomio $h(x) \in \mathbb{F}_q[x]$ tal que $g(x) = f(x) h(x)$ y lo denotaremos por $f(x) \vert g(x)$. El polinomio $f(x)$ se llama \emph{divisor} o \emph{factor} de $g(x)$.

El \emph{mayor común divisor} de $f(x)$ y $g(x)$ es el polinomio mónico en $\mathbb{F}_q[x]$ con mayor grado que divide a $f(x)$ y a $g(x)$. Este polinomio es único y se denota por $\mcd(f(x), \; g(x))$. Diremos que los polinomios $f(x)$ y $g(x)$ son \emph{primos relativos} si $\mcd(f(x), \; g(x)) = 1$.

El siguiente resultado es de gran utilidad, pues sirve para calcular los divisores de un polinomio e incluso para calcular el máximo común divisor. Nos dará las bases para definir posteriormente el Algoritmo de Euclides.

\begin{theorem}
    \label{th:div_alg}
    Sean $f(x)$ y $g(x)$ polinomios en $\mathbb{F}_q[x]$ con $g(x)$ no nulo.
    \begin{itemize}
        \item Existen dos polinomios únicos $c(x), r(x) \in \mathbb{F}_q[x]$ tales que
        $$f(x) = g(x) c(x) + r(x), \qquad \textit{donde } \gr(r(x)) < \gr(g(x)) \textit{ o } r(x) = 0.$$
        \item Si $f(x) = g(x) c(x) + r(x)$, entonces $\mcd(f(x), \; g(x)) = \mcd(g(x), \; r(x))$.
    \end{itemize}
\end{theorem}

Los polinomios $c(x)$ y $r(x)$ se llaman \emph{cociente} y \emph{resto}, respectivamente.

Usando este resultado de forma recursiva, obtendremos el máximo común divisor de los polinomios $f(x)$ y $g(x)$. Este procedimiento se conoce como \emph{Algoritmo de Euclides}. El siguiente resultado describe este algoritmo.

\begin{theorem}[Algoritmo de Euclides]
    Sean $f(x)$ y $g(x)$ polinomios definidos en $\mathbb{F}_q[x]$ con $g(x)$ no nulo.
    \begin{enumerate}
        \item Realizar los siguientes pasos hasta que $r_n(x) = 0$ para algún $n$:
        $$f(x) = g(x) c_1(x) + r_1(x), \qquad \textit{donde } \gr(r_1(x)) < \gr(x),$$
        $$g(x) = r_1(x) c_2(x) + r_2(x), \qquad \textit{donde } \gr(r_2(x)) < \gr(r_1),$$
        $$r_1(x) = r_2(x) c_3(x) + r_3(x), \qquad \textit{donde } \gr(r_3(x)) < \gr(r_2),$$
        $$\vdots$$
        $$r_{n-3}(x) = r_{n-2}(x) c_{n-1}(x) + r_{n-1}(x), \qquad \textit{donde } \gr(r_{n-1}(x)) < \gr(r_{n-2}),$$
        $$r_{n-2}(x) = r_{n-1}(x) c_{n}(x) + r_{n}(x), \qquad \textit{donde } r_n(x) = 0.$$
        Entonces $\mcd(f(x), \; g(x)) = cr_{n-1}(x)$, donde $c \in \mathbb{F}_q$ es una constante para que $c r_{n-1}(x)$ sea mónico.
        \item Existen polinomios $a(x), b(x) \in \mathbb{F}_q[x]$ tales que 
        $$a(x) f(x) + b(x) g(x) = \mcd(f(x), \; g(x)).$$
    \end{enumerate}
\end{theorem}

En cada paso el grado del resto se decrementa al menos en $1$, por lo que podemos asegurar que la secuencia de pasos anterior terminará en algún momento.

A continuación se muestran algunos resultados relevantes.

\begin{proposition}
    Sean $f(x)$ y $g(x)$ polinomios en $\mathbb{F}_q[x]$.
    \begin{itemize}
        \item Si $k(x)$ es un divisor de $f(x)$ y $g(x)$, entonces $k(x)$ es un divisor de $a(x) f(x) + b(x) g(x)$ para algunos $a(x), b(x) \in \mathbb{F}_q[x]$.
        \item Si $k(x)$ es un divisor de $f(x)$ y $g(x)$, entonces $k(x)$ es un divisor de $\mcd(f(x), \; g(x))$.
    \end{itemize}
\end{proposition}

\begin{proposition}
    Sea $f(x)$ un polinomio en $\mathbb{F}_q[x]$ de grado $n$.
    \begin{itemize}
        \item Si $\alpha \in \mathbb{F}_q$ es una raíz de $f(x)$, entonces $x - \alpha$ es un factor de $f(x)$.
        \item El polinomio $f(x)$ tiene como mucho $n$ raíces en cualquier cuerpo que contenga a $\mathbb{F}_q$.
    \end{itemize}
\end{proposition}

\begin{theorem}
    Los elementos de $\mathbb{F}_q$ son las raíces de $x^q - x$.
\end{theorem}

\subsubsection{Construcción de cuerpos finitos}

Para realizar la construcción de cuerpos finitos, previamente necesitaremos conocer el siguiente concepto y algunos resultados relacionados.

\begin{definition}
    Sea $f(x) \in \mathbb{F}_q[x]$ un polinomio no constante, decimos que es \emph{irreducible} sobre $\mathbb{F}_q$ si no se factoriza como producto de dos polinomios en $\mathbb{F}_q[x]$ de menor grado.
\end{definition}

\begin{theorem}
    Sea $f(x)$ un polinomio no constante. Entonces

    $$f(x) = p_1(x)^{a_1} \cdots p_k(x)^{a_k},$$

    donde cada $p_i(x)$ es irreducible y único salvo orden, y los elementos $a_i$ son únicos.
\end{theorem}

Como consecuencia de este resultado, tenemos que $\mathbb{F}_q[x]$ es un \emph{dominio de factorización única}.

El siguiente resultado nos muestra cómo construir un cuerpo finito de característica $p$ a partir del cociente de anillos de polinomios por polinomios irreducibles.

\begin{proposition}
    Sea $p$ un número primo y sea el polinomio $f(x) \in \mathbb{F}_p[x]$ irreducible en $\mathbb{F}_p$ y de grado $m$. Tenemos que el anillo cociente $\mathbb{F}_q[x]/\left(f(x)\right)$ es un cuerpo finito con $q = p^m$ elementos, es decir, con característica $p$.
\end{proposition}

Escribiremos los elementos del anillo cociente, que son las clases laterales $g(x) + (f(x))$ como vectores en $\mathbb{F}_p^m$ con la siguiente correspondencia:

\begin{equation}
    \label{pr:correspondencia}
    g_{m-1} x^{m-1} + g_{m-2} x^{m-2} + \cdots + g_{1} x + g_0 + (f(x)) \leftrightarrow (g_{m-1}, g_{m-2}, ..., g_1, g_0).
\end{equation}

Esta notación facilita la operación de sumar dos elementos. Sin embargo, la multiplicación es algo más complicada. Supongamos que queremos multiplicar $g_1(x) + (f(x))$ por $g_2(x) + (f(x))$. Para ello, usaremos el resultado \ref{th:div_alg} para obtener

\begin{equation}
    \label{pr:multiplicacion_clase_lateral}
    g_1(x) g_2(x) = f(x) h(x) + r(x),
\end{equation}

donde $\gr(r(x)) \leq m - 1$ o $r(x) = 0$. Entonces 

$$(g_1(x) + (f(x))) (g_2(x) + (f(x))) = r(x) + (f(x)).$$

Podemos simplificar la notación si reemplazamos $x$ por $\alpha$ donde $f(\alpha) = 0$. Por \ref{pr:multiplicacion_clase_lateral}, se cumple que $g_1(\alpha) g_2(\alpha) = r(\alpha)$ y la correspondencia \ref{pr:correspondencia} queda como sigue

$$g_{m-1} g_{m-2} \cdots g_1 g_0 \leftrightarrow  g_{m-1} \alpha^{m-1} g_{m-2} \alpha^{m-2}\cdots g_1 \alpha g_0.$$

De esta forma, multiplicamos los polinomios en $\alpha$ de forma usual y aplicamos al resultado que $f(\alpha) = 0$ para reducir las potencias de $\alpha$ mayores que $m-1$ a polinomios en $\alpha$ de grado menor que $m$.

\section{Algoritmos genéticos}

%TODO
Añadir introducción a la sección: qué es y para qué sirve (tratar problemas que no son P, o que no se sabe que sean P). Didáctica!!!!

Los \emph{algoritmos genéticos} son algoritmos de optimización, búsqueda y aprendizaje inspirados en los procesos de evolución 
natural y evolución genética.

% TODO: esquema del ciclo de la evolución
En general, los algoritmos genéticos siguen el siguiente procedimiento (explicarlo mejor).

\begin{lstlisting}
    t = 0
    inicializar la poblacion P(t)
    evaluar la poblacion P(t)
    Mientras (no se cumpla la condicion de parada) hacer 
        t = t + 1
        seleccionar P' desde P(t-1)
        recombinar P'
        mutar P'
        reemplazar P(t) a partir de P(t-1) y P'
        evaluar P(t)
\end{lstlisting}

Existen dos modelos de algoritmos genéticos, el modelo generacional y el modelo estacionario.

En el modelo generacional, durante cada iteración se crea una población completa con nuevos individuos.
Así, la nueva población reemplaza directamente a la antigua.

En el modelo estacionario, durante cada iteración se escogen dos padres de la población y se les aplican los operadores genéticos.
De este modo, los descendientes reemplazan a los cromosomas de la población anterior.
Este modelo es elitista y produce una convergencia rápida cuando se reemplazan los peores cromosomas de la población.

%TODO
Añadir más cosas


\section{Clases de complejidad}

Explicar que es un problema NP-completo y demás.