% 4. Criptografía post-cuántica basada en códigos. 
% Aquí depende de si vas a matricularte en la asignatura de Criptografía (la imparte Javier Lobillo). 
% Allí te explicarán los sistemas asimétricos. Si no te matriculas, lo puedes encontrar en el libro Introduction to Cryptography (capítulos 1,2,3).
% No te hace falta todo lo que hay en esos capítulos, pero, bueno, es interesante para una matemática/informática el conocer este tipo de cosas.
% Para la parte post-cuántica, tenemos el archivo 9783540887010-c1.pdf y el libro Post-QuantumCryptography 
% (capítulos Introduction to post-quantum cryptography y Code-based cryptography, también Quantum computing, si te apetece). 
% Para el criptosistema de McEliece (que es que implementaremos) está el archivo sander-report-s15.pdf, 
% donde se explica este y el criptosistema de Niederreter, que es equivalente en seguridad.

\chapter{Criptografía post-cuántica basada en códigos}