% 4. Criptografía post-cuántica basada en códigos. 
% Aquí depende de si vas a matricularte en la asignatura de Criptografía (la imparte Javier Lobillo). 
% Allí te explicarán los sistemas asimétricos. Si no te matriculas, lo puedes encontrar en el libro Introduction to Cryptography (capítulos 1,2,3).
% No te hace falta todo lo que hay en esos capítulos, pero, bueno, es interesante para una matemática/informática el conocer este tipo de cosas.
% Para la parte post-cuántica, tenemos el archivo 9783540887010-c1.pdf y el libro Post-QuantumCryptography 
% (capítulos Introduction to post-quantum cryptography y Code-based cryptography, también Quantum computing, si te apetece). 
% Para el criptosistema de McEliece (que es que implementaremos) está el archivo sander-report-s15.pdf, 
% donde se explica este y el criptosistema de Niederreiter, que es equivalente en seguridad.

% buscar la referencia original del 78 de McEliece que ahi viene mejor explicado

\chapter{Criptografía post-cuántica basada en códigos}

% TODO: introducción


El principal objetivo de la criptografía es proporcionar confidencialidad mediante métodos de cifrado. Cuando queremos enviar un mensaje a un destinatario, el canal de comunicación puede ser inseguro y otras personas podrían leerlo o incluso modificarlo de tal forma que el destinatario no se diera cuenta. Para prevenir estos ataques nos será de utilidad la criptografía.


\section{Introducción}

En general, los métodos de cifrado consisten en encriptar el mensaje, \emph{texto plano}, antes de ser transmitido, de esta forma obtenemos un \emph{texto cifrado} o \emph{criptograma}. Este texto cifrado se transmite al destinatario, quien lo \emph{desencripta} mediante una \emph{clave de descifrado}, la cual solo conocen el receptor y el emisor y previamente la intercambiaron.

Formalmente, dado un conjunto de los mensajes (textos planos) $\mathcal{M}$, un conjunto de los criptogramas $\mathcal{C}$ y el espacio de claves $\mathcal{K}_p \times \mathcal{K}_s$, un \emph{criptosistema} viene definido por dos aplicaciones

$$E : \mathcal{K}_p \times \mathcal{M} \rightarrow \mathcal{C},$$
$$D : \mathcal{K}_s \times \mathcal{C} \rightarrow \mathcal{M},$$

tales que para cualquier clave $k_p \in \mathcal{K}_p$, existe una clave $k_s \in \mathcal{K}_s$ de manera que dado cualquier mensaje $m \in \mathcal{M}$,

\begin{equation}
    \label{def:criptosistema}
    D(k_s, E(k_p, m)) = m.
\end{equation}

Para simplificar la notación de las funciones de cifrado y descifrado usaremos, fijadas las claves $k_p \in \mathcal{K}_p$ y su correspondiente $k_s \in \mathcal{K}_s$:

$$E_{k_p} : \mathcal{M} \rightarrow \mathcal{C}, \; \left[ E_{k_p} (m) = E(k_p, m) \right]$$
$$D_{k_s} : \mathcal{C} \rightarrow \mathcal{M}, \; \left[ D_{k_s} (c) = D(k_s, c) \right]$$

La propiedad \ref{def:criptosistema} se transforma en

$$D_{k_s} \left( E_{k_p}(m) \right) = m.$$

Para encriptar y desencriptar existen \emph{algoritmos de cifrado} y \emph{de descifrado}, respectivamente, y cada uno usará una clave secreta. Si esta clave es la misma en ambos algoritmos, diremos que los métodos de encriptación son \emph{simétricos}. Algunos ejemplos importantes de estos métodos son DES (\emph{Data Encryption Standard}) y AES (\emph{Advanced Encryption Standard}).

En 1976, Diffie y Hellman introdujeron un concepto revolucionario, la \emph{criptografía de Clave Pública}, también llamada \emph{criptografía asimétrica}, que permitió dar una solución al antiguo problema del intercambio de claves e indicar el camino a la firma digital. Los métodos de cifrado de \emph{clave pública} son \emph{asimétricos}. Cada receptor tiene una clave personal $k = (k_p, k_s)$, que consiste en dos partes: $k_p$ es la clave de cifrado y es pública, y $k_s$ es la clave de descifrado, que es privada. De esta forma, si queremos enviar un mensaje, lo encriptaremos mediante la clave pública $k_p$ del receptor. Así, el receptor poddrá descifrar el texto cifrado usando su clave privada $k_s$, que solo la conoce él. Al ser la clave pública, cualquiera puede encriptar un mensaje usándola, sin embargo descifrarlo sin saber la clave privada será casi imposible.

\section{Objetivos de la criptografía}

Además de proporcionar confidencialidad, la criptografía proporciona soluciones para otros problemas:

\begin{enumerate}
    \item \emph{Confidencialidad}. La información solo puede ser accesible por las entidades autorizadas.
    \item \emph{Integridad de datos}. El receptor de un mensaje debería ser capaz de determinar que el mensaje no ha sido modificado durante la transmisión.
    \item \emph{Autenticidad}. El receptor de un mensaje debería ser capaz de verificar su origen.
    \item \emph{No repudio}. El emisor de un mensaje debería ser incapaz de negar posteriormente que envió el mensaje.
\end{enumerate}

Para garantizar la integridad de los datos, hay métodos simétricos y de clave pública. El mensaje $m$ es aumentado por un \emph{código de autenticación de mensaje} (MAC). Este código es generado por un algoritmo que depende de la clave secreta. Así, el mensaje aumentado $(m, MAC(k,m))$ está protegido contra modificaciones. El receptor ahora puede comprobar la integridad del mensaje $(m, \bar{m})$ verificando que $MAC(k, m) = \bar{m}$. Más adelante veremos que la autenticación de códigos se puede implementar con funciones hash con clave.

Las firmas digitales requieren métodos de clave pública y proporcionan autenticación y no repudiablidad. Cualquier persona puede verificar si una firma digital es válida con la clave pública del firmante. Esto es, si firmamos con nuestra clave privada $k$, obtenemos la firma $Sign(k_s, m)$. El receptor recibe la firma $s$ del mensaje $m$ y comprueba con el algoritmo de verificación \emph{Verify} que se cumple que \emph{Verify}$(k_p, s, m) = ok$, siendo $k_p$ la clave pública del emisor.

\section{Criptografía simétrica}

Un criptosistema \emph{simétrico}, como hemos visto, viene determinado por dos aplicaciones

$$E : \mathcal{K} \times \mathcal{M} \rightarrow \mathcal{C},$$
$$D : \mathcal{K} \times \mathcal{C} \rightarrow \mathcal{M},$$

tales que para cualquier clave $k \in \mathcal{K}$ y cualquier mensaje $m \in \mathcal{M}$,

\begin{equation}
    \label{def:criptosistema_simetrico}
    D(k, E(k, m)) = m.
\end{equation}

Fijada $k \in \mathcal{K}$, usaremos la notación 

$$E_k : \mathcal{M} \rightarrow \mathcal{C},$$
$$D_k : \mathcal{C} \rightarrow \mathcal{M},$$

para las funciones de cifrado y descifrado. La propiedad \ref{def:criptosistema_simetrico} se transforma en

$$D_k \left( E_k (m) \right) = m.$$

Observamos que la clave que se usa para cifrar es la misma que se usa para descifrar.

Algunos ejemplos importantes de sistemas simétricos son DES (\emph{Data Encryption Standard}) y AES (\emph{Advanced Encryption Standard}).

\section{Criptografía asimétrica}

A diferencia de la criptografía simétrica, en la criptografía asimétrica los participantes en la comunicación no comparten una clave secreta. Cada uno tiene un par de claves: la \emph{clave secreta} $k_s$ conocida solo por él y una \emph{clave pública} conocida por todos.

Supongamos que Bob tiene un par de claves $(k_p, k_s)$ y Alice quiere encriptar un mensaje $m$ para Bob. Alice, como cualquier otra persona, conoce la clave pública $k_p$ de Bob. Alice usa una función de encriptación $E$ con la clave pública $k_p$ de Bob para obtener el texto cifrado $c = E_{k_p}(m)$. Esto solo puede ser seguro si es prácticamente inviable calcular $m$ de $c = E_{k_p}(m)$. Sin embargo, Bob sí es capaz de calcular el mensaje $m$, ya que puede usar su clave secreta. La función de encriptación $E_{k_p}$ debe tener la propiedad de que su pre-imagen $m$ del texto cifrado $c = E_{k_p}(m)$ sea fácil de calcular usando la clave secreta $k_s$ de Bob, quien es el único que puede descifrar el mensaje encriptado.

En la criptografía de clave pública, necesitamos unas funciones $\left( E_{k_p} \right)_{k_p \in K_P}$ tales que cada función $E_{k_p}$ se pueda cacular con un algoritmo eficiente. Sin embargo, su pre-imagen debería ser prácticamente inviable de calcular. Estas familias $\left( E_{k_p} \right)_{k_p \in K_P}$ se denominan \emph{funciones de una sola dirección}. En cada función $E_{k_p}$ de la familia, tiene que haber una información secreta $k_s$ para que el algoritmo sea eficiente y calcule la inversa de $E_{k_p}$. Las funciones con esa propiedad se denominan \emph{funciones con trampilla}.

En 1976, Diffie y Hellman presentaron la idea de la criptografía de clave pública, es decir, introdujeron métodos de clave pública para el acuerdo de clave y, además, describieron cómo las firmas digitales funcionarían. El primer criptosistema de clave pública que podía servir como un mecanismo de acuerdo de clave y como una firma digital fue el criptosistema RSA, que actualmente es el más conocido y usado. Este criptosistema lleva el nombre de sus inventores: Rivest, Shamir y Adleman. El criptosistema RSA se basa en la dificultad de factorizar grandes números, lo que le permite construir funciones de una sola dirección con una trampilla. Otras funciones de una sola dirección se basan en la dificultad de extraer logaritmos discretos. Estos dos problemas de la teoría de números son los cimientos de los criptosistemas de clave pública más usados actualmente.

Cada participante en un criptosistema de clave pública tiene una clave $k = (k_p, k_s)$, que consiste enu na clave pública ($k_p$) y una clave privada o secreta ($k_s$). Para garantizar la seguridad del criptosistema, debería ser inviable obtener la clave privada $k_s$ a partir de la clave pública $k_p$. Un algoritmo eficiente debería ser el encargado de elegir aleatoriamente ambas claves en un gran espacio de parámetros. Así, cualquiera puede usar $k_p$ para encriptar mensajes, pero solo quien posea $k_s$ podrá desencriptarlos.

En cuanto a las firmas digitales, supongamos que tenemos una familia de funciones con trampilla $\left( E_{k_p} \right)_{k_p \in K_P}$ donde cada función $E_{k_p}$ es biyectiva. Sea $k_p$ la clave pública de Alice, quien es la única de calcular la inversa $E_{k_p}^{-1}$ de $E_{k_p}$ pues para ello se necesita la clave privada $k_s$ de Alice. De esta forma, si Alice quiere firmar un mensaje $m$, tiene que calcular $E_{k_p}^{-1}(m)$, que será el valor de la firma $s$ de $m$. Todo el mundo puede verificar la firma de Alice $s$ pues cualquiera puede usar su clave pública $k_p$ y calcular $E_{k_p}(s)$. Si $E_{k_p} (s) = m$, entonces podemos asegurarnos de que Alice realmente firmó $m$ porque es la única que es capaz de calcular $E_{k_p}^{-1}(m)$.

Una importante aplicación de los criptosistemas de clave pública es que permiten intercambiar claves en sistemas de clave secreta. Si Alice conoce la clave pública de Bob, ella puede generar una clave de sesión, cifrarla con la clave pública de Bob y enviársela.  

Algunos sistemas conocidos de clave pública son:

\begin{itemize}
    \item \emph{RSA}: está basado en el problema de factorización de enteros.
    \item \emph{ElGamal}: está basado en el problema del logaritmo discreto.
    \item \emph{McEliece}: está basado en la teoría de los códigos Goppa.
    \item \emph{Curvas Elípticas}: son una generalización del sistema ElGamal y se basan en el problema de calcular logaritmos discretos en curvas elípticas.
\end{itemize}

\subsection{RSA}

El criptosistema RSA consiste en multiplicar dos números primos muy grandes y publicar su producto $n$. Una parte de la clave pública la conformará $n$, mientras que los factores de $n$ se mantienen en secreto y se usarán como clave privada. La idea básica es que los factores de $n$ no puedan recuperarse de $n$. Por lo que la seguridad de RSA radica en la dificultad del problema de factorización de enteros.

\subsubsection{Generación de claves}

Cada usuario del criptosistema RSA posee una clave pública y otra privada. Para generar este par de claves, se siguen los siguientes tres pasos:

\begin{enumerate}
    \item Se eligen aleatoriamente dos grandes números primos distintos $p$ y $q$ y se calcula su producto $n = p \cdot q$.
    \item Se elige un entero $e$ tal que $1 < e < \phi(n)$ y sea primo con $\phi(n)$. La clave pública será $(n, e)$.
    \item Se calcula $d$ que verifique $ed \equiv 1 \pmod{\phi(n)}$. La clave privada será $(n, d)$.
\end{enumerate}

Los números $n$, $e$ y $d$ se denominan \emph{módulo}, \emph{exponente de cifrado} y \emph{exponente de descifrado}, respectivamente. El exponente de descifrado $d$ se puede obtener con el algoritmo extendido de Euclides. Con este exponente es posible descifrar un texto cifrado y generar una firma digital.

\subsubsection{Cifrado y descifrado}

Supongamos que queremos cifrar un mensaje para Bob. Para ello, usaremos la clave pública $(n, e)$ de Bob. Los mensajes que podemos encriptar $m \in \{ 0, ..., n-1 \}$ se pueden considerar elementos de $\mathbb{Z}_n$.

Para cifrar un texto plano $m \in \mathbb{Z}_n$, podemos usar la función RSA

$$RSA_{n, e} : \mathbb{Z}_n \longrightarrow \mathbb{Z}_n$$
$$ RSA_{n, e}(m) = m^e \pmod{n},$$

es decir, el texto cifrado $c$ es $m^e$ módulo $n$.

Para que Bob descifre el texto cifrado $c$, puede usar la función RSA

$$RSA_{n, d} : \mathbb{Z}_n \longrightarrow \mathbb{Z}_n, \; c \longmapsto c^d,$$

donde $(n, d)$ es su clave secreta. De esta forma, Bob puede recuperar el texto plano, es decir, $m = c^d$ módulo $n$, pues las funciones de cifrado y descifrado $RSA_{n, e}$ y $RSA_{n, d}$ son inversas entre sí.

Con este procedimiento de cifrado, podemos cifrar secuencias de bit hasta $k := \lfloor \log_2{n} \rfloor$ bits. Si los mensajes son más largos, podemos dividirlos en bloques de longitud $k$ y cifrar cada uno por separado.

\subsubsection{Firma digital}

El criptosistema RSA también se puede usar para realizar firmas digitales. Sea $(n, e)$ la clave pública y $d$ el exponente de descifrado, si queremos firmar un mensaje $m$, le aplicamos el algoritmo de descifrado y obtenemos la \emph{firma} de $m$, $\sigma = m^d$. Decimos que $(m, \sigma)$ es un \emph{mensaje firmado}. Para verificar ese mensaje, basta con calcular $\sigma ^e$, donde $e$ es la clave pública del firmante, y comprobar que coincide con $m$.

\section{Criptografía post-cuántica}

Se cree que algumos sistemas criptográficos, tales como RSA, resisten los ataques de grandes ordenadores clásicos, pero no de los grandes ordenadores cuánticos. Sin embargo, se cree que algunas alternativas, como el sistema de McEliece con una clave de cuatro millones de bits, son capaces de resistir los ataques de grandes ordenadores clásicos y cuánticos.

Es por esto que surgen diversas preocupaciones, ante la amenaza de los ordenadores cuánticos se duda sobre si seguir usando RSA o simplemente cambiar a otros sistemas criptográficos que sean resistentes a dichos ordenadores. Sin embargo, esto no es tan sencillo pues necesitamos tiempo para mejorar la eficiencia, fomentar la confanza y mejorar de la usabilidad de la criptografía post-cuántica. En breve, todavía no estamos preparados para que el mundo cambie a la criptografía post-cuántica.

En esta sección estudiaremos los sistemas criptográficos basados en códigos, criptosistemas que usan un código de corrección de errores $\mathcal{C}$. Para ello, agregan un error a una palabra de $\mathcal{C}$ o al calcular un síndrome relativo a la matriz de paridad de $\mathcal{C}$.

El primero de esos sistemas es un sistema de cifrado de clave pública y fue propuesto por Robert J. McEliece in 1978. Este sistema usa como clave privada un código de Goppa binario aleatorio y, como clave pública, una matriz generadora aleatoria de una versión permutada aleatoriamente de ese código. El texto cifrado es una palabra código a la que se le han añadido algunos errores, y solo el que posee la clave privada puede eliminar esos errores. Actualmente, no se conoce ningún ataque que presente una amenaza grave a este sistema, ni los ordenadores cuánticos. Cabe destacar la seguridad y la rapidez del criptosistema de McEliece, ya que tanto los procedimientos de cifrado y descifrado son de baja complejidad.

\subsection{Criptosistema de McEliece}

En 1978, R. J. McEliece propuso un criptosistema de clave pública \cite{McEliece_1978}. El criptosistema de McEliece se basa en códigos lineales de corrección de errores y hasta ahora es el criptosistema más exitoso basado en nociones de teoría de codificación.

La construcción original en \cite{McEliece_1978} usa códigos de Goppa binarios para cifrar y descifrar mensajes. Sim embargo, han surgido otras variantes de este criptosistema que usan otros códigos lineales, pero la mayoría han resultado ser inseguros. A día de hoy, la construcción original de 1978 se considera segura con la correcta elección de parámetros.

Por esta razón, el criptosistemade McEliece se pone a la altura del RSA. Existen algunas diferencias entre ambos criptosistemas, el de McEliece es capaz de cifrar y descifrar mensajes más rápido. Sin embargo, los tamaños de las claves son mayores que los del RSA, que es la razón por la que el criptosistema de McEliece apenas se usa. La principal razón por la que está creciendo el interés en el criptosistema de McEliece es porque es uno de los mejores candidatos para criptosistemas de clave pública seguros post-cuánticos. El esquema Niederreiter, una variante del criptosistema de McEliece, también permite construir un esquema de firma digital segura.

En esta sección presentaremos el criptosistema de McEliece incluyendo algunas de sus variantes y estudiaremos su seguridad y los ataques más conocidos.

\subsection{Original construcción}

El criptosistema de McEliece usa códigos lineales de corrección de rrores para cifrar mensajes. Este criptosistema posee una clave privada, que se elige al generar la clave y contiene la descripción del código lineal estructurado, y una clave pública que se basa en una versión suficientemente aleatorizada de ese mismo código. De esta forma, será difícil descifrar un mensaje sin conocer la estructura del código lineal (clave privada), pues es lo que proporciona un descifrado rápido.

Los códigos que usa la construcción original son los códigos de Goppa binarios irreducibles. Estos códigos son muy adecuados pues poseen altas capacidades de corrección de erorres y matrices generadoras densas, que son difíciles de distinguir de matrices binarias aleatorias.

\subsubsection{Generación de claves}

La generación de la clave de este criptosistema se obtiene a partir de los siguientes pasos:

\begin{enumerate}
    \item Se elige un $[n, k, 2t + 1]$-código lineal aleatorio $\mathcal{C}$ sobre $\mathbb{F}_2$ que tenga un algoritmo de decodificación eficiente $\mathcal{D}$ que sea capaz de corregir hasta $t$ errores.
    \item Se calcula la matriz generadora $G$ de dimensión $k \times n$ para $\mathcal{C}$.
    \item Se genera una matriz no singular binaria aleatoria $S$ de dimensión $k \times k$.
    \item Se genera una matriz de permutaciones aleatoria $P$ de dimensión $n \times n$.
    \item Se calcula la matriz $G' = SGP$ de dimensión $k \times n$. La clave pública es $(G', t)$ y la clave privada es $(S, G, P, \mathcal{D})$.
\end{enumerate}

Es decir, la matriz $G'$ se obtiene al permutar las columnas de $G$ a partir de la matriz $P$ y luego realizar un cambio de base con la matriz $S$. De esta forma, la matriz $G'$ corresponde a un $[n, k, 2t + 1]$-código lineal que es equivalente permutacionalmente a la clave privada elegida. Llamaremos a $G'$ como la \emph{matriz generadora pública}.

\subsubsection{Cifrado y descifrado}

Una vez tenemos la clave, podemos \textbf{cifrar} un texto plano $\textbf{m} \in \{ 0, 1 \}^k$ eligiendo un vector aleatorio $\textbf{e} \in \{ 0, 1 \}^n$ de peso $t$ y calcular el texto cifrado como

$$\textbf{c} = \textbf{m} G' + \textbf{e}.$$

Para recuperar $\textbf{m}$ eficientemente, podemos usar el siguiente algoritmo de \textbf{descifrado}. Sea un criptograma $\textbf{c} \in \{ 0, 1 \}^n$, primero calculamos

$$\textbf{c} P^{-1} = (\textbf{m} S) G + \textbf{e} P^{-1}.$$

Ahora ya podemos aplicar el algoritmo de descifrado $\mathcal{D}$ a $\textbf{c} P^{-1}$ para obtener $\textbf{c'} = \textbf{m} S$, pues $(\textbf{m} S) G$ es una palabra código válida para el código lineal elegido y $\textbf{e} P^{-1}$ tiene peso $t$. Finalmente, podemos calcular el mensaje $\textbf{m}$ con

$$\textbf{m} = \textbf{c'} S^{-1}.$$

Por otra parte, a la hora de aplicar el algoritmo de Sugiyama, sabemos que este algoritmo de descifrado necesita saber el polinomio generador. De este modo consideraremos la clave privada como $(S, G, P, g(x))$, donde $g(x)$ es el polilnomio de Goppa para el código elegido.


\subsection{Criptosistema Niederreiter}

En 1986, H. Niederreiter propuso una variante relevante del criptosistema de McEliece en \cite{Niederreiter_1986}. A diferencia del criptosistema de McEliece, este criptosistema usa una matriz de paridad en vez de una matriz generadora.

Antes de introducir este criptosistema, necesitamos definir el concepto de síndrome de una palabra en $\mathbb{F}^n$.

\begin{definition}
    Sea $\mathcal{C}$ un $[n, k, d]$-código lineal sobre $\mathbb{F}$ y sea $H$ su matriz de paridad. El \emph{síndrome} de una palabra $\textbf{y} \in \mathbb{F}^n$ se define como

    $$\textbf{s} = \textbf{y} H^T.$$
\end{definition}

Sabemos que una palabra código de $\mathcal{C}$ tiene síndrome igual a $\textbf{0}$, según la definición de la matriz de paridad. Sean $\textbf{y}_1$, $\textbf{y}_2 \in \mathbb{F}^n$ dos vectores, entonces

$$\textbf{y}_1 - \textbf{y}_2 \in \mathcal{C} \Leftrightarrow (\textbf{y}_1 - \textbf{y}_2) H^T = \textbf{0} \Leftrightarrow \textbf{y}_1 H^T = \textbf{y}_2 H^T.$$

El hecho de que $\textbf{y}_1 - \textbf{y}_2$ sea una palabra código de $\mathcal{C}$ si y solo si los síndromes de $\textbf{y}_1$ e $\textbf{y}_2$ son iguales es la base para un método eficiente para implementar la decodificación de la palabra código más cercana que se llama \emph{decodificación del síndrome}.

Sea una palabra $\textbf{y} \in \mathbb{F}^n$, un algoritmo de decodificación de síndrome $\mathcal{D}$ encuentra una palabra con el mínimo peso $\textbf{e} \in \mathbb{F}^n$ tal que

$$\textbf{y} H^T = \textbf{e} H^T.$$

Si $\textbf{y}$ tiene la forma $\textbf{y} = \textbf{c} + \textbf{e'}$, donde $\textbf{c} \in \mathcal{C}$ y $\textbf{w}(\textbf{e'}) \leq t$, entonces $\textbf{e} = \textbf{e'}$, esto es, el algoritmo de decodificación de síndrome encuentra exactamente el vector error que se le añadió a la palabra código.

Ahora, ya podemos definir el criptosistema Niederreiter, que se basa en la idea de la decodificación del síndrome.

\subsubsection{Generación de claves}

La generación de la clave de este criptosistema se obtiene a partir de los siguientes pasos:

\begin{enumerate}
    \item Se elige un $[n, k, 2t + 1]$-código lineal aleatorio $\mathcal{C}$ sobre $\mathbb{F}_2$ que tenga un algoritmo de decodificación del síndrome eficiente $\mathcal{D}$ que sea capaz de corregir hasta $t$ errores.
    \item Se calcula la matriz de paridad $H$ de dimensión $(n - k) \times n$ para $\mathcal{C}$.
    \item Se genera una matriz no singular binaria aleatoria $S$ de dimensión $(n - k) \times (n - k)$.
    \item Se genera una matriz de permutaciones aleatoria $P$ de dimensión $n \times n$.
    \item Se calcula la matriz $H' = SHP$ de dimensión $(n - k) \times n$. La clave pública es $(H', t)$ y la clave privada es $(S, H, P, \mathcal{D})$.
\end{enumerate}

\subsubsection{Cifrado y descifrado}

Una vez tenemos la clave, podemos \textbf{cifrar} un texto plano $\textbf{m} \in \{ 0, 1 \}^k$ con peso $t$, calculando el texto cifrado como el síndrome de $\textbf{m}$

$$\textbf{c} = \textbf{m} H^{'T}.$$

Para recuperar $\textbf{m}$ eficientemente, podemos usar el siguiente algoritmo de \textbf{descifrado}. Sea un criptograma $\textbf{c}$, primero calculamos

$$S^{-1} \textbf{c}^T = H P \textbf{m}^T.$$

Luego, buscamos un vector $\textbf{z} \in \mathbb{F}^n$ tal que $H \textbf{z}^T = HP \textbf{m}^T$. Entonces $\textbf{z} - (P\textbf{m}^T)^T = \textbf{z} - \textbf{m}P^T$ es una palabra código válida en $\mathcal{C}$ debido a que los síndromes de las palabras son iguales. Como $\textbf{m}P^T$ tiene peso $t$, podemos aplicar $\mathcal{D}$ a $\textbf{z}$ para encontrar el vector error $\textbf{m}P^T$ y por lo tanto $\textbf{m}$.

Observemos que el mensaje en texto plano se representa como el error de la palabra código en lugar de la palabra con la información original. De este modo, el mensaje en texto plano debe codificarse adicionalmente en un vector de peso $t$ para el cifrado.

A diferencia con el criptosistema de McEliece, el algoritmo de decodificación del síndrome es más eficiente que el algoritmo de decodificación del criptosistema de McEliece. Además, tambien se puede usar para construir un esquema de firma digital.

Sin embargo, si un ataque es capaz de romper el criptosistema de McEliece, también puede romper el esquema Niederreiter y viceversa. Esto se debe a que los criptosistemas de McEliece y Niederreiter se basan en el uso de códigos lineales que son duales entre si y la matriz generadora se puede obtener eficientemente a partir de la matriz de paridad y viceversa.

\subsection{Seguridad del criptosistema de McEliece}

En esta sección estudiaremos la seguridad del criptosistema de McEliece original y analizaremos los ataques más conocidos ante este criptosistema. Como la seguridad de los esquemas de McEliece y Niederreiter es equivalente \cite{Equivalence_McEliece_Niederreiter}, luego todos los ataques que analicemos también estarán relacionados implícitamente con el esquema Niederreiter.

Fijemos un código de Goppa $\Gamma(L, g) \subset \mathbb{F}_2^n$, con $g(x) \in \mathbb{F}_{2^m}[x]$ y $L$ una tupla de $n$ elementos distintos de $\mathbb{F}_{2^m}$, capaz de corregir hasta $t$ errores. Tenemos que la dimensión del código es $k = n - tm$. Sea $G$ la matriz generadora binaria de dimensión $k \times n$ de $\Gamma$ y definimos $G' = SGP$ como la clave pública de McEliece, donde $S$ es la matriz binaria no singular de dimensión $k \times k$ y $P$ es la matriz de permutaciones de dimensión $n \times n$.

Para estudiar la seguridad de este criptosistema, tenemos que determinar cómo de difícil es determinar el mensaje $\textbf{m}$ a partir de conocer $G'$ y haber interceptado $\textbf{c}$. Como se indica en \cite{McEliece_1978}, pueden darse dos ataques básicos:

\begin{enumerate}
    \item Intentar recuperar la clave secreta $G$ a partir de $G'$ y así descifrar el mensaje.
    \item Intentar recuperar el mensaje original $\textbf{m}$ a partir del criptograma $\textbf{c}$ sin conocer la clave privada $G$.
\end{enumerate}

El primer ataque parece inviable si $n$ y $t$ son suficientemente grandes, pues existen demasiadas posibilidades tanto para $G$ como para $S$ y $P$.

El segundo ataque puede ser más prometedor para el adversario pues puede aproximarse mediante la decodificación de conjuntos de información.

La seguridad del criptosistema de McEliece se puede insinuar por la intratabilidad de los siguientes problemas fundamentales en la teoría de la codificación.

\begin{problemth}[Problema general de decodificación para códigos lineales]
    Sea $\mathcal{C}$ un $[n, k]$-código lineal sobre $\mathbb{F}$ e $\textbf{y} \in \mathbb{F}^n$. Encontrar una palabra código $\textbf{c} \in \mathcal{C}$ tal que la distancia $d(\textbf{y}, \textbf{c})$ sea mínima.
\end{problemth}

\begin{problemth}[Problema de encontrar una palabra código dado un peso]
    Sea $\mathcal{C}$ un $[n, k]$-código lineal sobre $\mathbb{F}$ y $w \in \mathbb{N}$. Encontrar una palabra código $\textbf{c} \in \mathcal{C}$ tal que $\textbf{w}(\textbf{c}) = w$.
\end{problemth}

Se ha demostrado que ambos problemas son NP-duro \cite{Intractability_coding_problems}. No obstante, esto no implica que romper el criptosistema de McEliece sea NP-duro, pues los códigos binarios de Goppa solo cubren una fracción de todos los códigos lineales posibles. Es por esto que la seguridad del criptosistema de McEliece se basa en la suposición de que la clave pública es indistinguible de cualquier matriz aleatoria.

\subsubsection{Seguridad post-cuántica}

El criptosistema de McEliece es inmune ante el algoritmo de Shor \cite{Shor_1997}, que si se implementara en un ordenador cuántico podría romper otros criptosistemas de clave pública tales como RSA. 