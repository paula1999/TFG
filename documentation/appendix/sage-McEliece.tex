\chapter[Implementación en SageMath del criptosistema de McEliece]{Implementación en SageMath del criptosistema de McEliece}
\label{annex:sage-McEliece}

En este anexo describimos la documentación de la clase desarrollada para implementar en SageMath el criptosistema de McEliece. Se ha implementado la clase \texttt{McEliece} para representar dicho sistema, esto es, permite generar la clave pública y la clave privada, las cuales se emplearán para cifrar un texto plano y descifrar un criptograma, respectivamente. También se describen las funciones auxiliares desarrolladas. 

El código desarrollado se encuentra en
\begin{center}
\url{https://github.com/paula1999/TFG/tree/main/src}.
\end{center}

\section{Clase para el criptosistema McEliece}

\begin{description}[leftmargin=1em, font=\normalfont\ttfamily, style=nextline]
  \item[class McEliece(self, n, q, g)]

  Representación del criptosistema de McEliece.

  \textsc{Argumentos}
  \begin{description}[font=\normalfont\ttfamily]
  \item[n] Tamaño del conjunto de definición.
  \item[q] Dimensión del cuerpo finito $GF(2^q)$.
  \item[g] Polinomio mónico con coeficientes en un cuerpo finito $GF(2^q)$.
  \end{description}

  \textsc{Ejemplos}
  \begin{lstlisting}[gobble=4]
    sage: L = GF(2^5)
    sage: a = L.gen()
    sage: R.<x> = L[]
    sage: g = x^3 + (a^4 + a^3 + 1)*x^2 + (a^4 + 1)*x + a^4 + a^2 + a + 1
    sage: n = 20
    sage: McEliece(n, q, g)
    > McEliece cryptosystem over [20, 5] Goppa code
  \end{lstlisting}

  \begin{description}[font=\ttfamily, style=nextline]
  \item[get\_S(self)] Obtiene la matriz no singular binaria aleatoria que es parte de la clave privada.
  
  \textsc{Salida}
  \begin{description}[font=\normalfont\ttfamily]
    \item[] Matriz no singular binaria aleatoria que es parte de la clave privada.
  \end{description}

  \textsc{Ejemplos}
  \begin{lstlisting}[gobble=4]
    sage: L = GF(2^5)
    sage: a = L.gen()
    sage: R.<x> = L[]
    sage: g = x^3 + (a^4 + a^3 + 1)*x^2 + (a^4 + 1)*x + a^4 + a^2 + a + 1
    sage: n = 20
    sage: ME = McEliece(n, q, g)
    sage: ME.get_S()
    > [1 1 1 1 1]
      [0 1 1 1 0]
      [1 0 1 0 1]
      [0 1 0 0 0]
      [0 1 1 1 1]
  \end{lstlisting}

  \item[get\_G(self)] Obtiene la matriz generadora asociada al código de Goppa que es parte de la clave privada.
  
  \textsc{Salida}
  \begin{description}[font=\normalfont\ttfamily]
    \item[] Matriz generadora asociada al código de Goppa que es parte de la clave privada.
  \end{description}

  \textsc{Ejemplos}
  \begin{lstlisting}[gobble=4]
    sage: L = GF(2^5)
    sage: a = L.gen()
    sage: R.<x> = L[]
    sage: g = x^3 + (a^4 + a^3 + 1)*x^2 + (a^4 + 1)*x + a^4 + a^2 + a + 1
    sage: n = 20
    sage: ME = McEliece(n, q, g)
    sage: ME.get_G()
    > [1 0 0 0 1 0 0 1 1 1 0 0 1 0 0 1 0 0 0 0]
      [0 1 0 0 0 0 0 1 1 1 1 0 0 1 0 1 1 0 1 1]
      [0 0 1 0 0 0 0 1 0 0 1 0 1 1 1 0 0 1 1 0]
      [0 0 0 1 1 1 0 0 0 0 0 0 1 1 0 0 0 1 0 1]
      [0 0 0 0 0 0 1 1 0 1 0 1 1 0 0 1 0 1 1 0]
  \end{lstlisting}

  \item[get\_P(self)] Obtiene la matriz de permutaciones aleatoria que es parte de la clave privada.
  
  \textsc{Salida}
  \begin{description}[font=\normalfont\ttfamily]
    \item[] Matriz de permutaciones aleatoria que es parte de la clave privada.
  \end{description}

  \textsc{Ejemplos}
  \begin{lstlisting}[gobble=4]
    sage: L = GF(2^5)
    sage: a = L.gen()
    sage: R.<x> = L[]
    sage: g = x^3 + (a^4 + a^3 + 1)*x^2 + (a^4 + 1)*x + a^4 + a^2 + a + 1
    sage: n = 20
    sage: ME = McEliece(n, q, g)
    sage: ME.get_P()
    > [0 0 0 1 0 0 0 0 0 0 0 0 0 0 0 0 0 0 0 0]
      [0 0 0 0 0 0 0 0 0 0 1 0 0 0 0 0 0 0 0 0]
      [0 0 0 0 0 0 0 0 0 0 0 0 0 0 0 1 0 0 0 0]
      [0 1 0 0 0 0 0 0 0 0 0 0 0 0 0 0 0 0 0 0]
      [0 0 0 0 0 0 0 1 0 0 0 0 0 0 0 0 0 0 0 0]
      [0 0 0 0 0 0 0 0 0 0 0 0 1 0 0 0 0 0 0 0]
      [0 0 0 0 1 0 0 0 0 0 0 0 0 0 0 0 0 0 0 0]
      [0 0 0 0 0 0 0 0 0 0 0 1 0 0 0 0 0 0 0 0]
      [0 0 0 0 0 0 0 0 0 0 0 0 0 1 0 0 0 0 0 0]
      [0 0 0 0 0 0 0 0 0 0 0 0 0 0 0 0 0 1 0 0]
      [0 0 0 0 0 0 0 0 0 1 0 0 0 0 0 0 0 0 0 0]
      [1 0 0 0 0 0 0 0 0 0 0 0 0 0 0 0 0 0 0 0]
      [0 0 0 0 0 0 0 0 0 0 0 0 0 0 0 0 0 0 0 1]
      [0 0 0 0 0 1 0 0 0 0 0 0 0 0 0 0 0 0 0 0]
      [0 0 0 0 0 0 0 0 0 0 0 0 0 0 1 0 0 0 0 0]
      [0 0 0 0 0 0 0 0 0 0 0 0 0 0 0 0 1 0 0 0]
      [0 0 0 0 0 0 0 0 0 0 0 0 0 0 0 0 0 0 1 0]
      [0 0 0 0 0 0 1 0 0 0 0 0 0 0 0 0 0 0 0 0]
      [0 0 1 0 0 0 0 0 0 0 0 0 0 0 0 0 0 0 0 0]
      [0 0 0 0 0 0 0 0 1 0 0 0 0 0 0 0 0 0 0 0]
  \end{lstlisting}

  \item[get\_public\_key(self)] Devuelve la matriz que es parte de la clave pública.

  \textsc{Salida}
  \begin{description}[font=\normalfont\ttfamily]
    \item[] Clave pública.
  \end{description}

  \textsc{Ejemplos}
  \begin{lstlisting}[gobble=4]
    sage: L = GF(2^5)
    sage: a = L.gen()
    sage: R.<x> = L[]
    sage: g = x^3 + (a^4 + a^3 + 1)*x^2 + (a^4 + 1)*x + a^4 + a^2 + a + 1
    sage: n = 20
    sage: McEliece(n, q, g)
    sage: ME.get_public_key()
    > [0 0 1 0 1 1 1 1 1 1 0 0 0 1 0 0 0 0 0 1]
      [1 1 0 1 1 1 1 1 1 1 0 1 1 0 0 1 1 1 1 1]
      [0 1 1 1 0 1 1 0 1 0 0 0 0 0 1 1 1 0 0 0]
      [0 0 0 0 1 0 1 1 1 0 0 1 1 0 0 0 1 1 1 0]
      [1 0 0 1 0 1 0 1 1 0 0 1 1 0 0 1 0 0 1 0]
  \end{lstlisting}

  \item[encrypt(self, m)] Devuelve el criptograma asociado al texto plano \texttt{m}.

  \textsc{Argumentos}
  \begin{description}[font=\normalfont\ttfamily]
    \item[m] Texto plano que se va a encriptar.
  \end{description}

  \textsc{Salida}
  \begin{description}[font=\normalfont\ttfamily]
    \item[] Mensaje \texttt{m} cifrado.
  \end{description}

  \textsc{Ejemplos}
  \begin{lstlisting}[gobble=4]
    sage: L = GF(2^5)
    sage: a = L.gen()
    sage: R.<x> = L[]
    sage: g = x^3 + (a^4 + a^3 + 1)*x^2 + (a^4 + 1)*x + a^4 + a^2 + a + 1
    sage: n = 20
    sage: McEliece(n, q, g)
    sage: message = vector(GF(2), (1, 1, 0, 0, 0))
    sage: ME.encrypt(message)
    > (1, 1, 1, 1, 0, 0, 0, 0, 0, 0, 0, 1, 1, 1, 0, 1, 1, 0, 1, 0)
  \end{lstlisting}

  \item[decrypt(self, c)] Devuelve el texto plano asociado al criptograma \texttt{c}.

  \textsc{Argumentos}
  \begin{description}[font=\normalfont\ttfamily]
    \item[c] Criptograma que se va a descifrar.
  \end{description}

  \textsc{Salida}
  \begin{description}[font=\normalfont\ttfamily]
    \item[] Criptograma \texttt{c} descifrado.
  \end{description}

  \textsc{Ejemplos}
  \begin{lstlisting}[gobble=4]
    sage: L = GF(2^5)
    sage: a = L.gen()
    sage: R.<x> = L[]
    sage: g = x^3 + (a^4 + a^3 + 1)*x^2 + (a^4 + 1)*x + a^4 + a^2 + a + 1
    sage: n = 20
    sage: McEliece(n, q, g)
    sage: encrypted_message = vector(GF(2), (1, 1, 1, 1, 0, 0, 0, 0, 0, 0, 0, 1, 1, 1, 0, 1, 1, 0, 1, 0))
    sage: ME.decrypt(encrypted_message)
    > (1, 1, 0, 0, 0)
  \end{lstlisting}
  \end{description}
\end{description}

\section{Funciones auxiliares}

\begin{description}[leftmargin=1em, font=\normalfont\ttfamily, style=nextline]
  \item[get\_weight(c)] Calcula el peso del vector \texttt{c}.

  \textsc{Argumentos}
  \begin{description}[font=\normalfont\ttfamily]
    \item[c] Vector al que se le va a calcular su peso.
  \end{description}

  \textsc{Salida}
  \begin{description}[font=\normalfont\ttfamily]
    \item[] Peso del vector \texttt{c}.
  \end{description}

  \textsc{Ejemplos}
  \begin{lstlisting}[gobble=4]
    sage: v = vector(GF(2), (1, 0, 1, 1, 0))
    sage: get_weight(v)
    > 3
  \end{lstlisting}
\end{description}