\chapter[Implementación en SageMath del criptosistema de Niederreiter]{Implementación en SageMath del criptosistema de Niederreiter}
\label{annex:sage-Niederreiter}

% TODO: introducción

El código desarrollado se encuentra en
\begin{center}
    \url{https://github.com/paula1999/TFG/tree/main/src}.
\end{center}


\section{Clase para el criptosistema Niederreiter}

\begin{description}[leftmargin=1em, font=\normalfont\ttfamily, style=nextline]
    \item[class Niederreiter(self, n, p, q, g)]
  
    Representación del criptosistema de Niederreiter
  
    \textsc{Argumentos}
    \begin{description}[font=\normalfont\ttfamily]
        % TODO
        \item[]
    \end{description}

    \textsc{Ejemplos}
    \begin{lstlisting}[gobble=4]
        % TODO
    \end{lstlisting}

    \begin{description}[font=\ttfamily, style=nextline]
        \item[get\_public\_key(self)] Devuelve la matriz generadora pública que es parte de la clave pública.

        \textsc{Ejemplos}
        \begin{lstlisting}[gobble=4]
            % TODO
        \end{lstlisting}

        \item[encrypt(self, m)] Devuelve el criptograma asociado al texto plano \texttt{m}.

        \textsc{Argumentos}
        \begin{description}[font=\normalfont\ttfamily]
            \item[m] Texto plano que se va a encriptar.
        \end{description}
        
        \textsc{Ejemplos}
        \begin{lstlisting}[gobble=4]
            % TODO
        \end{lstlisting}

        \item[decrypt(self, c)] Devuelve el texto plano asociado al criptograma \texttt{c}.

        \textsc{Argumentos}
        \begin{description}[font=\normalfont\ttfamily]
            \item[c] Criptograma que se va a desencriptar.
        \end{description}
        
        \textsc{Ejemplos}
        \begin{lstlisting}[gobble=4]
            % TODO
        \end{lstlisting}

    \end{description}
\end{description}

\section{Funciones auxiliares}

\begin{description}[leftmargin=1em, font=\normalfont\ttfamily, style=nextline]
    \item[get\_weight(c)]
  
    Calcula el peso del vector \texttt{c}.
  
    \textsc{Argumentos}
    \begin{description}[font=\normalfont\ttfamily]
        \item[c] Vector al que se le va a calcular su peso.
    \end{description}

    \textsc{Ejemplos}
    \begin{lstlisting}[gobble=4]
        % TODO
    \end{lstlisting}
\end{description}