\chapter[Implementación en SageMath de los algoritmos genéticos]{Implementación en SageMath de los algoritmos genéticos}
\label{annex:sage-geneticos}

En este anexo describimos la documentación de las funciones desarrolladas para implementar en SageMath los algoritmos genéticos GGA y CHC. También se describen las funciones auxiliares desarrolladas. 

El código desarrollado se encuentra en
\begin{center}
\url{https://github.com/paula1999/TFG/tree/main/src}.
\end{center}

\section{Algoritmo genético GGA}

\begin{description}[leftmargin=1em, font=\normalfont\ttfamily, style=nextline]
  \item[GGA(G, N, pc, max\_reinit, min\_fitness)] Obtiene una permutación de la matriz \texttt{G} cuya matriz escalonada tiene una fila con peso \texttt{min\_fitness}.

  \textsc{Argumentos}
  \begin{description}[font=\normalfont\ttfamily]
    \item[G] Matriz conformada por la clave pública de un criptosistema de McEliece y una fila con el mensaje encriptado.
    \item[N] Tamaño de la población.
    \item[pc] Probabilidad de cruce.
    \item[max\_reinit] Número máximo de evaluaciones de las soluciones que no producen mejora en el fitness de la mejor solución encontrada.
    \item[min\_fitness] Peso del vector error. 
  \end{description}

  \textsc{Salida}
  \begin{description}[font=\normalfont\ttfamily]
    \item[] Tupla formada por una permutación el mínimo fitness que produce.
  \end{description}

  \textsc{Ejemplos}
  \begin{lstlisting}[gobble=4]
    sage: q = 4
    sage: n = 14
    sage: t = 2
    sage: F = GF(2)
    sage: L = GF(2^q)
    sage: a = L.gen()
    sage: R.<x> = L[]
    sage: g = x^2 + (a^3 + a^2 + a)*x + a^2 + a + 1
    sage: min_fitness = floor(t/2)
    sage: ME = McEliece(n, q, g)
    sage: G = ME.get_public_key()
    sage: message = vector(F, (0, 0, 1, 1, 1, 0))
    sage: encrypted_message = ME.encrypt(message)
    sage: G = G.stack(encrypted_message)
    sage: GGA(G, 400, 0.7, 100000, min_fitness)
    > ((1, 11, 9, 14, 10, 3, 8, 6, 12, 7, 4, 5, 2, 13), 1)
  \end{lstlisting}
\end{description}

\section{Algoritmo genético CHC}

\begin{description}[leftmargin=1em, font=\normalfont\ttfamily, style=nextline]
  \item[CHC(G, N, tau, min\_fitness)] Obtiene una permutación de la matriz \texttt{G} cuya matriz escalonada tiene una fila con peso \texttt{min\_fitness}.

  \textsc{Argumentos}
  \begin{description}[font=\normalfont\ttfamily]
    \item[G] Matriz conformada por la clave pública de un criptosistema de McEliece y una fila con el mensaje encriptado.
    \item[N] Tamaño de la población.
    \item[tau] Tasa de actualización. $\tau \in [0,1]$.
    \item[min\_fitness] Peso del vector error. 
  \end{description}

  \textsc{Salida}
  \begin{description}[font=\normalfont\ttfamily]
    \item[] Tupla formada por una permutación el mínimo fitness que produce.
  \end{description}

  \textsc{Ejemplos}
  \begin{lstlisting}[gobble=4]
    sage: q = 4
    sage: n = 14
    sage: t = 2
    sage: F = GF(2)
    sage: L = GF(2^q)
    sage: a = L.gen()
    sage: R.<x> = L[]
    sage: g = x^2 + (a^3 + a^2 + a)*x + a^2 + a + 1
    sage: min_fitness = floor(t/2)
    sage: ME = McEliece(n, q, g)
    sage: G = ME.get_public_key()
    sage: message = vector(F, (0, 0, 1, 1, 1, 0))
    sage: encrypted_message = ME.encrypt(message)
    sage: G = G.stack(encrypted_message)
    sage: CHC(G, 400, 0.8, min_fitness)
    > ((9, 13, 7, 10, 11, 2, 8, 6, 14, 12, 5, 3, 4, 1), 1)
  \end{lstlisting}
\end{description}

\section{Funciones auxiliares}

\begin{description}[leftmargin=1em, font=\normalfont\ttfamily, style=nextline]
  \item[fitness\_vector(v)] Calcula el fitness de un vector \texttt{v}.

  \textsc{Argumentos}
  \begin{description}[font=\normalfont\ttfamily]
    \item[v] Vector binario al que se le va a calcular el fitness. 
  \end{description}

  \textsc{Salida}
  \begin{description}[font=\normalfont\ttfamily]
    \item[] Fitness del vector binario \texttt{v}.
  \end{description}

  \textsc{Ejemplos}
  \begin{lstlisting}[gobble=4]
    sage: fitness_vector([0, 1, 1, 0, 0, 0, 1])
    > 3
  \end{lstlisting}

  \item[fitness\_matrix(M)] Calcula el fitness mínimo de las filas de la matriz \texttt{M}.

  \textsc{Argumentos}
  \begin{description}[font=\normalfont\ttfamily]
    \item[M] Matriz binaria de la que queremos calcular su fitness mínimo.
  \end{description}

  \textsc{Salida}
  \begin{description}[font=\normalfont\ttfamily]
    \item[] Fitness mínimo de las filas de la matriz \texttt{M}.
  \end{description}

  \textsc{Ejemplos}
  \begin{lstlisting}[gobble=4]
    sage: M = matrix(GF(2), [[0, 1], [1, 1]])
    sage: fitness_matrix(M)
    > 1
  \end{lstlisting}

  \item[permutation\_matrix(x)] Calcula la matriz asociada a la permutación \texttt{x}.

  \textsc{Argumentos}
  \begin{description}[font=\normalfont\ttfamily]
    \item[x] Permutación que se le aplica a la matriz \texttt{M}.
  \end{description}

  \textsc{Salida}
  \begin{description}[font=\normalfont\ttfamily]
    \item[] La matriz de permutaciones asociada a \texttt{x}.
  \end{description}

  \textsc{Ejemplos}
  \begin{lstlisting}[gobble=4]
    sage: x = Permutation([3, 2, 1])
    sage: permutation_matrix(x)
    > [0 0 1]
      [0 1 0]
      [1 0 0]
  \end{lstlisting}

  \item[fitness\_permutation(M, x)] Calcula el fitness mínimo de las filas de la forma escalonada reducida de la matriz \texttt{M} permutada por la permutación \texttt{x}.

  \textsc{Argumentos}
  \begin{description}[font=\normalfont\ttfamily]
    \item[M] Matriz binaria de la que queremos calcular su fitness mínimo.
    \item[x] Permutación que se le aplica a la matriz \texttt{M}.
  \end{description}

  \textsc{Salida}
  \begin{description}[font=\normalfont\ttfamily]
    \item[] Fitness mínimo de las filas de la forma escalonada reducida de la matriz \texttt{M} permutada por la permutación \texttt{x}.
  \end{description}

  \textsc{Ejemplos}
  \begin{lstlisting}[gobble=4]
    sage: M = matrix(GF(2), [[1, 0, 0], [0, 1, 1], [1, 0, 1]])
    sage: x = Permutation([3,2,1])
    sage: fitness_permutation(M, x)
    > 1
  \end{lstlisting}

  \item[random\_sol(n)] Calcula una solución aleatoria de tamaño \texttt{n}.

  \textsc{Argumentos}
  \begin{description}[font=\normalfont\ttfamily]
    \item[n] Tamaño de la solución.
  \end{description}

  \textsc{Salida}
  \begin{description}[font=\normalfont\ttfamily]
    \item[] Un vector aleatorio de tamaño \texttt{n}.
  \end{description}

  \textsc{Ejemplos}
  \begin{lstlisting}[gobble=4]
    sage: random_sol(5)
    > [1, 4, 5, 2, 3]
  \end{lstlisting}

  \item[crossover(p1, p2)] Operador cruce entre dos padres \texttt{p1} y \texttt{p2}.

  \textsc{Argumentos}
  \begin{description}[font=\normalfont\ttfamily]
    \item[p1] Vector de dimensión \texttt{n}.
    \item[p2] Vector de dimensión \texttt{n}.
  \end{description}

  \textsc{Salida}
  \begin{description}[font=\normalfont\ttfamily]
    \item[] Un vector aleatorio de tamaño \texttt{n}.
  \end{description}

  \textsc{Ejemplos}
  \begin{lstlisting}[gobble=4]
    sage: crossover([1, 3, 2], [3, 1, 2])
    > ([3, 2, 1], [2, 1, 3])
  \end{lstlisting}

  \item[mutation(p)] Operador mutación de \texttt{p}.

  \textsc{Argumentos}
  \begin{description}[font=\normalfont\ttfamily]
    \item[p] Individuo que se va a mutar.
  \end{description}

  \textsc{Salida}
  \begin{description}[font=\normalfont\ttfamily]
    \item[] Un vector mutado.
  \end{description}

  \textsc{Ejemplos}
  \begin{lstlisting}[gobble=4]
    sage: crossover([1, 3, 2], [3, 1, 2])
    > ([3, 2, 1], [2, 1, 3])
  \end{lstlisting}

  \item[distance(x, y)] Calcula la distancia de Hamming entre \texttt{x} e \texttt{y}.

  \textsc{Argumentos}
  \begin{description}[font=\normalfont\ttfamily]
    \item[x] Vector de tamaño \texttt{n}.
    \item[y] Vector de tamaño \texttt{n}.
  \end{description}

  \textsc{Salida}
  \begin{description}[font=\normalfont\ttfamily]
    \item[] Distancia Hamming entre los vectores \texttt{x} e \texttt{y}.
  \end{description}

  \textsc{Ejemplos}
  \begin{lstlisting}[gobble=4]
    sage: distance([1, 3, 2], [3, 1, 2])
    > 2
  \end{lstlisting}

  \item[actualizar(P, tau)] Calcula el decremento del cruce como un porcentaje \texttt{tau} de la máxima distancia de Hamming entre los individuos de la población \texttt{P}.

  \textsc{Argumentos}
  \begin{description}[font=\normalfont\ttfamily]
    \item[P] Población.
    \item[tau] Tasa de actualización. $\tau \in [0,1]$.
  \end{description}

  \textsc{Salida}
  \begin{description}[font=\normalfont\ttfamily]
    \item[] Tupla formada por la distancia y el decremento del cruce.
  \end{description}

  \textsc{Ejemplos}
  \begin{lstlisting}[gobble=4]
    sage: P = matrix([[1, 4, 2, 3], [2, 1, 3, 4]])
    sage: tau = 0.8
    sage: actualizar(P, tau)
    > (1.33333333333333, 3.20000000000000)
  \end{lstlisting}
\end{description}
